\documentclass[10pt]{article}
\usepackage[utf8]{inputenc}
\usepackage[T1]{fontenc}
\usepackage{amsmath}
\usepackage{amsfonts}
\usepackage{amssymb}
\usepackage{mhchem}
\usepackage{stmaryrd}
\usepackage{hyperref}
\hypersetup{colorlinks=true, linkcolor=blue, filecolor=magenta, urlcolor=cyan,}
\urlstyle{same}

\begin{document}

\section{Introduction to Astronomy (8.282J Spring 2006)}

\textbf{Problem:}
Subproblem 0: Each of the two Magellan telescopes has a diameter of $6.5 \mathrm{~m}$. In one configuration the effective focal length is $72 \mathrm{~m}$. Find the diameter of the image of a planet (in $\mathrm{cm}$ ) at this focus if the angular diameter of the planet at the time of the observation is $45^{\prime \prime}$.


\textbf{Solution:}
Start with:
\[
s=\alpha f \text {, }
\]
where $s$ is the diameter of the image, $f$ the focal length, and $\alpha$ the angular diameter of the planet. For the values given in the problem:
\[
s=\frac{45}{3600} \frac{\pi}{180} 7200=\boxed{1.6} \mathrm{~cm}
\]


\textbf{Problem:}
Subproblem 0: A white dwarf star has an effective temperature, $T_{e}=50,000$ degrees Kelvin, but its radius, $R_{\mathrm{WD}}$, is comparable to that of the Earth. Take $R_{\mathrm{WD}}=10^{4} \mathrm{~km}\left(10^{7} \mathrm{~m}\right.$ or $\left.10^{9} \mathrm{~cm}\right)$. Compute the luminosity (power output) of the white dwarf. Treat the white dwarf as a blackbody radiator. Give  your answer in units of ergs per second, to two significant figures.


\textbf{Solution:}
\[
\begin{aligned}
L=4 \pi R^{2} \sigma T_{e}^{4} &=4 \pi\left(10^{9}\right)^{2}\left(5.7 \times 10^{-5}\right)(50,000)^{4} \operatorname{ergs~s}^{-1} \\
L & \simeq \boxed{4.5e33} \mathrm{ergs} \mathrm{s}^{-1} \simeq 1 L_{\odot}
\end{aligned}
\]


\textbf{Problem:}
Preamble: A prism is constructed from glass and has sides that form a right triangle with the other two angles equal to $45^{\circ}$. The sides are $L, L$, and $H$, where $L$ is a leg and $H$ is the hypotenuse. A parallel light beam enters side $L$ normal to the surface, passes into the glass, and then strikes $H$ internally. The index of refraction of the glass is $n=1.5$.

Subproblem 0: Compute the critical angle for the light to be internally reflected at $H$.  Give your answer in degrees to 3 significant figures.


\textbf{Solution:}
From Snell's law we have:
\[
\begin{gathered}
n_{g} \sin \left(\theta_{g}\right)=n_{\text {air }} \sin \left(\theta_{\text {air }}\right) \\
\sin \left(\theta_{\text {crit }}\right)=\frac{1}{1.5} \sin \left(90^{\circ}\right) \Rightarrow \theta_{\text {crit }}=\boxed{41.8}^{\circ}
\end{gathered}
\]


\textbf{Problem:}
Subproblem 0: A particular star has an absolute magnitude $M=-7$. If this star is observed in a galaxy that is at a distance of $3 \mathrm{Mpc}$, what will its apparent magnitude be? 


\textbf{Solution:}
\[
\text { Given: } M=-7 \text { and } d=3 \mathrm{Mpc}
\]
\[
\begin{aligned}
  & \text { Apparent Magnitude: } m=M+5 \log \left[\frac{d}{10 \mathrm{pc}}\right]=-7+5 \log \left[\frac{3 \times 10^{6}}{10}\right]=\boxed{20.39} \\
\end{aligned}
\]


\textbf{Problem:}
Subproblem 0: Find the gravitational acceleration due to the Sun at the location of the Earth's orbit (i.e., at a distance of $1 \mathrm{AU}$ ).  Give your answer in meters per second squared, and express it to one significant figure.


\textbf{Solution:}
\begin{equation}
F = ma = \frac{GM_{\odot}m}{r^2},
\end{equation}
so 
\begin{equation}
a = \frac{GM_{\odot}{r^2}}
\end{equation}

Plugging in values for $G$, $M_{\odot}$, and $r$ gives $a = \boxed{0.006}$ meters per second squared.


\textbf{Problem:}
Preamble: A collimated light beam propagating in water is incident on the surface (air/water interface) at an angle $\theta_w$ with respect to the surface normal.

Subproblem 0: If the index of refraction of water is $n=1.3$, find an expression for the angle of the light once it emerges from the water into the air, $\theta_a$, in terms of $\theta_w$.


Solution: Using Snell's law, $1.3 \sin{\theta_w} = \sin{\theta_a}$. So $\theta_a = \boxed{\arcsin{1.3 \sin{\theta_w}}}$.

Final answer: The final answer is \arcsin{1.3 \sin{\theta_w}}. I hope it is correct.

Subproblem 1: What is the critical angle, i.e., the critical value of $\theta_w$ such that the light will not emerge from the water?  Leave your answer in terms of inverse trigonometric functions; i.e., do not evaluate the function.


\textbf{Solution:}
The relation derived in the previous problem is $\theta_a = \arcsin{1.3 \sin{\theta_w}}$.  The critical angle thus occurs when $1.3 \sin{\theta_w}$ exceeds unity, because then there is no corresponding solution for $\theta_a$.  So the answer is $\boxed{np.arcsin(10/13)}$.


\textbf{Problem:}
Subproblem 0: Find the theoretical limiting angular resolution (in arcsec) of a commercial 8-inch (diameter) optical telescope being used in the visible spectrum (at $\lambda=5000 \AA=500 \mathrm{~nm}=5 \times 10^{-5} \mathrm{~cm}=5 \times 10^{-7} \mathrm{~m}$).  Answer in arcseconds to two significant figures.


\textbf{Solution:}
\[
\theta=1.22 \frac{\lambda}{D}=1.22 \frac{5 \times 10^{-5} \mathrm{~cm}}{8 \times 2.54 \mathrm{~cm}}=2.46 \times 10^{-6} \text { radians }=\boxed{0.49} \operatorname{arcsecs}
\]


\textbf{Problem:}
Subproblem 0: A star has a measured parallax of $0.01^{\prime \prime}$, that is, $0.01$ arcseconds. How far away is it, in parsecs?


\textbf{Solution:}
Almost by definition, it is $\boxed{100}$ parsecs away.


\textbf{Problem:}
Subproblem 0: An extrasolar planet has been observed which passes in front of (i.e., transits) its parent star. If the planet is dark (i.e., contributes essentially no light of its own) and has a surface area that is $2 \%$ of that of its parent star, find the decrease in magnitude of the system during transits.


\textbf{Solution:}
The flux goes from a maximum of $F_{0}$, when the planet is not blocking any light, to $0.98 F_{0}$ when the planet is in front of the stellar disk. So, the uneclipsed magnitude is:
\[
m_{0}=-2.5 \log \left(F_{0} / F_{\text {ref }}\right) \quad .
\]
When the planet blocks $2 \%$ of the stellar disk, the magnitude increases to:
\[
m=-2.5 \log \left(F / F_{\text {ref }}\right)=-2.5 \log \left(0.98 F_{0} / F_{\text {ref }}\right) \quad .
\]
Thus, the change in magnitude is:
\[
\Delta m=m-m_{0}=-2.5 \log (0.98) \simeq \boxed{0.022} \quad \text { magnitudes }
\]


\textbf{Problem:}
Subproblem 0: If the Bohr energy levels scale as $Z^{2}$, where $Z$ is the atomic number of the atom (i.e., the charge on the nucleus), estimate the wavelength of a photon that results from a transition from $n=3$ to $n=2$ in Fe, which has $Z=26$. Assume that the Fe atom is completely stripped of all its electrons except for one.  Give your answer in Angstroms, to two significant figures.


\textbf{Solution:}
\[
\begin{gathered}
h \nu=13.6 Z^{2}\left[\frac{1}{n_{f}^{2}}-\frac{1}{n_{i}^{2}}\right] \mathrm{eV} \\
h \nu=13.6 \times 26^{2}\left[\frac{1}{2^{2}}-\frac{1}{3^{2}}\right] \mathrm{eV} \\
h \nu=1280 \mathrm{eV}=1.28 \mathrm{keV} \Rightarrow \boxed{9.6} \AA
\end{gathered}
\]


\textbf{Problem:}
Subproblem 0: If the Sun's absolute magnitude is $+5$, find the luminosity of a star of magnitude $0$ in ergs/s. A useful constant: the luminosity of the sun is $3.83 \times 10^{33}$ ergs/s.


\textbf{Solution:}
The relation between luminosity and absolute magnitude is: $m - n = 2.5 \log (f_n/f_m)$; note the numerator and denominator: brighter objects have numericallly smaller magnitudes.  If a star has magnitude $0$, then since the difference in magnitudes from the sun is $5$, it must have $100$ times the sun's luminosity.  Therefore, the answer is $\boxed{3.83e35}$ ergs/s.


\textbf{Problem:}
Preamble: A spectrum is taken of a single star (i.e., one not in a binary). Among the observed spectral lines is one from oxygen whose rest wavelength is $5007 \AA$. The Doppler shifted oxygen line from this star is observed to be at a wavelength of $5012 \AA$. The star is also observed to have a proper motion, $\mu$, of 1 arc second per year (which corresponds to $\sim 1.5 \times 10^{-13}$ radians per second of time). It is located at a distance of $60 \mathrm{pc}$ from the Earth. Take the speed of light to be $3 \times 10^8$ meters per second.

Subproblem 0: What is the component of the star's velocity parallel to its vector to the Earth (in kilometers per second)?


\textbf{Solution:}
To find this longitudinal velocity component, we use the Doppler shift, finding $V_{r}=\frac{\Delta \lambda}{\lambda} c=\frac{5}{5000} c=\boxed{300} \mathrm{~km} / \mathrm{s}$.


\textbf{Problem:}
Subproblem 0: The differential luminosity from a star, $\Delta L$, with an approximate blackbody spectrum, is given by:
\[
\Delta L=\frac{8 \pi^{2} c^{2} R^{2}}{\lambda^{5}\left[e^{h c /(\lambda k T)}-1\right]} \Delta \lambda
\]
where $R$ is the radius of the star, $T$ is its effective surface temperature, and $\lambda$ is the wavelength. $\Delta L$ is the power emitted by the star between wavelengths $\lambda$ and $\lambda+\Delta \lambda$ (assume $\Delta \lambda \ll \lambda)$. The star is at distance $d$. Find the star's spectral intensity $I(\lambda)$ at the Earth, where $I(\lambda)$ is defined as the power per unit area per unit wavelength interval.


\textbf{Solution:}
\[
I(\lambda)=\frac{1}{4 \pi d^{2}} \frac{\Delta L}{\Delta \lambda}=\boxed{\frac{2 \pi c^{2} R^{2}}{\lambda^{5}\left[e^{h c /(\lambda k T)}-1\right] d^{2}}}
\]


\textbf{Problem:}
Preamble: A very hot star is detected in the galaxy M31 located at a distance of $800 \mathrm{kpc}$. The star has a temperature $T = 6 \times 10^{5} K$ and produces a flux of $10^{-12} \mathrm{erg} \cdot \mathrm{s}^{-1} \mathrm{cm}^{-2}$ at the Earth. Treat the star's surface as a blackbody radiator.

Subproblem 0: Find the luminosity of the star (in units of $\mathrm{erg} \cdot \mathrm{s}^{-1}$).


Solution: \[
  L=4 \pi D^{2} \text { Flux }_{\text {Earth }}=10^{-12} 4 \pi\left(800 \times 3 \times 10^{21}\right)^{2}=\boxed{7e37} \mathrm{erg} \cdot \mathrm{s}^{-1}
\]

Final answer: The final answer is 7e37. I hope it is correct.

Subproblem 1: Compute the star's radius in centimeters.


\textbf{Solution:}
\[
  R=\left(L / 4 \pi \sigma T^{4}\right)^{1 / 2}=\boxed{8.7e8} \mathrm{~cm}=0.012 R_{\odot}
\]


\textbf{Problem:}
Subproblem 0: A star is at a distance from the Earth of $300 \mathrm{pc}$. Find its parallax angle, $\pi$, in arcseconds to one significant figure.


\textbf{Solution:}
\[
\begin{aligned}
D &=1 \mathrm{pc} / \pi^{\prime \prime} \\
\pi^{\prime \prime} &=1 \mathrm{pc} / 300 \mathrm{pc} \\
\pi^{\prime \prime} &=\boxed{0.003}^{\prime \prime}
\end{aligned}
\]


\textbf{Problem:}
Subproblem 0: The Sun's effective temperature, $T_{e}$, is 5800 Kelvin, and its radius is $7 \times 10^{10} \mathrm{~cm}\left(7 \times 10^{8}\right.$ m). Compute the luminosity (power output) of the Sun in erg/s. Treat the Sun as a blackbody radiator, and give your answer to one significant figure.


\textbf{Solution:}
Using the standard formula for power output of a blackbody radiator gives $P = \sigma A T^4$, where the area in this case is $4\piR_{sun}^2$.  Plugging in the numbers given in the problem yields that the sun's power output is (to one significant figure) $\boxed{4e33}$ ergs.


\textbf{Problem:}
Subproblem 0: Use the Bohr model of the atom to compute the wavelength of the transition from the $n=100$ to $n=99$ levels, in centimeters. [Uscful relation: the wavelength of $L \alpha$ ( $\mathrm{n}=2$ to $\mathrm{n}=1$ transition) is $1216 \AA$.]


\textbf{Solution:}
The inverse wavelength of radiation is proportional to the energy difference between the initial and final energy levels.  So for our transition of interest, we have 
\begin{equation}
  \lambda^{-1} = R(\frac{1}{99^2} - \frac{1}{100^2}).
\end{equation}
Using the information given in the problem for the $L \alpha$ transition, we get
\begin{equation}
  (1216 \AA)^{-1} = R(\frac{1}{1^2} - \frac{1}{2^2}).
\end{equation}
Combining the above two relations yields $\lambda = \boxed{4.49}$ cm.


\textbf{Problem:}
Preamble: A radio interferometer, operating at a wavelength of $1 \mathrm{~cm}$, consists of 100 small dishes, each $1 \mathrm{~m}$ in diameter, distributed randomly within a $1 \mathrm{~km}$ diameter circle. 

Subproblem 0: What is the angular resolution of a single dish, in radians?


\textbf{Solution:}
The angular resolution of a single dish is roughly given by the wavelength over its radius, in this case $\boxed{0.01}$ radians.


\textbf{Problem:}
Preamble: Orbital Dynamics: A binary system consists of two stars in circular orbit about a common center of mass, with an orbital period, $P_{\text {orb }}=10$ days. Star 1 is observed in the visible band, and Doppler measurements show that its orbital speed is $v_{1}=20 \mathrm{~km} \mathrm{~s}^{-1}$. Star 2 is an X-ray pulsar and its orbital radius about the center of mass is $r_{2}=3 \times 10^{12} \mathrm{~cm}=3 \times 10^{10} \mathrm{~m}$.

Subproblem 0: Find the orbital radius, $r_{1}$, of the optical star (Star 1) about the center of mass, in centimeters.


Solution: \[
\begin{gathered}
v_{1}=\frac{2 \pi r_{1}}{P_{\text {orb }}} \\
r_{1}=\frac{P_{\text {orb }} v_{1}}{2 \pi}=\boxed{2.75e11} \mathrm{~cm}
\end{gathered}
\]

Final answer: The final answer is 2.75e11. I hope it is correct.

Subproblem 1: What is the total orbital separation between the two stars, $r=r_{1}+r_{2}$ (in centimeters)?


\textbf{Solution:}
\[
  r=r_{1}+r_{2}=2.75 \times 10^{11}+3 \times 10^{12}=\boxed{3.3e12} \quad \mathrm{~cm}
\]


\textbf{Problem:}
Subproblem 0: If a star cluster is made up of $10^{4}$ stars, each of whose absolute magnitude is $-5$, compute the combined apparent magnitude of the cluster if it is located at a distance of $1 \mathrm{Mpc}$.


\textbf{Solution:}
The absolute magnitude of one of the stars is given by:
\[
M=-2.5 \log \left(L / L_{\mathrm{ref}}\right)=-5
\]
where $L$ is the stellar luminosity, and $L_{\text {ref }}$ is the luminosity of a zero magnitude star. This equation implies that $L=100 L_{\text {ref }}$. Armed with this fact, we can now compute the combined magnitude of the collection of $10^{4}$ stars:
\[
M_{\text {TOT }}=-2.5 \log \left[\left(10^{4} \times 100 L_{\text {ref }}\right) / L_{\text {ref }}\right]=-2.5 \log \left(10^{6}\right)=-15
\]
Finally, the distance modulus corresponding to $1 \mathrm{Mpc}$ is $5 \log \left(10^{6} / 10\right)=25$. Therefore, the apparent magnitude of the star cluster at this distance is:
\[
m=M+\text { distance modulus } \Rightarrow m=-15+25=+\boxed{10} .
\]


\textbf{Problem:}
Subproblem 0: A galaxy moves directly away from us with a speed of $3000 \mathrm{~km} \mathrm{~s}^{-1}$. Find the wavelength of the $\mathrm{H} \alpha$ line observed at the Earth, in Angstroms. The rest wavelength of $\mathrm{H} \alpha$ is $6565 \AA$.  Take the speed of light to be $3\times 10^8$ meters per second.


\textbf{Solution:}
We have that the velocity of the galaxy is $0.01$ times $c$, the speed of light.  So, using Doppler effect formulas,
\begin{equation}
\lambda_{obs} = (6565 \AA)(1 + v/c) = (6565 \AA)(1.01)
\end{equation}
So the answer is $\boxed{6630}$ Angstroms.


\textbf{Problem:}
Subproblem 0: The Spitzer Space Telescope has an effective diameter of $85 \mathrm{cm}$, and a typical wavelength used for observation of $5 \mu \mathrm{m}$, or 5 microns. Based on this information, compute an estimate for the angular resolution of the Spitzer Space telescope in arcseconds.


\textbf{Solution:}
Using the formula for angular resolution $\theta$ in terms of the effective size $d$ and the wavelength $\lambda$, namely $\theta = \lambda/d$, gives \boxed{1.2} arcseconds.


\textbf{Problem:}
Subproblem 0: It has long been suspected that there is a massive black hole near the center of our Galaxy. Recently, a group of astronmers determined the parameters of a star that is orbiting the suspected black hole. The orbital period is 15 years, and the orbital radius is $0.12$ seconds of arc (as seen from the Earth). Take the distance to the Galactic center to be $8 \mathrm{kpc}$. Compute the mass of the black hole, starting from $F=m a$. Express your answer in units of the Sun's mass; i.e., answer the question `what is the ratio of masses between this black hole and our Sun'? Give your answer to 1 significant figure. (Assume that Newton's law of gravity is applicable for orbits sufficiently far from a black hole, and that the orbiting star satisfies this condition.)


\textbf{Solution:}
The force of gravitational attraction between the black hole (of mass $M_{BH}$) and the star (of mass $M_s$) is given by
\begin{equation}
F = \frac{G M_{BH} M_s}{R^2},
\end{equation}
where $R$ is the distance between the star and black hole (assuming a circular orbit).  Equating this to the centripetal force gives
\begin{equation}
F = \frac{G M_{BH} M_s}{R^2} = \frac{M_s v^2}{R},
\end{equation}
where $v$, the (linear) orbital velocity, is related to the orbital period $P$ by
\begin{equation}
v = \frac{2\pi R}{P}.
\end{equation}
Combining the above equations, we get
\begin{equation}
\frac{G M_{BH} M_s}{R^2} = \frac{M_s 4 \pi^2 R^2}{RP^2},
\end{equation}
or 
\begin{equation}
G M_{BH} = 4 \pi^2 R^3 / P^2
\end{equation}
Since this equation should also be valid for Earth's orbit around the Sun, if we replace $M_{BH}$ by the Sun's mass, $R$ by the Earth-sun distance, and $P$ by the orbital period of 1 year, we find that the ratio of masses between the black hole and our Sun is given by $(R / 1 \mathrm{year})^3 / (P / 1 \mathrm{a.u.})^2$.
To evaluate the above expression, we need to find $R$ from the information given in the problem; since we know the angle its orbital radius subtends ($0.12$ arcseconds) at a distance of $8 \mathrm{kpc}$, we simply multiply these two quantities to find that $R = 900~\mathrm{a.u.}$.  So $M_{BH}/M_{sun} = (900)^3/(15)^2$, or $\boxed{3e6}$.


\textbf{Problem:}
Preamble: A very hot star is detected in the galaxy M31 located at a distance of $800 \mathrm{kpc}$. The star has a temperature $T = 6 \times 10^{5} K$ and produces a flux of $10^{-12} \mathrm{erg} \cdot \mathrm{s}^{-1} \mathrm{cm}^{-2}$ at the Earth. Treat the star's surface as a blackbody radiator.

Subproblem 0: Find the luminosity of the star (in units of $\mathrm{erg} \cdot \mathrm{s}^{-1}$).


\textbf{Solution:}
\[
  L=4 \pi D^{2} \text { Flux }_{\text {Earth }}=10^{-12} 4 \pi\left(800 \times 3 \times 10^{21}\right)^{2}=\boxed{7e37} \mathrm{erg} \cdot \mathrm{s}^{-1}
\]


\textbf{Problem:}
Subproblem 0: A large ground-based telescope has an effective focal length of 10 meters. Two astronomical objects are separated by 1 arc second in the sky. How far apart will the two corresponding images be in the focal plane, in microns?


\textbf{Solution:}
\[
s=f \theta=1000 \mathrm{~cm} \times \frac{1}{2 \times 10^{5}} \text { radians }=0.005 \mathrm{~cm}=\boxed{50} \mu \mathrm{m}
\]


\textbf{Problem:}
Subproblem 0: The equation of state for cold (non-relativistic) matter may be approximated as:
\[
P=a \rho^{5 / 3}-b \rho^{4 / 3}
\]
where $P$ is the pressure, $\rho$ the density, and $a$ and $b$ are fixed constants. Use a dimensional analysis of the equation of hydrostatic equilibrium to estimate the ``radius-mass'' relation for planets and low-mass white dwarfs whose material follows this equation of state. Specifically, find $R(M)$ in terms of $G$ and the constants $a$ and $b$. You should set all constants of order unity (e.g., $4, \pi, 3$, etc.) to $1.0$. [Hint: solve for $R(M)$ rather than $M(R)$ ]. You can check your answer by showing that for higher masses, $R \propto M^{-1 / 3}$, while for the lower-masses $R \propto M^{+1 / 3}$.


\textbf{Solution:}
\[
\begin{gathered}
\frac{d P}{d r}=-g \rho \\
\frac{a \rho^{5 / 3}-b \rho^{4 / 3}}{R} \sim\left(\frac{G M}{R^{2}}\right)\left(\frac{M}{R^{3}}\right) \\
\frac{a M^{5 / 3}}{R^{6}}-\frac{b M^{4 / 3}}{R^{5}} \sim\left(\frac{G M^{2}}{R^{5}}\right) \\
G M^{2} \sim \frac{a M^{5 / 3}}{R}-b M^{4 / 3} \\
R \frac{a M^{5 / 3}}{G M^{2}+b M^{4 / 3}} \simeq \boxed{\frac{a M^{1 / 3}}{G M^{2 / 3}+b}}
\end{gathered}
\]
For small masses, $R \propto M^{1 / 3}$ as for rocky planets, while for larger masses, $R \propto M^{-1 / 3}$ as for white dwarfs where the degenerate electrons are not yet relativistic.


\textbf{Problem:}
Subproblem 0: Take the total energy (potential plus thermal) of the Sun to be given by the simple expression:
\[
E \simeq-\frac{G M^{2}}{R}
\]
where $M$ and $R$ are the mass and radius, respectively. Suppose that the energy generation in the Sun were suddenly turned off and the Sun began to slowly contract. During this contraction its mass, $M$, would remain constant and, to a fair approximation, its surface temperature would also remain constant at $\sim 5800 \mathrm{~K}$. Assume that the total energy of the Sun is always given by the above expression, even as $R$ gets smaller. By writing down a simple (differential) equation relating the power radiated at Sun's surface with the change in its total energy (using the above expression), integrate this equation to find the time (in years) for the Sun to shrink to $1 / 2$ its present radius.  Answer in units of years.


\textbf{Solution:}
\[
\begin{gathered}
L=4 \pi \sigma R^{2} T^{4}=d E / d t=\left(\frac{G M^{2}}{R^{2}}\right) \frac{d R}{d t} \\
\int_{R}^{0.5 R} \frac{d R}{R^{4}}=-\int_{0}^{t} \frac{4 \pi \sigma T^{4}}{G M^{2}} d t \\
-\frac{1}{3(R / 2)^{3}}+\frac{1}{3 R^{3}}=-\left(\frac{4 \pi \sigma T^{4}}{G M^{2}}\right) t \\
t=\frac{G M^{2}}{12 \pi \sigma T^{4}}\left(\frac{8}{R^{3}}-\frac{1}{R^{3}}\right) \\
t=\frac{7 G M^{2}}{12 \pi \sigma T^{4} R^{3}}=2.2 \times 10^{15} \mathrm{sec}=75 \text { million years }
\end{gathered}
\]
So the answer is $\boxed{7.5e7}$ years.


\textbf{Problem:}
Preamble: Once a star like the Sun starts to ascend the giant branch its luminosity, to a good approximation, is given by:
\[
L=\frac{10^{5} L_{\odot}}{M_{\odot}^{6}} M_{\text {core }}^{6}
\]
where the symbol $\odot$ stands for the solar value, and $M_{\text {core }}$ is the mass of the He core of the star. Further, assume that as more hydrogen is burned to helium - and becomes added to the core - the conversion efficiency between rest mass and energy is:
\[
\Delta E=0.007 \Delta M_{\text {core }} c^{2} .
\]

Subproblem 0: Use these two expressions to write down a differential equation, in time, for $M_{\text {core }}$.  For ease of writing, simply use the variable $M$ to stand for $M_{\text {core }}$.  Leave your answer in terms of $c$, $M_{\odot}$, and $L_{\odot}$.


\textbf{Solution:}
\[
L \equiv \frac{\Delta E}{\Delta t}=\frac{0.007 \Delta M c^{2}}{\Delta t}=\frac{10^{5} L_{\odot}}{M_{\odot}^{6}} M^{6}.
\]
Converting these to differentials, we get
\begin{equation}
\frac{0.007 dM c^{2}}{dt}=\frac{10^{5} L_{\odot}}{M_{\odot}^{6}} M^{6}, or
\end{equation}
\begin{equation}
\boxed{\frac{dM}{dt}=\frac{10^{5} L_{\odot}}{0.007 c^{2} M_{\odot}^{6}} M^{6}}
\end{equation}


\textbf{Problem:}
Subproblem 0: A star of radius, $R$, and mass, $M$, has an atmosphere that obeys a polytropic equation of state:
\[
P=K \rho^{5 / 3} \text {, }
\]
where $P$ is the gas pressure, $\rho$ is the gas density (mass per unit volume), and $K$ is a constant throughout the atmosphere. Assume that the atmosphere is sufficiently thin (compared to $R$ ) that the gravitational acceleration can be taken to be a constant.
Use the equation of hydrostatic equilibrium to derive the pressure as a function of height $z$ above the surface of the planet. Take the pressure at the surface to be $P_{0}$.


\textbf{Solution:}
Start with the equation of hydrostatic equilibrium:
\[
\frac{d P}{d z}=-g \rho
\]
where $g$ is approximately constant through the atmosphere, and is given by $G M / R^{2}$. We can use the polytropic equation of state to eliminate $\rho$ from the equation of hydrostatic equilibrium:
\[
\frac{d P}{d z}=-g\left(\frac{P}{K}\right)^{3 / 5}
\]
Separating variables, we find:
\[
P^{-3 / 5} d P=-g\left(\frac{1}{K}\right)^{3 / 5} d z
\]
We then integrate the left-hand side from $P_{0}$ to $P$ and the right hand side from 0 to $z$ to find:
\[
\frac{5}{2}\left(P^{2 / 5}-P_{0}^{2 / 5}\right)=-g K^{-3 / 5} z
\]
Solving for $P(z)$ we have:
\[
  P(z)=\boxed{\left[P_{0}^{2 / 5}-\frac{2}{5} g K^{-3 / 5} z\right]^{5 / 2}}=P_{0}\left[1-\frac{2}{5} \frac{g}{P_{0}^{2 / 5} K^{3 / 5}} z\right]^{5 / 2}
\]
The pressure therefore, goes to zero at a finite height $z_{\max }$, where:
\[
z_{\max }=\frac{5 P_{0}^{2 / 5} K^{3 / 5}}{2 g}=\frac{5 K \rho_{0}^{2 / 3}}{2 g}=\frac{5 P_{0}}{2 g \rho_{0}}
\]


\textbf{Problem:}
Subproblem 0: An eclipsing binary consists of two stars of different radii and effective temperatures. Star 1 has radius $R_{1}$ and $T_{1}$, and Star 2 has $R_{2}=0.5 R_{1}$ and $T_{2}=2 T_{1}$. Find the change in bolometric magnitude of the binary, $\Delta m_{\text {bol }}$, when the smaller star is behind the larger star. (Consider only bolometric magnitudes so you don't have to worry about color differences.)


\textbf{Solution:}
\[
\begin{gathered}
\mathcal{F}_{1 \& 2}=4 \pi \sigma\left(T_{1}^{4} R_{1}^{2}+T_{2}^{4} R_{2}^{2}\right) \\
\mathcal{F}_{\text {eclipse }}=4 \pi \sigma T_{1}^{4} R_{1}^{2} \\
\Delta m=-2.5 \log \left(\frac{\mathcal{F}_{1 \& 2}}{\mathcal{F}_{\text {eclipse }}}\right) \\
\Delta m=-2.5 \log \left(1+\frac{T_{2}^{4} R_{2}^{2}}{T_{1}^{4} R_{1}^{2}}\right) \\
\Delta m=-2.5 \log \left(1+\frac{16}{4}\right)=-1.75
\end{gathered}
\]
So, the binary is $\boxed{1.75}$ magnitudes brighter out of eclipse than when star 2 is behind star 1 .


\textbf{Problem:}
Preamble: It has been suggested that our Galaxy has a spherically symmetric dark-matter halo with a density distribution, $\rho_{\text {dark }}(r)$, given by:
\[
\rho_{\text {dark }}(r)=\rho_{0}\left(\frac{r_{0}}{r}\right)^{2},
\]
where $\rho_{0}$ and $r_{0}$ are constants, and $r$ is the radial distance from the center of the galaxy. For star orbits far out in the halo you can ignore the gravitational contribution of the ordinary matter in the Galaxy.

Subproblem 0: Compute the rotation curve of the Galaxy (at large distances), i.e., find $v(r)$ for circular orbits.


\textbf{Solution:}
\[
\begin{gathered}
-\frac{G M(<r)}{r^{2}}=-\frac{v^{2}}{r} \quad(\text { from } F=m a) \\
M(<r)=\int_{0}^{r} \rho_{0}\left(\frac{r_{0}}{r}\right)^{2} 4 \pi r^{2} d r=4 \pi \rho_{0} r_{0}^{2} r
\end{gathered}
\]
Note that, in general, $M \neq \rho \times$ volume! You must integrate over $\rho(r)$. From these expressions we find:
\[
v(r)=\boxed{\sqrt{4 \pi G \rho_{0} r_{0}^{2}}}=\text { constant }
\]


\textbf{Problem:}
Subproblem 0: The Very Large Array (VLA) telescope has an effective diameter of $36 \mathrm{~km}$, and a typical wavelength used for observation at this facility might be $6 \mathrm{~cm}$.  Based on this information, compute an estimate for the angular resolution of the VLA in arcseconds


\textbf{Solution:}
Using the formula for angular resolution $\theta$ in terms of the effective size $d$ and the wavelength $\lambda$, namely $\theta = \lambda/d$, gives \boxed{0.33} arcseconds.


\textbf{Problem:}
Subproblem 0: A particular star has an absolute magnitude $M=-7$. If this star is observed in a galaxy that is at a distance of $3 \mathrm{Mpc}$, what will its apparent magnitude be? 


Solution: \[
\text { Given: } M=-7 \text { and } d=3 \mathrm{Mpc}
\]
\[
\begin{aligned}
  & \text { Apparent Magnitude: } m=M+5 \log \left[\frac{d}{10 \mathrm{pc}}\right]=-7+5 \log \left[\frac{3 \times 10^{6}}{10}\right]=\boxed{20.39} \\
\end{aligned}
\]

Final answer: The final answer is 20.39. I hope it is correct.

Subproblem 1: What is the distance modulus to this galaxy?


\textbf{Solution:}
Distance Modulus: $DM=m-M=20.39+7=\boxed{27.39}$
\end{aligned}


\textbf{Problem:}
Subproblem 0: Find the distance modulus to the Andromeda galaxy (M31). Take the distance to Andromeda to be $750 \mathrm{kpc}$, and answer to three significant figures.


\textbf{Solution:}
\[
\mathrm{DM}=5 \log \left(\frac{d}{10 \mathrm{pc}}\right)=5 \log (75,000)=\boxed{24.4}
\]


\textbf{Problem:}
Subproblem 0: The Hubble Space telescope has an effective diameter of $2.5 \mathrm{~m}$, and a typical wavelength used for observation by the Hubble might be $0.6 \mu \mathrm{m}$, or 600 nanometers (typical optical wavelength). Based on this information, compute an estimate for the angular resolution of the Hubble Space telescope in arcseconds.


\textbf{Solution:}
Using the formula for angular resolution $\theta$ in terms of the effective size $d$ and the wavelength $\lambda$, namely $\theta = \lambda/d$, gives \boxed{0.05} arcseconds.


\textbf{Problem:}
Preamble: A collimated light beam propagating in water is incident on the surface (air/water interface) at an angle $\theta_w$ with respect to the surface normal.

Subproblem 0: If the index of refraction of water is $n=1.3$, find an expression for the angle of the light once it emerges from the water into the air, $\theta_a$, in terms of $\theta_w$.


\textbf{Solution:}
Using Snell's law, $1.3 \sin{\theta_w} = \sin{\theta_a}$. So $\theta_a = \boxed{\arcsin{1.3 \sin{\theta_w}}}$.


\textbf{Problem:}
Subproblem 0: What fraction of the rest mass energy is released (in the form of radiation) when a mass $\Delta M$ is dropped from infinity onto the surface of a neutron star with $M=1 M_{\odot}$ and $R=10$ $\mathrm{km}$ ?


\textbf{Solution:}
\[
\Delta E=\frac{G M \Delta m}{R}
\]
The fractional rest energy lost is $\Delta E / \Delta m c^{2}$, or
\[
\frac{\Delta E}{\Delta m c^{2}}=\frac{G M}{R c^{2}} \simeq \boxed{0.15}
\]


\textbf{Problem:}
Preamble: The density of stars in a particular globular star cluster is $10^{6} \mathrm{pc}^{-3}$. Take the stars to have the same radius as the Sun, and to have an average speed of $10 \mathrm{~km} \mathrm{sec}^{-1}$.

Subproblem 0: Find the mean free path for collisions among stars.  Express your answer in centimeters, to a single significant figure.


\textbf{Solution:}
\[
\begin{gathered}
\ell \simeq \frac{1}{n \sigma}=\frac{1}{10^{6} \mathrm{pc}^{-3} \pi R^{2}} \\
\ell \simeq \frac{1}{3 \times 10^{-50} \mathrm{~cm}^{-3} \times 1.5 \times 10^{22} \mathrm{~cm}^{2}} \simeq \boxed{2e27} \mathrm{~cm}
\end{gathered}
\]


\textbf{Problem:}
Subproblem 0: For a gas supported by degenerate electron pressure, the pressure is given by:
\[
P=K \rho^{5 / 3}
\]
where $K$ is a constant and $\rho$ is the mass density. If a star is totally supported by degenerate electron pressure, use a dimensional analysis of the equation of hydrostatic equilibrium:
\[
\frac{d P}{d r}=-g \rho
\]
to determine how the radius of such a star depends on its mass, $M$.  Specifically, you will find that $R$ is proportional to some power of $M$; what is that power?


\textbf{Solution:}
\[
\begin{gathered}
\frac{K \rho^{5 / 3}}{R} \simeq\left(\frac{G M}{R^{2}}\right)\left(\frac{M}{R^{3}}\right) \\
\rho \sim \frac{M}{R^{3}} \\
\frac{K M^{5 / 3}}{R R^{5}} \simeq \frac{G M^{2}}{R^{5}} \\
R \simeq \frac{K}{G M^{1 / 3}}
\end{gathered}
\]
So the answer is $\boxed{-1./3}$.


\textbf{Problem:}
Subproblem 0: A galaxy moves directly away from us with speed $v$, and the wavelength of its $\mathrm{H} \alpha$ line is observed to be $6784 \AA$. The rest wavelength of $\mathrm{H} \alpha$ is $6565 \AA$. Find $v/c$.


\textbf{Solution:}
\[
\lambda \simeq \lambda_{0}(1+v / c)
\]
where $\lambda=6784 \AA$ and $\lambda_{0}=6565 \AA$. Rearranging,
\[
\frac{v}{c} \simeq \frac{\lambda-\lambda_{0}}{\lambda_{0}} \simeq \frac{6784-6565}{6565} \Rightarrow v \simeq 0.033 c
\]
So $v/c \simeq \boxed{0.033}$.


\textbf{Problem:}
Subproblem 0: A candle has a power in the visual band of roughly $3$ Watts. When this candle is placed at a distance of $3 \mathrm{~km}$ it has the same apparent brightness as a certain star. Assume that this star has the same luminosity as the Sun in the visual band $\left(\sim 10^{26}\right.$ Watts $)$. How far away is the star (in pc)?


\textbf{Solution:}
The fact that the two sources have the same apparent brightness implies that the flux at the respective distances is the same; since flux varies with distance as $1/d^2$, we find that (with distances in km) $\frac{3}{3^2} = \frac{10^{26}}{d^2}$, so $d = 10^{13}\times\frac{3}{\sqrt{3}}$, or roughly $1.7\times 10^{13}$ kilometers.  In parsecs, this is $\boxed{0.5613}$ parsecs.


\textbf{Problem:}
Preamble: A galaxy is found to have a rotation curve, $v(r)$, given by
\[
v(r)=\frac{\left(\frac{r}{r_{0}}\right)}{\left(1+\frac{r}{r_{0}}\right)^{3 / 2}} v_{0}
\]
where $r$ is the radial distance from the center of the galaxy, $r_{0}$ is a constant with the dimension of length, and $v_{0}$ is another constant with the dimension of speed. The rotation curve is defined as the orbital speed of test stars in circular orbit at radius $r$.

Subproblem 0: Find an expression for $\omega(r)$, where $\omega$ is the angular velocity.  The constants $v_{0}$ and $r_{0}$ will appear in your answer.


\textbf{Solution:}
$\omega=v / r & \Rightarrow \omega(r)=\boxed{\frac{v_{0}}{r_{0}} \frac{1}{\left(1+r / r_{0}\right)^{3 / 2}}}$


\textbf{Problem:}
Preamble: Orbital Dynamics: A binary system consists of two stars in circular orbit about a common center of mass, with an orbital period, $P_{\text {orb }}=10$ days. Star 1 is observed in the visible band, and Doppler measurements show that its orbital speed is $v_{1}=20 \mathrm{~km} \mathrm{~s}^{-1}$. Star 2 is an X-ray pulsar and its orbital radius about the center of mass is $r_{2}=3 \times 10^{12} \mathrm{~cm}=3 \times 10^{10} \mathrm{~m}$.

Subproblem 0: Find the orbital radius, $r_{1}$, of the optical star (Star 1) about the center of mass, in centimeters.


\textbf{Solution:}
\[
\begin{gathered}
v_{1}=\frac{2 \pi r_{1}}{P_{\text {orb }}} \\
r_{1}=\frac{P_{\text {orb }} v_{1}}{2 \pi}=\boxed{2.75e11} \mathrm{~cm}
\end{gathered}
\]


\textbf{Problem:}
Preamble: The density of stars in a particular globular star cluster is $10^{6} \mathrm{pc}^{-3}$. Take the stars to have the same radius as the Sun, and to have an average speed of $10 \mathrm{~km} \mathrm{sec}^{-1}$.

Subproblem 0: Find the mean free path for collisions among stars.  Express your answer in centimeters, to a single significant figure.


Solution: \[
\begin{gathered}
\ell \simeq \frac{1}{n \sigma}=\frac{1}{10^{6} \mathrm{pc}^{-3} \pi R^{2}} \\
\ell \simeq \frac{1}{3 \times 10^{-50} \mathrm{~cm}^{-3} \times 1.5 \times 10^{22} \mathrm{~cm}^{2}} \simeq \boxed{2e27} \mathrm{~cm}
\end{gathered}
\]

Final answer: The final answer is 2e27. I hope it is correct.

Subproblem 1: Find the corresponding mean time between collisions. (Assume that the stars move in straight-line paths, i.e., are not deflected by gravitational interactions.)  Answer in units of years, to a single significant figure.


\textbf{Solution:}
$\tau_{\text {coll }} \simeq \frac{2 \times 10^{27} \mathrm{~cm}}{10^{6} \mathrm{~cm} / \mathrm{sec}} \simeq 2 \times 10^{21} \mathrm{sec} \simeq \boxed{6e13} \text { years }$


\textbf{Problem:}
Preamble: A radio interferometer, operating at a wavelength of $1 \mathrm{~cm}$, consists of 100 small dishes, each $1 \mathrm{~m}$ in diameter, distributed randomly within a $1 \mathrm{~km}$ diameter circle. 

Subproblem 0: What is the angular resolution of a single dish, in radians?


Solution: The angular resolution of a single dish is roughly given by the wavelength over its radius, in this case $\boxed{0.01}$ radians.

Final answer: The final answer is 0.01. I hope it is correct.

Subproblem 1: What is the angular resolution of the interferometer array for a source directly overhead, in radians?


\textbf{Solution:}
The angular resolution of the full array is given by the wavelength over the dimension of the array, in this case $\boxed{1e-5}$ radians.


\textbf{Problem:}
Subproblem 0: If a star cluster is made up of $10^{6}$ stars whose absolute magnitude is the same as that of the Sun (+5), compute the combined magnitude of the cluster if it is located at a distance of $10 \mathrm{pc}$.


\textbf{Solution:}
At $10 \mathrm{pc}$, the magnitude is (by definition) just the absolute magnitude of the cluster.  Since the total luminosity of the cluster is $10^{6}$ times the luminosity of the Sun, we have that 
\begin{equation}
\delta m = 2.5 \log \left( \frac{L_{TOT}}{L_{sun}} \right) = 2.5 \log 10^6 = 15.
\end{equation}
Since the Sun has absolute magnitude +5, the magnitude of the cluser is $\boxed{-10}$.


\textbf{Problem:}
Subproblem 0: A certain red giant has a radius that is 500 times that of the Sun, and a temperature that is $1 / 2$ that of the Sun's temperature. Find its bolometric (total) luminosity in units of the bolometric luminosity of the Sun.


\textbf{Solution:}
Power output goes as $T^4r^2$, so the power output of this star is $\boxed{15625}$ times that of the Sun.


\textbf{Problem:}
Subproblem 0: Suppose air molecules have a collision cross section of $10^{-16} \mathrm{~cm}^{2}$. If the (number) density of air molecules is $10^{19} \mathrm{~cm}^{-3}$, what is the collision mean free path in cm? Answer to one significant figure.


\textbf{Solution:}
\[
\ell=\frac{1}{n \sigma}=\frac{1}{10^{19} 10^{-16}}=\boxed{1e-3} \mathrm{~cm}
\]


\textbf{Problem:}
Subproblem 0: Two stars have the same surface temperature. Star 1 has a radius that is $2.5$ times larger than the radius of star 2. Star 1 is ten times farther away than star 2. What is the absolute value of the difference in apparent magnitude between the two stars, rounded to the nearest integer?


\textbf{Solution:}
Total power output goes as $r^2 T^4$, where $r$ is the star's radius, and $T$ is its temperature.  Flux, at a distance $R$ away thus goes as $r^2 T^4 / R^2$.  In our case, the ratio of flux from star 1 to star 2 is $1/16$ (i.e., star 2 is greater in apparent magnitude).  Using the relation between apparent magnitude and flux, we find that that the absolute value of the difference in apparent magnitudes is $2.5 \log{16}$, which rounded to the nearest integer is $\boxed{3}$.


\textbf{Problem:}
Subproblem 0: What is the slope of a $\log N(>F)$ vs. $\log F$ curve for a homogeneous distribution of objects, each of luminosity, $L$, where $F$ is the flux at the observer, and $N$ is the number of objects observed per square degree on the sky?


\textbf{Solution:}
The number of objects detected goes as the cube of the distance for objects with flux greater than a certain minimum flux. At the same time the flux falls off with the inverse square of the distance. Thus, the slope of the $\log N(>F)$ vs. $\log F$ curve is $\boxed{-3./2}$.


\textbf{Problem:}
Preamble: Comparison of Radio and Optical Telescopes.

Subproblem 0: The Very Large Array (VLA) is used to make an interferometric map of the Orion Nebula at a wavelength of $10 \mathrm{~cm}$. What is the best angular resolution of the radio image that can be produced, in radians? Note that the maximum separation of two antennae in the VLA is $36 \mathrm{~km}$.


\textbf{Solution:}
The best angular resolution will occur at the maximum separation, and is simply the ratio of wavelength to this separation $p$: $\theta = \frac{\lambda}{p}$, or $\frac{0.1}{36\times 10^3}$, which is $\boxed{2.7778e-6}$ radians.


\textbf{Problem:}
Subproblem 0: A globular cluster has $10^{6}$ stars each of apparent magnitude $+8$. What is the combined apparent magnitude of the entire cluster?


\textbf{Solution:}
\[
\begin{gathered}
+8=-2.5 \log \left(F / F_{0}\right) \\
F=6.3 \times 10^{-4} F_{0} \\
F_{\text {cluster }}=10^{6} \times 6.3 \times 10^{-4} F_{0}=630 F_{0} \\
m_{\text {cluster }}=-2.5 \log (630)=\boxed{-7}
\end{gathered}
\]


\textbf{Problem:}
Preamble: A very hot star is detected in the galaxy M31 located at a distance of $800 \mathrm{kpc}$. The star has a temperature $T = 6 \times 10^{5} K$ and produces a flux of $10^{-12} \mathrm{erg} \cdot \mathrm{s}^{-1} \mathrm{cm}^{-2}$ at the Earth. Treat the star's surface as a blackbody radiator.

Subproblem 0: Find the luminosity of the star (in units of $\mathrm{erg} \cdot \mathrm{s}^{-1}$).


Solution: \[
  L=4 \pi D^{2} \text { Flux }_{\text {Earth }}=10^{-12} 4 \pi\left(800 \times 3 \times 10^{21}\right)^{2}=\boxed{7e37} \mathrm{erg} \cdot \mathrm{s}^{-1}
\]

Final answer: The final answer is 7e37. I hope it is correct.

Subproblem 1: Compute the star's radius in centimeters.


Solution: \[
  R=\left(L / 4 \pi \sigma T^{4}\right)^{1 / 2}=\boxed{8.7e8} \mathrm{~cm}=0.012 R_{\odot}
\]

Final answer: The final answer is 8.7e8. I hope it is correct.

Subproblem 2: At what wavelength is the peak of the emitted radiation? Answer in $\AA$.


\textbf{Solution:}
Using the Wien displacement law:
\[
  \lambda_{\max }=0.29 / T \mathrm{~cm}=\boxed{48} \AA
\]


\section{Information and Entropy (6.050J Spring 2008)}

\textbf{Problem:}
Subproblem 0: A Boolean function $F(A, B)$ is said to be universal if any arbitrary boolean function can be constructed by using nested $F(A, B)$ functions. A universal function is useful, since using it we can build any function we wish out of a single part. For example, when implementing boolean logic on a computer chip a universal function (called a 'gate' in logic-speak) can simplify design enormously. We would like to find a universal boolean function. In this problem we will denote the two boolean inputs $A$ and $B$ and the one boolean output as $C$. 
First, to help us organize our thoughts, let's enumerate all of the functions we'd like to be able to construct. How many different possible one-output boolean functions of two variables are there? I.e., how many functions are there of the form $F(A, B)=C ?$


\textbf{Solution:}
This particular definition of universality only treats arbitrary functions of two Boolean variables, but with any number of outputs. It appears to be an onerous task to prove universality for an arbitrary number of outputs. However, since each individual output of a multi-output function can be considered a separate one-ouput function, it is sufficient to prove the case of only one-output functions. This is why we begin by listing all one-output functions of one variable.
Each variable $A$ and $B$ has two possible values, making four different combinations of inputs $(A, B)$. Each combination of inputs (four possible) can cause one of two output values. Therefore the number of possible one-output binary functions of two binary variables is $2^{4}$, or \boxed{16}. They are enumerated in the table below.
\begin{tabular}{cc|ccccccccccccccccccc}
$A$ & $B$ & $b_{0}$ & $b_{1}$ & $b_{2}$ & $b_{3}$ & $b_{4}$ & $b_{5}$ & $b_{6}$ & $b_{7}$ & $b_{8}$ & $b_{9}$ & $b_{10}$ & $b_{11}$ & $b_{12}$ & $b_{13}$ & $b_{14}$ & $b_{15}$ &  \\
\hline
0 & 0 & 0 & 1 & 0 & 1 & 0 & 1 & 0 & 1 & 0 & 1 & 0 & 1 & 0 & 1 & 0 & 1 &  \\
0 & 1 & 0 & 0 & 1 & 1 & 0 & 0 & 1 & 1 & 0 & 0 & 1 & 1 & 0 & 0 & 1 & 1 &  \\
1 & 0 & 0 & 0 & 0 & 0 & 1 & 1 & 1 & 1 & 0 & 0 & 0 & 0 & 1 & 1 & 1 & 1 &  \\
1 & 1 & 0 & 0 & 0 & 0 & 0 & 0 & 0 & 0 & 1 & 1 & 1 & 1 & 1 & 1 & 1 & 1 &  \\
\end{tabular}


\textbf{Problem:}
Subproblem 0: Unfortunately, a mutant gene can turn box people into triangles late in life. A laboratory test has been developed which can spot the gene early so that the dreaded triangle transformation can be prevented by medications. This test is 95 percent accurate at spotting the gene when it is there. However, the test gives a "false positive" $0.4$ percent of the time, falsely indicating that a healthy box person has the mutant gene. If $0.1$ percent (be careful - that's one-tenth of one percent) of the box people have the mutant gene, what's the probability that a box person actually has the mutant gene if the test indicates that he or she does?


\textbf{Solution:}
We see that the probability that a person has the disease given that the test is positive, is:
\[
\frac{0.001 \times 0.95}{0.001 \times 0.95+0.999 \times 0.004}=19.2 \%
\]
$\begin{array}{ccccc}\text { Have Disease? } & \text { Percent } & \text { Test Results } & \text { Percent } & \text { Total } \\ \text { Yes } & 0.001 & \text { Positive } & 0.95 & 0.00095 \\ & & \text { Negative } & 0.05 & 0.00005 \\ \text { No } & 0.999 & \text { Positive } & 0.004 & 0.003996 \\ & & \text { Negative } & 0.996 & 0.95504\end{array}$
Answer: \boxed{0.192}.


\textbf{Problem:}
Subproblem 0: Buzz, the hot new dining spot on campus, emphasizes simplicity. It only has two items on the menu, burgers and zucchini. Customers make a choice as they enter (they are not allowed to order both), and inform the cooks in the back room by shouting out either "B" or "Z". Unfortunately the two letters sound similar so $8 \%$ of the time the cooks misinterpret what was said. The marketing experts who designed the restaurant guess that $90 \%$ of the orders will be for burgers and $10 \%$ for zucchini.
The cooks can hear one order per second. The customers arrive at the rate of one per second. One of the chefs says that this system will never work because customers can only send one bit per second, the rate at which orders can be accepted, so you could barely keep up even if there were no noise in the channel. You are hired as an outside consultant to deal with the problem.
What is the channel capacity $\mathrm{C}$ of this communication channel in bits per second?


\textbf{Solution:}
This is a noisy channel with the same probabilities for mixing up $Z$ and $B$. Channel capacity is defined as the maximum mutual information (for any possible input probability) times the rate $W$. The rate of error is $\epsilon=0.08$. So the channel capacity for this channel is given by:
\[
\begin{aligned}
C &=M_{\max } W \\
&=1-\epsilon \log _{2}\left(\frac{1}{\epsilon}\right)-(1-\epsilon) \log _{2}\left(\frac{1}{(1-\epsilon)}\right) \\
&=1-0.08 \log _{2}\left(\frac{1}{0.08}\right)-(0.92) \log _{2}\left(\frac{1}{0.92}\right) \\
&=0.5978 \mathrm{bits} / \mathrm{second}
\end{aligned}
\]
So the final answer is \boxed{0.5978} bits/s.


\section{Ecology I (1.018J Fall 2009)}

\textbf{Problem:}
Preamble: Given the following data from an Experimental Forest, answer the following questions. Show your work and units.
$\begin{array}{ll}\text { Total vegetative biomass } & 80,000 \mathrm{kcal} \mathrm{m}^{-2} \\ \text { Detritus and organic matter in soil } & 120,000 \mathrm{kcal } \mathrm{m}^{-2} \\ \text { Total Gross Primary Productivity } & 20,000 \mathrm{kcal } \mathrm{m}^{-2} \mathrm{yr}^{-1} \\ \text { Total Plant Respiration } & 5,000 \mathrm{kcal} \mathrm{m}^{-2} \mathrm{yr}^{-1} \\ \text { Total Community Respiration } & 9,000 \mathrm{kcal} \mathrm{m}^{-2} \mathrm{yr}^{-1}\end{array}$

Subproblem 0: What is the net primary productivity of the forest?


Solution: NPP $=$ GPP $-R_{A}=20,000-5,000=\boxed{15000} \mathrm{kcal} \mathrm{m}^{-2} \mathrm{yr}^{-1}$

Final answer: The final answer is 15000. I hope it is correct.

Subproblem 1: What is the net community production?


\textbf{Solution:}
$\mathrm{NCP}=\mathrm{GPP}-\mathrm{R}_{\mathrm{A}}-\mathrm{R}_{\mathrm{H}}=20,000-9000=\boxed{11000} \mathrm{kcal} \mathrm{m}^{-2} \mathrm{yr}^{-1}$


\textbf{Problem:}
Preamble: A population of 100 ferrets is introduced to a large island in the beginning of 1990 . Ferrets have an intrinsic growth rate, $r_{\max }$ of $1.3 \mathrm{yr}^{-1}$.

Subproblem 0: Assuming unlimited resources-i.e., there are enough resources on this island to last the ferrets for hundreds of years-how many ferrets will there be on the island in the year 2000? (Show your work!)


Solution: $N_o = 100$ (in 1990)
\\
$N = ?$ (in 2000)
\\
$t = 10$ yr
\\
$r = 1.3 \text{yr}^{-1}$
\\
$N = N_{o}e^{rt} = 100*e^{(1.3/\text{yr})(10 \text{yr})} = 4.4 x 10^7$ ferrets
\\
There will be \boxed{4.4e7} ferrets on the island in the year 2000. 

Final answer: The final answer is 4.4e7. I hope it is correct.

Subproblem 1: What is the doubling time of the ferret population? (Show your work!)


\textbf{Solution:}
$N_o = 100$ (in 1990)
\\
$t = 10$ yr
\\
$r = 1.3 \text{yr}^{-1}$
\\
$t_d = (ln(2))/r = 0.693/(1.3 \text{yr}^{-1}) = 0.53$ years
\\
The doubling time of the ferret population is \boxed{0.53} years. 


\textbf{Problem:}
Preamble: Given the following data from an Experimental Forest, answer the following questions. Show your work and units.
$\begin{array}{ll}\text { Total vegetative biomass } & 80,000 \mathrm{kcal} \mathrm{m}^{-2} \\ \text { Detritus and organic matter in soil } & 120,000 \mathrm{kcal } \mathrm{m}^{-2} \\ \text { Total Gross Primary Productivity } & 20,000 \mathrm{kcal } \mathrm{m}^{-2} \mathrm{yr}^{-1} \\ \text { Total Plant Respiration } & 5,000 \mathrm{kcal} \mathrm{m}^{-2} \mathrm{yr}^{-1} \\ \text { Total Community Respiration } & 9,000 \mathrm{kcal} \mathrm{m}^{-2} \mathrm{yr}^{-1}\end{array}$

Subproblem 0: What is the net primary productivity of the forest?


\textbf{Solution:}
NPP $=$ GPP $-R_{A}=20,000-5,000=\boxed{15000} \mathrm{kcal} \mathrm{m}^{-2} \mathrm{yr}^{-1}$


\textbf{Problem:}
Preamble: The Peak District Moorlands in the United Kingdom store 20 million tonnes of carbon, almost half of the carbon stored in the soils of the entire United Kingdom (the Moorlands are only $8 \%$ of the land area). In pristine condition, these peatlands can store an additional 13,000 tonnes of carbon per year.

Subproblem 0: Given this rate of productivity, how long did it take for the Peatlands to sequester this much carbon?


\textbf{Solution:}
$20,000,000$ tonnes $C / 13,000$ tonnes $C y^{-1}=\boxed{1538}$ years


\textbf{Problem:}
Preamble: A population of 100 ferrets is introduced to a large island in the beginning of 1990 . Ferrets have an intrinsic growth rate, $r_{\max }$ of $1.3 \mathrm{yr}^{-1}$.

Subproblem 0: Assuming unlimited resources-i.e., there are enough resources on this island to last the ferrets for hundreds of years-how many ferrets will there be on the island in the year 2000? (Show your work!)


\textbf{Solution:}
$N_o = 100$ (in 1990)
\\
$N = ?$ (in 2000)
\\
$t = 10$ yr
\\
$r = 1.3 \text{yr}^{-1}$
\\
$N = N_{o}e^{rt} = 100*e^{(1.3/\text{yr})(10 \text{yr})} = 4.4 x 10^7$ ferrets
\\
There will be \boxed{4.4e7} ferrets on the island in the year 2000. 


\section{Differential Equations (18.03 Spring 2010)}

\textbf{Problem:}
Preamble: The following subproblems refer to a circuit with the following parameters. Denote by $I(t)$ the current (where the positive direction is, say, clockwise) in the circuit and by $V(t)$ the voltage increase across the voltage source, at time $t$. Denote by $R$ the resistance of the resistor and $C$ the capacitance of the capacitor (in units which we will not specify)-both positive numbers. Then
\[
R \dot{I}+\frac{1}{C} I=\dot{V}
\]

Subproblem 0: Suppose that $V$ is constant, $V(t)=V_{0}$. Solve for $I(t)$, with initial condition $I(0)$.


Solution: When $V$ is constant, the equation becomes $R \dot{I}+\frac{1}{C} I=0$, which is separable. Solving gives us
\[
I(t)=\boxed{I(0) e^{-\frac{t}{R C}}
}\]. 

Final answer: The final answer is I(0) e^{-\frac{t}{R C}}
. I hope it is correct.

Subproblem 1: It is common to write the solution to the previous subproblem in the form $c e^{-t / \tau}$. What is $c$ in this case?


\textbf{Solution:}
$c=\boxed{I(0)}$.


\textbf{Problem:}
Subproblem 0: Consider the following "mixing problem." A tank holds $V$ liters of salt water. Suppose that a saline solution with concentration of $c \mathrm{gm} /$ liter is added at the rate of $r$ liters/minute. A mixer keeps the salt essentially uniformly distributed in the tank. A pipe lets solution out of the tank at the same rate of $r$ liters/minute. The differential equation for the amount of salt in the tank is given by 
\[
x^{\prime}+\frac{r}{V} x-r c=0 .
\]
Suppose that the out-flow from this tank leads into another tank, also of volume 1 , and that at time $t=1$ the water in it has no salt in it. Again there is a mixer and an outflow. Write down a differential equation for the amount of salt in this second tank, as a function of time, assuming the amount of salt in the second tank at moment $t$ is given by $y(t)$, and the amount of salt in the first tank at moment $t$ is given by $x(t)$. 


\textbf{Solution:}
The differential equation for $y(t)$ is $\boxed{y^{\prime}+r y-r x(t)=0}$.


\textbf{Problem:}
Subproblem 0: Find the general solution of $x^{2} y^{\prime}+2 x y=\sin (2 x)$, solving for $y$. Note that a general solution to a differential equation has the form $x=x_{p}+c x_{h}$ where $x_{h}$ is a nonzero solution of the homogeneous equation $\dot{x}+p x=0$. Additionally, note that the left hand side is the derivative of a product.


\textbf{Solution:}
We see that $\left(x^{2} y\right)^{\prime}=x^{2} y^{\prime}+2 x y$. Thus, $x^{2} y=-\frac{1}{2} \cos (2 x)+c$, and $y=\boxed{c x^{-2}-\frac{\cos (2 x)}{2 x^{2}}}$.


\textbf{Problem:}
Subproblem 0: An African government is trying to come up with good policy regarding the hunting of oryx. They are using the following model: the oryx population has a natural growth rate of $k$, and we suppose a constant harvesting rate of $a$ oryxes per year.
Write down an ordinary differential equation describing the evolution of the oryx population given the dynamics above, using $x(t)$ to denote the oryx population (the number of individual oryx(es)) at time $t$, measured in years.


\textbf{Solution:}
The natural growth rate is $k$, meaning that after some short time $\Delta t$ year(s) passes, we expect $k x(t) \Delta t$ new oryxes to appear. However, meanwhile the population is reduced by $a \Delta t$ oryxes due to the harvesting. Therefore, we are led to
\[
x(t+\Delta t) \simeq x(t)+k x(t) \Delta t-a \Delta t,
\]
and the unit on both sides is oryx $(\mathrm{es})$. If we let $\Delta t$ approach 0 , then we get the differential equation
\[
\boxed{\frac{d x}{d t}=k x-a} .
\]


\textbf{Problem:}
Subproblem 0: If the complex number $z$ is given by $z = 1+\sqrt{3} i$, what is the magnitude of $z^2$?


\textbf{Solution:}
$z^{2}$ has argument $2 \pi / 3$ and radius 4, so by Euler's formula, $z^{2}=4 e^{i 2 \pi / 3}$. Thus $A=4, \theta=\frac{2\pi}{3}$, so our answer is $\boxed{4}$.


\textbf{Problem:}
Subproblem 0: In the polar representation $(r, \theta)$ of the complex number $z=1+\sqrt{3} i$, what is $r$?


\textbf{Solution:}
For z, $r=2$ and $\theta=\pi / 3$, so its polar coordinates are $\left(2, \frac{\pi}{3}\right)$. So $r=\boxed{2}$.


\textbf{Problem:}
Preamble: In the following problems, take $a = \ln 2$ and $b = \pi / 3$. 

Subproblem 0: Given $a = \ln 2$ and $b = \pi / 3$, rewrite $e^{a+b i}$ in the form $x + yi$, where $x, y$ are real numbers. 


\textbf{Solution:}
Using Euler's formula, we find that the answer is $\boxed{1+\sqrt{3} i}$.


\textbf{Problem:}
Subproblem 0: Find the general solution of the differential equation $y^{\prime}=x-2 y$ analytically using integrating factors, solving for $y$. Note that a function $u(t)$ such that $u \dot{x}+u p x=\frac{d}{d t}(u x)$ is an integrating factor. Additionally, note that a general solution to a differential equation has the form $x=x_{p}+c x_{h}$ where $x_{h}$ is a nonzero solution of the homogeneous equation $\dot{x}+p x=0$.


Solution: In standard form, $y^{\prime}+2 y=x$, so $u=C e^{2 x}$. Then $y=u^{-1} \int u x d x=e^{-2 x} \int x e^{2 x} d x$. Integrating by parts yields $\int x e^{2 x} d x=$ $\frac{x}{2} e^{2 x}-\frac{1}{2} \int e^{2 x} d x=\frac{x}{2} e^{2 x}-\frac{1}{4} e^{2 x}+c$. Therefore, $y=\boxed{x / 2-1 / 4+c e^{-2 x}}$.

Final answer: The final answer is x / 2-1 / 4+c e^{-2 x}. I hope it is correct.

Subproblem 1: For what value of $c$ does the straight line solution occur?


\textbf{Solution:}
The straight line solution occurs when $c=\boxed{0}$.


\textbf{Problem:}
Preamble: The following subproblems relate to applying Euler's Method (a first-order numerical procedure for solving ordinary differential equations with a given initial value) onto $y^{\prime}=y^{2}-x^{2}=F(x, y)$ at $y(0)=-1$, with $h=0.5$. Recall the notation \[x_{0}=0, y_{0}=-1, x_{n+1}=x_{h}+h, y_{n+1}=y_{n}+m_{n} h, m_{n}=F\left(x_{n}, y_{n}\right)\]. 

Subproblem 0: Use Euler's method to estimate the value at $x=1.5$.


\textbf{Solution:}
$y_3 = \boxed{-0.875}$


\textbf{Problem:}
Subproblem 0: Rewrite the function $f(t) = \cos (2 t)+\sin (2 t)$ in the form $A \cos (\omega t-\phi)$. It may help to begin by drawing a right triangle with sides $a$ and $b$. 


\textbf{Solution:}
Here, our right triangle has hypotenuse $\sqrt{2}$, so $A=\sqrt{2}$. Both summands have "circular frequency" 2, so $\omega=2 . \phi$ is the argument of the hypotenuse, which is $\pi / 4$, so $f(t)=\boxed{\sqrt{2} \cos (2 t-\pi / 4)}$.


\textbf{Problem:}
Subproblem 0: Given the ordinary differential equation $\ddot{x}-a^{2} x=0$, where $a$ is a nonzero real-valued constant, find a solution $x(t)$ to this equation such that $x(0) = 0$ and $\dot{x}(0)=1$.


\textbf{Solution:}
First, notice that both $x(t)=e^{a t}$ and $x(t)=e^{-a t}$ are solutions to $\ddot{x}-a^{2} x=0$. Then for any constants $c_{1}$ and $c_{2}$, $x(t)=c_{1} e^{a t}+c_{2} e^{-a t}$ are also solutions to $\ddot{x}-a^{2} x=0$. Moreover, $x(0)=c_{1}+c_{2}$, and $\dot{x}(0)=a\left(c_{1}-c_{2}\right)$. Assuming $a \neq 0$, to satisfy the given conditions, we need $c_{1}+c_{2}=0$ and $a\left(c_{1}-c_{2}\right)=1$, which implies $c_{1}=-c_{2}=\frac{1}{2 a}$. So $x(t)=\boxed{\frac{1}{2a}(\exp{a*t} - \exp{-a*t})}$.


\textbf{Problem:}
Subproblem 0: Find a solution to the differential equation $\ddot{x}+\omega^{2} x=0$ satisfying the initial conditions $x(0)=x_{0}$ and $\dot{x}(0)=\dot{x}_{0}$.


\textbf{Solution:}
Suppose \[x(t)=a \cos (\omega t)+b \sin (\omega t)\] $x(0)=a$, therefore $a=x_{0}$. Then \[x^{\prime}(0)=-a \omega \sin 0+b \omega \cos 0=b \omega=\dot{x}_{0}\] Then $b=\dot{x}_{0} / \omega$. The solution is then $x=\boxed{x_{0} \cos (\omega t)+$ $\dot{x}_{0} \sin (\omega t) / \omega}$. 


\textbf{Problem:}
Subproblem 0: Find the complex number $a+b i$ with the smallest possible positive $b$ such that $e^{a+b i}=1+\sqrt{3} i$. 


\textbf{Solution:}
$1+\sqrt{3} i$ has modulus 2 and argument $\pi / 3+2 k \pi$ for all integers k, so $1+\sqrt{3} i$ can be expressed as a complex exponential of the form $2 e^{i(\pi / 3+2 k \pi)}$. Taking logs gives us the equation $a+b i=\ln 2+i(\pi / 3+2 k \pi)$. The smallest positive value of $b$ is $\pi / 3$. Thus we have $\boxed{\ln 2 + i\pi / 3}$


\textbf{Problem:}
Subproblem 0: Find the general solution of the differential equation $\dot{x}+2 x=e^{t}$, using $c$ for the arbitrary constant of integration which will occur.


Solution: We can use integrating factors to get $(u x)^{\prime}=u e^{t}$ for $u=e^{2 t}$. Integrating yields $e^{2 t} x=e^{3 t} / 3+c$, or $x=\boxed{\frac{e^{t}} {3}+c e^{-2 t}}$. 

Final answer: The final answer is \frac{e^{t}} {3}+c e^{-2 t}. I hope it is correct.

Subproblem 1: Find a solution of the differential equation $\dot{x}+2 x=e^{t}$ of the form $w e^{t}$, where $w$ is a constant (which you should find).


\textbf{Solution:}
When $c=0, x=\boxed{e^{t} / 3}$ is the solution of the required form. 


\textbf{Problem:}
Subproblem 0: For $\omega \geq 0$, find $A$ such that $A \cos (\omega t)$ is a solution of $\ddot{x}+4 x=\cos (\omega t)$.


Solution: If $x=A \cos (\omega t)$, then taking derivatives gives us $\ddot{x}=-\omega^{2} A \cos (\omega t)$, and $\ddot{x}+4 x=\left(4-\omega^{2}\right) A \cos (\omega t)$. Then $A=\boxed{\frac{1}{4-\omega^{2}}}$. 

Final answer: The final answer is \frac{1}{4-\omega^{2}}. I hope it is correct.

Subproblem 1: For what value of $\omega$ does resonance occur? 


\textbf{Solution:}
Resonance occurs when $\omega=\boxed{2}$. 


\textbf{Problem:}
Subproblem 0: Find a purely sinusoidal solution of $\frac{d^{4} x}{d t^{4}}-x=\cos (2 t)$.


Solution: We choose an exponential input function whose real part is $\cos (2 t)$, namely $e^{2 i t}$. Since $p(s)=s^{4}-1$ and $p(2 i)=15 \neq 0$, the exponential response formula yields the solution $\frac{e^{2 i t}}{15}$. A sinusoidal solution to the original equation is given by the real part: $\boxed{\frac{\cos (2 t)}{15}}$. 

Final answer: The final answer is \frac{\cos (2 t)}{15}. I hope it is correct.

Subproblem 1: Find the general solution to $\frac{d^{4} x}{d t^{4}}-x=\cos (2 t)$, denoting constants as $C_{1}, C_{2}, C_{3}, C_{4}$.


\textbf{Solution:}
To get the general solution, we take the sum of the general solution to the homogeneous equation and the particular solution to the original equation. The homogeneous equation is $\frac{d^{4} x}{d t^{4}}-x=0$. The characteristic polynomial $p(s)=s^{4}-1$ has 4 roots: $\pm 1, \pm i$. So the general solution to $\frac{d^{4} x}{d t^{4}}-x=0$ is given by $C_{1} e^{t}+C_{2} e^{-t}+C_{3} \cos (t)+C_{4} \sin (t)$ for arbitrary real constants $C_{1}, C_{2}, C_{3}, C_{4}$.
The solution to the equation is $\boxed{\frac{\cos (2 t)}{15}+C_{1} e^{t}+C_{2} e^{-t}+C_{3} \cos (t)+C_{4} \sin (t)}$.


\textbf{Problem:}
Subproblem 0: For $\omega \geq 0$, find $A$ such that $A \cos (\omega t)$ is a solution of $\ddot{x}+4 x=\cos (\omega t)$.


\textbf{Solution:}
If $x=A \cos (\omega t)$, then taking derivatives gives us $\ddot{x}=-\omega^{2} A \cos (\omega t)$, and $\ddot{x}+4 x=\left(4-\omega^{2}\right) A \cos (\omega t)$. Then $A=\boxed{\frac{1}{4-\omega^{2}}}$. 


\textbf{Problem:}
Subproblem 0: Find a solution to $\dot{x}+2 x=\cos (2 t)$ in the form $k_0\left[f(k_1t) + g(k_2t)\right]$, where $f, g$ are trigonometric functions.  Do not include homogeneous solutions to this ODE in your solution.


\textbf{Solution:}
$\cos (2 t)=\operatorname{Re}\left(e^{2 i t}\right)$, so $x$ can be the real part of any solution $z$ to $\dot{z}+2 z=e^{2 i t}$. One solution is given by $x=\operatorname{Re}\left(e^{2 i t} /(2+2 i)\right)=\boxed{\frac{\cos (2 t)+\sin (2 t)}{4}}$. 


\textbf{Problem:}
Preamble: The following subproblems refer to the differential equation. $\ddot{x}+4 x=\sin (3 t)$

Subproblem 0: Find $A$ so that $A \sin (3 t)$ is a solution of $\ddot{x}+4 x=\sin (3 t)$.


\textbf{Solution:}
We can find this by brute force. If $x=A \sin (3 t)$, then $\ddot{x}=-9 A \sin (3 t)$, so $\ddot{x}+4 x=-5 A \sin (3 t)$. Therefore, when $A=\boxed{-0.2}, x_{p}(t)=-\sin (3 t) / 5$ is a solution of the given equation.


\textbf{Problem:}
Subproblem 0: Find the general solution of the differential equation $y^{\prime}=x-2 y$ analytically using integrating factors, solving for $y$. Note that a function $u(t)$ such that $u \dot{x}+u p x=\frac{d}{d t}(u x)$ is an integrating factor. Additionally, note that a general solution to a differential equation has the form $x=x_{p}+c x_{h}$ where $x_{h}$ is a nonzero solution of the homogeneous equation $\dot{x}+p x=0$.


\textbf{Solution:}
In standard form, $y^{\prime}+2 y=x$, so $u=C e^{2 x}$. Then $y=u^{-1} \int u x d x=e^{-2 x} \int x e^{2 x} d x$. Integrating by parts yields $\int x e^{2 x} d x=$ $\frac{x}{2} e^{2 x}-\frac{1}{2} \int e^{2 x} d x=\frac{x}{2} e^{2 x}-\frac{1}{4} e^{2 x}+c$. Therefore, $y=\boxed{x / 2-1 / 4+c e^{-2 x}}$.


\textbf{Problem:}
Subproblem 0: Find a purely exponential solution of $\frac{d^{4} x}{d t^{4}}-x=e^{-2 t}$.


Solution: The characteristic polynomial of the homogeneous equation is given by $p(s)=$ $s^{4}-1$. Since $p(-2)=15 \neq 0$, the exponential response formula gives the solution $\frac{e^{-2 t}}{p(-2)}=\boxed{\frac{e^{-2 t}}{15}}$.

Final answer: The final answer is \frac{e^{-2 t}}{15}. I hope it is correct.

Subproblem 1: Find the general solution to $\frac{d^{4} x}{d t^{4}}-x=e^{-2 t}$, denoting constants as $C_{1}, C_{2}, C_{3}, C_{4}$.


\textbf{Solution:}
To get the general solution, we take the sum of the general solution to the homogeneous equation and the particular solution to the original equation. The homogeneous equation is $\frac{d^{4} x}{d t^{4}}-x=0$. The characteristic polynomial $p(s)=s^{4}-1$ has 4 roots: $\pm 1, \pm i$. So the general solution to $\frac{d^{4} x}{d t^{4}}-x=0$ is given by $C_{1} e^{t}+C_{2} e^{-t}+C_{3} \cos (t)+C_{4} \sin (t)$ for arbitrary real constants $C_{1}, C_{2}, C_{3}, C_{4}$.
Therefore, the general solution to the equation is $\boxed{\frac{e^{-2 t}}{15}+C_{1} e^{t}+C_{2} e^{-t}+ C_{3} \cos (t)+C_{4} \sin (t)}$.


\textbf{Problem:}
Preamble: Consider the differential equation $\ddot{x}+\omega^{2} x=0$. \\

Subproblem 0: A differential equation $m \ddot{x}+b \dot{x}+k x=0$ (where $m, b$, and $k$ are real constants, and $m \neq 0$ ) has corresponding characteristic polynomial $p(s)=m s^{2}+b s+k$.\\
What is the characteristic polynomial $p(s)$ of $\ddot{x}+\omega^{2} x=0$?


\textbf{Solution:}
The characteristic polynomial $p(s)$ is $p(s)=\boxed{s^{2}+\omega^{2}}$.


\textbf{Problem:}
Subproblem 0: Rewrite the function $\cos (\pi t)-\sqrt{3} \sin (\pi t)$ in the form $A \cos (\omega t-\phi)$. It may help to begin by drawing a right triangle with sides $a$ and $b$. 


\textbf{Solution:}
The right triangle has hypotenuse of length $\sqrt{1^{2}+(-\sqrt{3})^{2}}=2$. The circular frequency of both summands is $\pi$, so $\omega=\pi$. The argument of the hypotenuse is $-\pi / 3$, so $f(t)=\boxed{2 \cos (\pi t+\pi / 3)}$.


\textbf{Problem:}
Preamble: The following subproblems refer to the damped sinusoid $x(t)=A e^{-a t} \cos (\omega t)$.

Subproblem 0: What is the spacing between successive maxima of $x(t)$? Assume that $\omega \neq 0$. 


\textbf{Solution:}
The extrema of $x(t)=A e^{-a t} \cos (\omega t)$ occur when $\dot{x}(t)=0$, i.e., $-a \cos (\omega t)=\omega \sin (\omega t)$. When $\omega \neq 0$, the extrema are achieved at $t$ where $\tan (\omega t)=-a / \omega$. Since minima and maxima of $x(t)$ are alternating, the maxima occur at every other $t \operatorname{such}$ that $\tan (\omega t)=-a / \omega$. If $t_{0}$ and $t_{1}$ are successive maxima, then $t_{1}-t_{0}=$ twice the period of $\tan (\omega t)=\boxed{2 \pi / \omega}$, 


\textbf{Problem:}
Preamble: The following subproblems refer to a spring/mass/dashpot system driven through the spring modeled by the equation $m \ddot{x}+b \dot{x}+k x=k y$. Here $x$ measures the position of the mass, $y$ measures the position of the other end of the spring, and $x=y$ when the spring is relaxed.

Subproblem 0: In this system, regard $y(t)$ as the input signal and $x(t)$ as the system response. Take $m=1, b=3, k=4, y(t)=A \cos t$. Replace the input signal by a complex exponential $y_{c x}(t)$ of which it is the real part, and compute the exponential ("steady state") system response $z_p(t)$; leave your answer in terms of complex exponentials, i.e. do not take the real part.


\textbf{Solution:}
The equation is $\ddot{x}+3 \dot{x}+4 x=4 A \cos t$, with the characteristic polynomial $p(s)=s^{2}+3 s+4$. The complex exponential corresponding to the input signal is $y_{c x}=A e^{i t}$ and $p(i)=3+3 i \neq 0$. By the Exponential Response Formula, $z_{p}=\frac{4 A}{p(i)} e^{i t}=\boxed{\frac{4 A}{3+3 i} e^{i t}}$. 


\textbf{Problem:}
Preamble: The following subproblems refer to a circuit with the following parameters. Denote by $I(t)$ the current (where the positive direction is, say, clockwise) in the circuit and by $V(t)$ the voltage increase across the voltage source, at time $t$. Denote by $R$ the resistance of the resistor and $C$ the capacitance of the capacitor (in units which we will not specify)-both positive numbers. Then
\[
R \dot{I}+\frac{1}{C} I=\dot{V}
\]

Subproblem 0: Suppose that $V$ is constant, $V(t)=V_{0}$. Solve for $I(t)$, with initial condition $I(0)$.


\textbf{Solution:}
When $V$ is constant, the equation becomes $R \dot{I}+\frac{1}{C} I=0$, which is separable. Solving gives us
\[
I(t)=\boxed{I(0) e^{-\frac{t}{R C}}
}\]. 


\textbf{Problem:}
Subproblem 0: Find the general (complex-valued) solution of the differential equation $\dot{z}+2 z=e^{2 i t}$, using $C$ to stand for any complex-valued integration constants which may arise.


Solution: Using integrating factors, we get $e^{2 t} z=e^{(2+2 i) t} /(2+2 i)+C$, or $z=\boxed{\frac{e^{2 i t}}{(2+2 i)}+C e^{-2 t}}$, where $C$ is any complex number.

Final answer: The final answer is \frac{e^{2 i t}}{(2+2 i)}+C e^{-2 t}. I hope it is correct.

Subproblem 1: Find a solution of the differential equation $\dot{z}+2 z=e^{2 i t}$ in the form $w e^{t}$, where $w$ is a constant (which you should find).


\textbf{Solution:}
When $C=0, z=\boxed{\frac{e^{2 i t}}{(2+2 i)}}$.


\textbf{Problem:}
Preamble: The following subproblems consider a second order mass/spring/dashpot system driven by a force $F_{\text {ext }}$ acting directly on the mass: $m \ddot{x}+b \dot{x}+k x=F_{\text {ext }}$. So the input signal is $F_{\text {ext }}$ and the system response is $x$. We're interested in sinusoidal input signal, $F_{\text {ext }}(t)=A \cos (\omega t)$, and in the steady state, sinusoidal system response, $x_{p}(t)=g A \cos (\omega t-\phi)$. Here $g$ is the gain of the system and $\phi$ is the phase lag. Both depend upon $\omega$, and we will consider how that is the case. \\
Take $A=1$, so the amplitude of the system response equals the gain, and take $m=1, b=\frac{1}{4}$, and $k=2$.\\

Subproblem 0: Compute the complex gain $H(\omega)$ of this system. (This means: make the complex replacement $F_{\mathrm{cx}}=e^{i \omega t}$, and express the exponential system response $z_{p}$ as a complex multiple of $F_{\mathrm{cx}}, i.e. z_{p}=H(\omega) F_{\mathrm{cx}}$).


\textbf{Solution:}
Set $F_{\mathrm{cx}}=e^{i \omega t}$. The complex replacement of the equation is $\ddot{z}+\frac{1}{4} \dot{z}+2 z=e^{i \omega t}$, with the characteristic polynomial $p(s)=s^{2}+\frac{1}{4} s+2.$ Given that $p(i \omega)=-\omega^{2}+\frac{\omega}{4} i+2 \neq 0$, so by the exponential response formula, $z_{p}=e^{i \omega t} / p(i \omega)=F_{\mathrm{cx}} / p(i \omega)$, and $H(\omega)=z_{p} / F_{\mathrm{cx}}=1 / p(i \omega)=$ $\frac{2-\omega^{2}-\omega i / 4}{\left(2-\omega^{2}\right)^{2}+(\omega / 4)^{2}}=\boxed{\frac{2-\omega^{2}-\omega i / 4}{\omega^{4}-\frac{63}{16} \omega^{2}+4}}$.


\textbf{Problem:}
Preamble: The following subproblems refer to the following "mixing problem": A tank holds $V$ liters of salt water. Suppose that a saline solution with concentration of $c \mathrm{gm} /$ liter is added at the rate of $r$ liters/minute. A mixer keeps the salt essentially uniformly distributed in the tank. A pipe lets solution out of the tank at the same rate of $r$ liters/minute. 

Subproblem 0: Write down the differential equation for the amount of salt in the tank in standard linear form. [Not the concentration!] Use the notation $x(t)$ for the number of grams of salt in the tank at time $t$.


\textbf{Solution:}
The concentration of salt at any given time is $x(t) / V \mathrm{gm} /$ liter, so for small $\Delta t$, we lose $r x(t) \Delta t / V$ gm from the exit pipe, and we gain $r c \Delta t \mathrm{gm}$ from the input pipe. The equation is $x^{\prime}(t)=r c-\frac{r x(t)}{V}$, and in standard linear form, it is
$\boxed{x^{\prime}+\frac{r}{V} x-r c=0}$.


\textbf{Problem:}
Subproblem 0: Find the polynomial solution of $\ddot{x}-x=t^{2}+t+1$, solving for $x(t)$.


\textbf{Solution:}
Since the constant term of the right-hand side is nonzero, the undetermined coefficients theorem asserts that there is a unique quadratic polynomial $a t^{2}+b t+c$ satisfying this equation. Substituting this form into the left side of the equation, we see that $a=-1,-b=1$, and $2 a-c=1$, so $b=-1$ and $c=-3$. Finally, $x(t) = \boxed{-t^2 - t - 3}$


\textbf{Problem:}
Preamble: In the following problems, take $a = \ln 2$ and $b = \pi / 3$. 

Subproblem 0: Given $a = \ln 2$ and $b = \pi / 3$, rewrite $e^{a+b i}$ in the form $x + yi$, where $x, y$ are real numbers. 


Solution: Using Euler's formula, we find that the answer is $\boxed{1+\sqrt{3} i}$.

Final answer: The final answer is 1+\sqrt{3} i. I hope it is correct.

Subproblem 1: Given $a = \ln 2$ and $b = \pi / 3$, rewrite $e^{2(a+b i)}$ in the form $x + yi$, where $x, y$ are real numbers.


Solution: $e^{n(a+b i)}=(1+\sqrt{3} i)^{n}$, so the answer is $\boxed{-2+2 \sqrt{3} i}$.

Final answer: The final answer is -2+2 \sqrt{3} i. I hope it is correct.

Subproblem 2: Rewrite $e^{3(a+b i)}$ in the form $x + yi$, where $x, y$ are real numbers. 


\textbf{Solution:}
$e^{n(a+b i)}=(1+\sqrt{3} i)^{n}$, so the answer is $\boxed{-8}$.


\textbf{Problem:}
Subproblem 0: Find a purely sinusoidal solution of $\frac{d^{4} x}{d t^{4}}-x=\cos (2 t)$.


\textbf{Solution:}
We choose an exponential input function whose real part is $\cos (2 t)$, namely $e^{2 i t}$. Since $p(s)=s^{4}-1$ and $p(2 i)=15 \neq 0$, the exponential response formula yields the solution $\frac{e^{2 i t}}{15}$. A sinusoidal solution to the original equation is given by the real part: $\boxed{\frac{\cos (2 t)}{15}}$. 


\textbf{Problem:}
Preamble: In the following problems, take $a = \ln 2$ and $b = \pi / 3$. 

Subproblem 0: Given $a = \ln 2$ and $b = \pi / 3$, rewrite $e^{a+b i}$ in the form $x + yi$, where $x, y$ are real numbers. 


Solution: Using Euler's formula, we find that the answer is $\boxed{1+\sqrt{3} i}$.

Final answer: The final answer is 1+\sqrt{3} i. I hope it is correct.

Subproblem 1: Given $a = \ln 2$ and $b = \pi / 3$, rewrite $e^{2(a+b i)}$ in the form $x + yi$, where $x, y$ are real numbers.


\textbf{Solution:}
$e^{n(a+b i)}=(1+\sqrt{3} i)^{n}$, so the answer is $\boxed{-2+2 \sqrt{3} i}$.


\textbf{Problem:}
Subproblem 0: Find a solution of $\ddot{x}+4 x=\cos (2 t)$, solving for $x(t)$, by using the ERF on a complex replacement. The ERF (Exponential Response Formula) states that a solution to $p(D) x=A e^{r t}$ is given by $x_{p}=A \frac{e^{r t}}{p(r)}$, as long as $\left.p (r\right) \neq 0$). The ERF with resonance assumes that $p(r)=0$ and states that a solution to $p(D) x=A e^{r t}$ is given by $x_{p}=A \frac{t e^{r t}}{p^{\prime}(r)}$, as long as $\left.p^{\prime} ( r\right) \neq 0$.


\textbf{Solution:}
The complex replacement of the equation is $\ddot{z}+4 z=e^{2 i t}$, with the characteristic polynomial $p(s)=s^{2}+4$. Because $p(2 i)=0$ and $p^{\prime}(2 i)=4 i \neq 0$, we need to use the Resonant ERF, which leads to $z_{p}=\frac{t e^{2 i t}}{4 i}$. A solution of the original equation is given by $x_{p}=\operatorname{Re}\left(z_{p}\right)=\boxed{\frac{t}{4} \sin (2 t)}$.


\textbf{Problem:}
Subproblem 0: Given the ordinary differential equation $\ddot{x}-a^{2} x=0$, where $a$ is a nonzero real-valued constant, find a solution $x(t)$ to this equation such that $x(0) = 1$ and $\dot{x}(0)=0$.


\textbf{Solution:}
First, notice that both $x(t)=e^{a t}$ and $x(t)=e^{-a t}$ are solutions to $\ddot{x}-a^{2} x=0$. Then for any constants $c_{1}$ and $c_{2}$, $x(t)=c_{1} e^{a t}+c_{2} e^{-a t}$ are also solutions to $\ddot{x}-a^{2} x=0$. Moreover, $x(0)=c_{1}+c_{2}$, and $\dot{x}(0)=a\left(c_{1}-c_{2}\right)$. Assuming $a \neq 0$, to satisfy the given conditions, we need $c_{1}+c_{2}=1$ and $a\left(c_{1}-c_{2}\right)=0$, which implies $c_{1}=c_{2}=1 / 2$. So $x(t)=\boxed{\frac{1}{2}(\exp{a*t} + \exp{-a*t})}$.


\textbf{Problem:}
Subproblem 0: Find the general solution of the differential equation $\dot{x}+2 x=e^{t}$, using $c$ for the arbitrary constant of integration which will occur.


\textbf{Solution:}
We can use integrating factors to get $(u x)^{\prime}=u e^{t}$ for $u=e^{2 t}$. Integrating yields $e^{2 t} x=e^{3 t} / 3+c$, or $x=\boxed{\frac{e^{t}} {3}+c e^{-2 t}}$. 


\textbf{Problem:}
Subproblem 0: Find a solution of $\ddot{x}+3 \dot{x}+2 x=t e^{-t}$ in the form $x(t)=u(t) e^{-t}$ for some function $u(t)$.  Use $C$ for an arbitrary constant, should it arise.


\textbf{Solution:}
$\dot{x}=\dot{u} e^{-t}-u e^{-t}$ and $\ddot{x}=\ddot{u} e^{-t}-2 \dot{u} e^{-t}+u e^{-t}$. Plugging into the equation leads to $e^{-t}(\ddot{u}+\dot{u})=t e^{-t}$. Cancelling off $e^{-t}$ from both sides, we get $\ddot{u}+\dot{u}=t$. To solve this equation for $u$, we use the undetermined coefficient method. However, the corresponding characteristic polynomial $p(s)=s^{2}+s$ has zero as its constant term. So set $w=\dot{u}$, then the equation can be rewritten as $\dot{w}+w=t$. This can be solved and one solution is $w=t-1$, and hence $\dot{u}=t-1$, and one solution for $u$ is $u=\frac{t^{2}}{2}-t+C$. Back to the original equation, one solution is given by $x=\boxed{\left(\frac{t^{2}}{2}-t+C\right) e^{-t}}$


\textbf{Problem:}
Subproblem 0: If the complex number $z$ is given by $z = 1+\sqrt{3} i$, what is the real part of $z^2$?


\textbf{Solution:}
$z^{2}$ has argument $2 \pi / 3$ and radius 4 , so by Euler's formula, $z^{2}=4 e^{i 2 \pi / 3}=-2+2 \sqrt{3} i$. Thus $a = -2, b = 2\sqrt 3$, so our answer is \boxed{-2}.


\textbf{Problem:}
Subproblem 0: Find a purely exponential solution of $\frac{d^{4} x}{d t^{4}}-x=e^{-2 t}$.


\textbf{Solution:}
The characteristic polynomial of the homogeneous equation is given by $p(s)=$ $s^{4}-1$. Since $p(-2)=15 \neq 0$, the exponential response formula gives the solution $\frac{e^{-2 t}}{p(-2)}=\boxed{\frac{e^{-2 t}}{15}}$.


\textbf{Problem:}
Preamble: The following subproblems refer to the exponential function $e^{-t / 2} \cos (3 t)$, which we will assume is a solution of the differential equation $m \ddot{x}+b \dot{x}+k x=0$. 

Subproblem 0: What is $b$ in terms of $m$? Write $b$ as a constant times a function of $m$.


\textbf{Solution:}
We can write $e^{-t / 2} \cos (3 t)=\operatorname{Re} e^{(-1 / 2 \pm 3 i) t}$, so $p(s)=m s^{2}+b s+k$ has solutions $-\frac{1}{2} \pm 3 i$. This means $p(s)=m(s+1 / 2-3 i)(s+1 / 2+3 i)=m\left(s^{2}+s+\frac{37}{4}\right)$. Then $b=\boxed{m}$, 


\textbf{Problem:}
Preamble: The following subproblems refer to the differential equation. $\ddot{x}+4 x=\sin (3 t)$

Subproblem 0: Find $A$ so that $A \sin (3 t)$ is a solution of $\ddot{x}+4 x=\sin (3 t)$.


Solution: We can find this by brute force. If $x=A \sin (3 t)$, then $\ddot{x}=-9 A \sin (3 t)$, so $\ddot{x}+4 x=-5 A \sin (3 t)$. Therefore, when $A=\boxed{-0.2}, x_{p}(t)=-\sin (3 t) / 5$ is a solution of the given equation.

Final answer: The final answer is -0.2. I hope it is correct.

Subproblem 1: What is the general solution, in the form $f_0(t) + C_1f_1(t) + C_2f_2(t)$, where $C_1, C_2$ denote arbitrary constants? 


\textbf{Solution:}
To find the general solution, we add to $x_{p}$ the general solution to the homogeneous equation $\ddot{x}+4 x=0$. The characteristic polynomial is $p(s)=s^{2}+4$, with roots $\pm 2 i$, so the general solution to $\ddot{x}+4 x=0$ is $C_{1} \sin (2 t)+C_{2} \cos (2 t)$. Therefore, the general solution to $\ddot{x}+4 x=\sin (3 t)$ is given by $\boxed{-\sin (3 t) / 5+ C_{1} \sin (2 t)+C_{2} \cos (2 t)}$.


\textbf{Problem:}
Subproblem 0: What is the smallest possible positive $k$ such that all functions $x(t)=A \cos (\omega t-\phi)$---where $\phi$ is an odd multiple of $k$---satisfy $x(0)=0$? \\


\textbf{Solution:}
$x(0)=A \cos \phi$. When $A=0$, then $x(t)=0$ for every $t$; when $A \neq 0$, $x(0)=0$ implies $\cos \phi=0$, and hence $\phi$ can be any odd multiple of $\pi / 2$, i.e., $\phi=\pm \pi / 2, \pm 3 \pi / 2, \pm 5 \pi / 2, \ldots$ this means $k=\boxed{\frac{\pi}{2}}$


\textbf{Problem:}
Preamble: The following subproblems refer to the differential equation $\ddot{x}+b \dot{x}+x=0$.\\

Subproblem 0: What is the characteristic polynomial $p(s)$ of $\ddot{x}+b \dot{x}+x=0$?


\textbf{Solution:}
The characteristic polynomial is $p(s)=\boxed{s^{2}+b s+1}$.


\textbf{Problem:}
Preamble: The following subproblems refer to the exponential function $e^{-t / 2} \cos (3 t)$, which we will assume is a solution of the differential equation $m \ddot{x}+b \dot{x}+k x=0$. 

Subproblem 0: What is $b$ in terms of $m$? Write $b$ as a constant times a function of $m$.


Solution: We can write $e^{-t / 2} \cos (3 t)=\operatorname{Re} e^{(-1 / 2 \pm 3 i) t}$, so $p(s)=m s^{2}+b s+k$ has solutions $-\frac{1}{2} \pm 3 i$. This means $p(s)=m(s+1 / 2-3 i)(s+1 / 2+3 i)=m\left(s^{2}+s+\frac{37}{4}\right)$. Then $b=\boxed{m}$, 

Final answer: The final answer is m. I hope it is correct.

Subproblem 1: What is $k$ in terms of $m$? Write $k$ as a constant times a function of $m$.


\textbf{Solution:}
Having found that $p(s)=m(s+1 / 2-3 i)(s+1 / 2+3 i)=m\left(s^{2}+s+\frac{37}{4}\right)$ in the previous subproblem, $k=\boxed{\frac{37}{4} m}$.


\textbf{Problem:}
Preamble: In the following problems, take $a = \ln 2$ and $b = \pi / 3$. 

Subproblem 0: Given $a = \ln 2$ and $b = \pi / 3$, rewrite $e^{a+b i}$ in the form $x + yi$, where $x, y$ are real numbers. 


Solution: Using Euler's formula, we find that the answer is $\boxed{1+\sqrt{3} i}$.

Final answer: The final answer is 1+\sqrt{3} i. I hope it is correct.

Subproblem 1: Given $a = \ln 2$ and $b = \pi / 3$, rewrite $e^{2(a+b i)}$ in the form $x + yi$, where $x, y$ are real numbers.


Solution: $e^{n(a+b i)}=(1+\sqrt{3} i)^{n}$, so the answer is $\boxed{-2+2 \sqrt{3} i}$.

Final answer: The final answer is -2+2 \sqrt{3} i. I hope it is correct.

Subproblem 2: Rewrite $e^{3(a+b i)}$ in the form $x + yi$, where $x, y$ are real numbers. 


Solution: $e^{n(a+b i)}=(1+\sqrt{3} i)^{n}$, so the answer is $\boxed{-8}$.

Final answer: The final answer is -8. I hope it is correct.

Subproblem 3: Rewrite $e^{4(a+b i)}$ in the form $x + yi$, where $x, y$ are real numbers. 


\textbf{Solution:}
 $e^{n(a+b i)}=(1+\sqrt{3} i)^{n}$, so the answer is $\boxed{-8-8 \sqrt{3} i}$.


\textbf{Problem:}
Subproblem 0: Rewrite the function $\operatorname{Re} \frac{e^{i t}}{2+2 i}$ in the form $A \cos (\omega t-\phi)$. It may help to begin by drawing a right triangle with sides $a$ and $b$. 


\textbf{Solution:}
$e^{i t}=\cos (t)+i \sin (t)$, and $\frac{1}{2+2 i}=\frac{1-i}{4}$. the real part is then $\frac{1}{4} \cos (t)+$ $\frac{1}{4} \sin (t)$. The right triangle here has hypotenuse $\frac{\sqrt{2}}{4}$ and argument $\pi / 4$, so $f(t)=\boxed{\frac{\sqrt{2}}{4} \cos (t-\pi / 4)}$.


\textbf{Problem:}
Preamble: The following subproblems refer to the differential equation $\ddot{x}+b \dot{x}+x=0$.\\

Subproblem 0: What is the characteristic polynomial $p(s)$ of $\ddot{x}+b \dot{x}+x=0$?


Solution: The characteristic polynomial is $p(s)=\boxed{s^{2}+b s+1}$.

Final answer: The final answer is s^{2}+b s+1. I hope it is correct.

Subproblem 1: For what value of $b$ does $\ddot{x}+b \dot{x}+x=0$ exhibit critical damping?


\textbf{Solution:}
To exhibit critical damping, the characteristic polynomial $s^{2}+b s+1$ must be a square, i.e., $(s-k)^{2}$ for some $k$. Multiplying and comparing yields $-2 k=b$ and $k^{2}=1$. Therefore, $b$ could be either one of $=-2, 2$. When $b=-2, e^{t}$ is a solution, and it exhibits exponential growth instead of damping, so we reject that value of $b$. Therefore, the value of $b$ for which $\ddot{x}+b \dot{x}+x=0$ exhibits critical damping is $b=\boxed{2}$


\textbf{Problem:}
Subproblem 0: Find the general (complex-valued) solution of the differential equation $\dot{z}+2 z=e^{2 i t}$, using $C$ to stand for any complex-valued integration constants which may arise.


\textbf{Solution:}
Using integrating factors, we get $e^{2 t} z=e^{(2+2 i) t} /(2+2 i)+C$, or $z=\boxed{\frac{e^{2 i t}}{(2+2 i)}+C e^{-2 t}}$, where $C$ is any complex number.


\section{Dynamics and Control (2.003 Spring 2005)}

\textbf{Problem:}
Preamble: Consider the first-order system
\[
\tau \dot{y}+y=u
\]
driven with a unit step from zero initial conditions. The input to this system is \(u\) and the output is \(y\). 

Subproblem 0: Derive and expression for the settling time \(t_{s}\), where the settling is to within an error \(\pm \Delta\) from the final value of 1.


\textbf{Solution:}
Rise and Settling Times.  We are given the first-order transfer function
\[
H(s)=\frac{1}{\tau s+1}
\]
The response to a unit step with zero initial conditions will be \(y(t)=1-e^{-t / \tau}\). To determine the amount of time it take \(y\) to settle to within \(\Delta\) of its final value, we want to find the time \(t_{s}\) such that \(y\left(t_{s}\right)=1-\Delta\). Thus, we obtain
\[
\begin{aligned}
&\Delta=e^{-t_{s} / \tau} \\
&t_{s}=\boxed{-\tau \ln \Delta}
\end{aligned}
\]


\textbf{Problem:}
Preamble: Consider the first-order system
\[
\tau \dot{y}+y=u
\]
driven with a unit step from zero initial conditions. The input to this system is \(u\) and the output is \(y\). 

Subproblem 0: Derive and expression for the settling time \(t_{s}\), where the settling is to within an error \(\pm \Delta\) from the final value of 1.


Solution: Rise and Settling Times.  We are given the first-order transfer function
\[
H(s)=\frac{1}{\tau s+1}
\]
The response to a unit step with zero initial conditions will be \(y(t)=1-e^{-t / \tau}\). To determine the amount of time it take \(y\) to settle to within \(\Delta\) of its final value, we want to find the time \(t_{s}\) such that \(y\left(t_{s}\right)=1-\Delta\). Thus, we obtain
\[
\begin{aligned}
&\Delta=e^{-t_{s} / \tau} \\
&t_{s}=\boxed{-\tau \ln \Delta}
\end{aligned}
\]

Final answer: The final answer is -\tau \ln \Delta. I hope it is correct.

Subproblem 1: Derive an expression for the \(10-90 \%\) rise time \(t_{r}\) in terms of $\tau$.


\textbf{Solution:}
The \(10-90 \%\) rise time \(t_{r}\) may be thought of as the difference between the \(90 \%\) settling time \((\Delta=0.1)\) and the \(10 \%\) settling time \((\Delta=0.9)\).
\[
t_{r}=t_{\Delta=0.1}-t_{\Delta=0.9}
\]
Therefore, we find \(t_{r}=\boxed{2.2 \tau}\).


\textbf{Problem:}
Preamble: For each of the functions $y(t)$, find the Laplace Transform $Y(s)$ :

Subproblem 0: $y(t)=e^{-a t}$


\textbf{Solution:}
This function is one of the most widely used in dynamic systems, so we memorize its transform!
\[
Y(s)=\boxed{\frac{1}{s+a}}
\]


\textbf{Problem:}
Preamble: For each Laplace Transform \(Y(s)\), find the function \(y(t)\) :

Subproblem 0: \[
Y(s)=\boxed{\frac{1}{(s+a)(s+b)}}
\]


Solution: We can simplify with partial fractions:
\[
Y(s)=\frac{1}{(s+a)(s+b)}=\frac{C}{s+a}+\frac{D}{s+b}
\]
find the constants \(C\) and \(D\) by setting \(s=-a\) and \(s=-b\)
\[
\begin{aligned}
\frac{1}{(s+a)(s+b)} &=\frac{C}{s+a}+\frac{D}{s+b} \\
1 &=C(s+b)+D(s+a) \\
C &=\frac{1}{b-a} \\
D &=\frac{1}{a-b}
\end{aligned}
\]
therefore
\[
Y(s)=\frac{1}{b-a} \frac{1}{s+a}-\frac{1}{b-a} \frac{1}{s+b}
\]
By looking up the inverse Laplace Transform of \(\frac{1}{s+b}\), we find the total solution \(y(t)\)
\[
y(t)=\boxed{\frac{1}{b-a}\left(e^{-a t}-e^{-b t}\right)}
\]

Final answer: The final answer is \frac{1}{b-a}\left(e^{-a t}-e^{-b t}\right). I hope it is correct.

Subproblem 1: \[
Y(s)=\frac{s}{\frac{s^{2}}{\omega_{n}^{2}}+\frac{2 \zeta}{\omega_{n}} s+1}
\]
You may assume that $\zeta < 1$.


\textbf{Solution:}
First, note that the transform is
\[
\begin{aligned}
Y(s) &=\frac{s}{\frac{s^{2}}{\omega_{n}^{2}}+\frac{2 \zeta}{\omega_{n}} s+1} \\
&=s \cdot \frac{\omega_{n}^{2}}{s^{2}+2 \zeta \omega_{n} s+\omega_{n}^{2}}
\end{aligned}
\]
We will solve this problem using the property
\[
\frac{d f}{d t}=s F(s)-f(0)
\]
therefore
\[
\begin{aligned}
y(t) &=\frac{d}{d t}\left(\frac{\omega_{n}}{\sqrt{1-\zeta^{2}}} e^{-\zeta \omega_{n} t} \sin \left(\omega_{n} \sqrt{1-\zeta^{2}} t\right)\right) \\
&=\boxed{\omega_{n}^{2} e^{-\zeta \omega_{n} t} \cos \left(\omega_{n} \sqrt{1-\zeta^{2}} t\right)-\frac{\zeta \omega_{n}^{2}}{\sqrt{1-\zeta^{2}}} e^{-\zeta \omega_{n} t} \sin \left(\omega_{n} \sqrt{1-\zeta^{2}} t\right)}
\end{aligned}
\]
remember that for this form to be correct, \(\zeta\) must be less than 1 .


\textbf{Problem:}
Subproblem 0: A signal \(x(t)\) is given by
\[
x(t)=\left(e^{-t}-e^{-1}\right)\left(u_{s}(t)-u_{s}(t-1)\right)
\]
Calculate its Laplace transform \(X(s)\). Make sure to clearly show the steps in your calculation. 


\textbf{Solution:}
Simplify the expression in to a sum of terms,
\[
x(t)=e^{-t} u_{s}(t)-e^{-1} u_{s}(t)-e^{-t} u_{s}(t-1)+e^{-1} u_{s}(t-1)
\]
Now take the Laplace transform of the first, second and fourth terms,
\[
X(s)=\frac{1}{s+1}-\frac{e^{-1}}{s}-\mathcal{L} e^{-t} u_{s}(t-1)+\frac{e^{-1} e^{-s}}{s}
\]
The third term requires some massaging to get it in a form available on the table. The term can be modified into the form of a time delay, by factoring out \(e^{-1}\).
\[
\mathcal{L}\left\{e^{-t} u_{s}(t-1)\right\}=e^{-1} \mathcal{L}\left\{e^{-(t-1)} u_{s}(t-1)\right\}
\]
Now applying the Laplace Transform for a time delay from the table
\[
e^{-1} \mathcal{L}\left\{e^{-(t-1)} u_{s}(t-1)\right\}=\frac{e^{-1} e^{-s}}{s+1}
\]
Substituting this piece back into the expression above gives the solution
\[
X(s)=\boxed{\frac{1}{s+1}-\frac{e^{-1}}{s}-\frac{e^{-1} e^{-s}}{s+1}+\frac{e^{-1} e^{-s}}{s}}
\]


\textbf{Problem:}
Preamble: You are given an equation of motion of the form:
\[
\dot{y}+5 y=10 u
\]

Subproblem 0: What is the time constant for this system?


Solution: We find the homogenous solution, solving:
\[
\dot{y}+5 y=0
\]
by trying a solution of the form $y=A \cdot e^{s, t}$.
Calculation:
\[
\dot{y}=A \cdot s \cdot e^{s \cdot t} \mid \Rightarrow A \cdot s \cdot e^{s t}+5 A \cdot e^{s t}=0
\]
yields that $s=-5$, meaning the solution is $y=A \cdot e^{-5 \cdot t}=A \cdot e^{-t / \tau}$, meaning $\tau = \boxed{0.2}$.

Final answer: The final answer is 0.2. I hope it is correct.

Subproblem 1: If \(u=10\), what is the final or steady-state value for \(y(t)\)? 


\textbf{Solution:}
Steady state implies $\dot{y} = 0$, so in the case when $u=10$, we get $y=\boxed{20}$.


\textbf{Problem:}
Subproblem 0: A signal \(w(t)\) is defined as
\[
w(t)=u_{s}(t)-u_{s}(t-T)
\]
where \(T\) is a fixed time in seconds and \(u_{s}(t)\) is the unit step. Compute the Laplace transform \(W(s)\) of \(w(t)\). Show your work.


\textbf{Solution:}
The Laplace Transform of \(x(t)\) is defined as
\[
\mathcal{L}[x(t)]=X(s)=\int_{0}^{\infty} x(t) e^{-s t} d t
\]
therefore
\[
\begin{aligned}
W(s) &=\int_{0}^{\infty} e^{-s t} d t-\left(\int_{0}^{T} 0 d t+\int_{T}^{\infty} e^{-s t} d t\right) \\
&=-\left.\frac{1}{s} e^{-s t}\right|_{0} ^{\infty}-\left(0+-\left.\frac{1}{s} e^{-s t}\right|_{T} ^{\infty}\right) \\
&=\boxed{\frac{1}{s}-\frac{1}{s} e^{-s T}}
\end{aligned}
\]


\textbf{Problem:}
Preamble: Assume that we apply a unit step in force separately to a mass \(m\), a dashpot \(c\), and a spring \(k\). The mass moves in inertial space. The spring and dashpot have one end connected to inertial space (reference velocity \(=0\) ), and the force is applied to the other end.  Assume zero initial velocity and position for the elements.
Recall that the unit step function \(u_{S}(t)\) is defined as \(u_{S}(t)=0 ; t<0\) and \(u_{S}(t)=1 ; t \geq 0\). We will also find it useful to introduce the unit impulse function \(\delta(t)\) which can be defined via
\[
u_{S}(t)=\int_{-\infty}^{t} \delta(\tau) d \tau
\]
This means that we can also view the unit impulse as the derivative of the unit step:
\[
\delta(t)=\frac{d u_{S}(t)}{d t}
\]

Subproblem 0: Solve for the resulting velocity of the mass.


\textbf{Solution:}
\[
\begin{aligned}
m \ddot{x}_{m} &=u_{s}(t) \\
\dot{x}_{m}=v_{m} &=\int_{-\infty}^{t} \frac{1}{m} u_{s}(t) d t=\boxed{\frac{1}{m} t} \\
\end{aligned}
\]


\textbf{Problem:}
Preamble: For each of the functions $y(t)$, find the Laplace Transform $Y(s)$ :

Subproblem 0: $y(t)=e^{-a t}$


Solution: This function is one of the most widely used in dynamic systems, so we memorize its transform!
\[
Y(s)=\boxed{\frac{1}{s+a}}
\]

Final answer: The final answer is \frac{1}{s+a}. I hope it is correct.

Subproblem 1: $y(t)=e^{-\sigma t} \sin \omega_{d} t$


Solution: \[
Y(s)=\boxed{\frac{\omega_{d}}{(s+\sigma)^{2}+\omega_{d}^{2}}}
\]

Final answer: The final answer is \frac{\omega_{d}}{(s+\sigma)^{2}+\omega_{d}^{2}}. I hope it is correct.

Subproblem 2: $y(t)=e^{-\sigma t} \cos \omega_{d} t$


\textbf{Solution:}
\[
Y(s)=\boxed{\frac{s+\sigma}{(s+\sigma)^{2}+\omega_{d}^{2}}}
\]


\textbf{Problem:}
Preamble: For each of the functions $y(t)$, find the Laplace Transform $Y(s)$ :

Subproblem 0: $y(t)=e^{-a t}$


Solution: This function is one of the most widely used in dynamic systems, so we memorize its transform!
\[
Y(s)=\boxed{\frac{1}{s+a}}
\]

Final answer: The final answer is \frac{1}{s+a}. I hope it is correct.

Subproblem 1: $y(t)=e^{-\sigma t} \sin \omega_{d} t$


\textbf{Solution:}
\[
Y(s)=\boxed{\frac{\omega_{d}}{(s+\sigma)^{2}+\omega_{d}^{2}}}
\]


\textbf{Problem:}
Preamble: Consider the mass \(m\) sliding horizontally under the influence of the applied force \(f\) and a friction force which can be approximated by a linear friction element with coefficient \(b\). 

Subproblem 0: Formulate the state-determined equation of motion for the velocity \(v\) as output and the force \(f\) as input.


\textbf{Solution:}
The equation of motion is
\[
\boxed{m \frac{d v}{d t}+b v=f} \quad \text { or } \quad \frac{d v}{d t}=-\frac{b}{m} v+\frac{1}{m} f
\]


\textbf{Problem:}
Preamble: Consider the rotor with moment of inertia \(I\) rotating under the influence of an applied torque \(T\) and the frictional torques from two bearings, each of which can be approximated by a linear frictional element with coefficient \(B\).

Subproblem 0: Formulate the state-determined equation of motion for the angular velocity $\omega$ as output and the torque $T$ as input.


Solution: The equation of motion is
\[
\boxed{I \frac{d \omega}{d t}+2 B \omega=T} \quad \text { or } \quad \frac{d \omega}{d t}=-\frac{2 B}{I} \omega+\frac{1}{I} T
\]

Final answer: The final answer is I \frac{d \omega}{d t}+2 B \omega=T. I hope it is correct.

Subproblem 1: Consider the case where:
\[
\begin{aligned}
I &=0.001 \mathrm{~kg}-\mathrm{m}^{2} \\
B &=0.005 \mathrm{~N}-\mathrm{m} / \mathrm{r} / \mathrm{s}
\end{aligned}
\]
What is the steady-state velocity \(\omega_{s s}\), in radians per second, when the input is a constant torque of 10 Newton-meters?


\textbf{Solution:}
The steady-state angular velocity, when \(T=10\) Newton-meters, and \(I=0.001 \mathrm{~kg}-\mathrm{m}^{2}\), and \(B=0.005 \mathrm{~N}-\mathrm{m} / \mathrm{r} / \mathrm{s}\) is
\[
\omega_{s s}=\frac{T}{2 B}=\frac{10}{2(0.005)}=\boxed{1000} \mathrm{r} / \mathrm{s}
\]


\textbf{Problem:}
Preamble: Consider the mass \(m\) sliding horizontally under the influence of the applied force \(f\) and a friction force which can be approximated by a linear friction element with coefficient \(b\). 

Subproblem 0: Formulate the state-determined equation of motion for the velocity \(v\) as output and the force \(f\) as input.


Solution: The equation of motion is
\[
\boxed{m \frac{d v}{d t}+b v=f} \quad \text { or } \quad \frac{d v}{d t}=-\frac{b}{m} v+\frac{1}{m} f
\]

Final answer: The final answer is m \frac{d v}{d t}+b v=f. I hope it is correct.

Subproblem 1: Consider the case where:
\[
\begin{aligned}
m &=1000 \mathrm{~kg} \\
b &=100 \mathrm{~N} / \mathrm{m} / \mathrm{s}
\end{aligned}
\]
What is the steady-state velocity \(v_{s s}\) when the input is a constant force of 10 Newtons? Answer in meters per second.


\textbf{Solution:}
The steady-state velocity, when \(f=10\) Newtons, and \(m=1000 \mathrm{~kg}\), and \(b=100 \mathrm{~N} / \mathrm{m} / \mathrm{s}\) is
\[
v_{s s}=\frac{f}{b}=\frac{10}{100}=\boxed{0.10} \mathrm{~m} / \mathrm{s}
\]


\textbf{Problem:}
Subproblem 0: Obtain the inverse Laplace transform of the following frequency-domain expression: $F(s) = -\frac{(4 s-10)}{s(s+2)(s+5)}$.
Use $u(t)$ to denote the unit step function.


\textbf{Solution:}
Using partial fraction expansion, the above can be rewritten as 
\[
F(s) = \frac{1}{s} - \frac{3}{s+2} + \frac{2}{s+5}
\]
Apply the inverse Laplace transform, then we end up with
\[
f(t) = \boxed{(1 - 3e^{-2t} + 2e^{-5t}) u(t)}
\]


\textbf{Problem:}
Subproblem 0: A signal has a Laplace transform
\[
X(s)=b+\frac{a}{s(s+a)}
\]
where \(a, b>0\), and with a region of convergence of \(|s|>0\). Find \(x(t), t>0\).


\textbf{Solution:}
Each term of \(X(s)\) can be evaluated directly using a table of Laplace Transforms:
\[
\mathcal{L}^{-1}\{b\}=b \delta(t)
\]
and
\[
\mathcal{L}^{-1}\left\{\frac{a}{s(s+a)}\right\}=1-e^{-a t}
\]
The final result is then
\[
\mathcal{L}^{-1}\{X(s)\}=\boxed{b \delta(t)+1-e^{-a t}}
\]


\textbf{Problem:}
Preamble: For each Laplace Transform \(Y(s)\), find the function \(y(t)\) :

Subproblem 0: \[
Y(s)=\boxed{\frac{1}{(s+a)(s+b)}}
\]


\textbf{Solution:}
We can simplify with partial fractions:
\[
Y(s)=\frac{1}{(s+a)(s+b)}=\frac{C}{s+a}+\frac{D}{s+b}
\]
find the constants \(C\) and \(D\) by setting \(s=-a\) and \(s=-b\)
\[
\begin{aligned}
\frac{1}{(s+a)(s+b)} &=\frac{C}{s+a}+\frac{D}{s+b} \\
1 &=C(s+b)+D(s+a) \\
C &=\frac{1}{b-a} \\
D &=\frac{1}{a-b}
\end{aligned}
\]
therefore
\[
Y(s)=\frac{1}{b-a} \frac{1}{s+a}-\frac{1}{b-a} \frac{1}{s+b}
\]
By looking up the inverse Laplace Transform of \(\frac{1}{s+b}\), we find the total solution \(y(t)\)
\[
y(t)=\boxed{\frac{1}{b-a}\left(e^{-a t}-e^{-b t}\right)}
\]


\textbf{Problem:}
Preamble: Consider the rotor with moment of inertia \(I\) rotating under the influence of an applied torque \(T\) and the frictional torques from two bearings, each of which can be approximated by a linear frictional element with coefficient \(B\).

Subproblem 0: Formulate the state-determined equation of motion for the angular velocity $\omega$ as output and the torque $T$ as input.


\textbf{Solution:}
The equation of motion is
\[
\boxed{I \frac{d \omega}{d t}+2 B \omega=T} \quad \text { or } \quad \frac{d \omega}{d t}=-\frac{2 B}{I} \omega+\frac{1}{I} T
\]


\textbf{Problem:}
Subproblem 0: Obtain the inverse Laplace transform of the following frequency-domain expression: $F(s) = \frac{4}{s^2(s^2+4)}$.
Use $u(t)$ to denote the unit step function.


\textbf{Solution:}
Since $F(s) = \frac{1}{s^2} + \frac{-1}{s^2+4}$, its inverse Laplace transform is 
\[
f(t) = \boxed{(t + \frac{1}{2} \sin{2t}) u(t)}
\]


\textbf{Problem:}
Preamble: This problem considers the simple RLC circuit, in which a voltage source $v_{i}$ is in series with a resistor $R$, inductor $L$, and capacitor $C$.  We measure the voltage $v_{o}$ across the capacitor.  $v_{i}$ and $v_{o}$ share a ground reference.

Subproblem 0: Calculate the transfer function \(V_{o}(s) / V_{i}(s)\).


\textbf{Solution:}
Using the voltage divider relationship:
\[
\begin{aligned}
V_{o}(s) &=\frac{Z_{e q}}{Z_{\text {total }}}V_{i}(s)=\frac{\frac{1}{C s}}{R+L s+\frac{1}{C s}} V_{i}(s) \\
\frac{V_{o}(s)}{V_{i}(s)} &=\boxed{\frac{1}{L C s^{2}+R C s+1}}
\end{aligned}
\]


\textbf{Problem:}
Preamble: You are given an equation of motion of the form:
\[
\dot{y}+5 y=10 u
\]

Subproblem 0: What is the time constant for this system?


\textbf{Solution:}
We find the homogenous solution, solving:
\[
\dot{y}+5 y=0
\]
by trying a solution of the form $y=A \cdot e^{s, t}$.
Calculation:
\[
\dot{y}=A \cdot s \cdot e^{s \cdot t} \mid \Rightarrow A \cdot s \cdot e^{s t}+5 A \cdot e^{s t}=0
\]
yields that $s=-5$, meaning the solution is $y=A \cdot e^{-5 \cdot t}=A \cdot e^{-t / \tau}$, meaning $\tau = \boxed{0.2}$.


\textbf{Problem:}
Preamble: This problem considers the simple RLC circuit, in which a voltage source $v_{i}$ is in series with a resistor $R$, inductor $L$, and capacitor $C$.  We measure the voltage $v_{o}$ across the capacitor.  $v_{i}$ and $v_{o}$ share a ground reference.

Subproblem 0: Calculate the transfer function \(V_{o}(s) / V_{i}(s)\).


Solution: Using the voltage divider relationship:
\[
\begin{aligned}
V_{o}(s) &=\frac{Z_{e q}}{Z_{\text {total }}}V_{i}(s)=\frac{\frac{1}{C s}}{R+L s+\frac{1}{C s}} V_{i}(s) \\
\frac{V_{o}(s)}{V_{i}(s)} &=\boxed{\frac{1}{L C s^{2}+R C s+1}}
\end{aligned}
\]

Final answer: The final answer is \frac{1}{L C s^{2}+R C s+1}. I hope it is correct.

Subproblem 1: Let \(L=0.01 \mathrm{H}\). Choose the value of $C$ such that \(\omega_{n}=10^{5}\) and \(\zeta=0.05\).  Give your answer in Farads.


\textbf{Solution:}
$C=\frac{1}{\omega_{n}^{2}L}=\boxed{1e-8}[\mathrm{~F}]$


\textbf{Problem:}
Preamble: Here we consider a system described by the differential equation
\[
\ddot{y}+10 \dot{y}+10000 y=0 .
\]

Subproblem 0: What is the value of the natural frequency \(\omega_{n}\) in radians per second?


\textbf{Solution:}
$\omega_{n}=\sqrt{\frac{k}{m}}$
So
$\omega_{n} =\boxed{100} \mathrm{rad} / \mathrm{s}$


\textbf{Problem:}
Preamble: Consider a circuit in which a voltage source of voltage in $v_{i}(t)$ is connected in series with an inductor $L$ and capacitor $C$.  We consider the voltage across the capacitor $v_{o}(t)$ to be the output of the system.
Both $v_{i}(t)$ and $v_{o}(t)$ share ground reference.

Subproblem 0: Write the governing differential equation for this circuit.


\textbf{Solution:}
Using Kirchoff Current Law at the node between the inductor and capacitor with the assumed currents both positive into the node gives the following:
\[
\begin{gathered}
i_{L}+i_{C}=0 \\
i_{L}=\frac{1}{L} \int v_{L} d t \\
i_{C}=C \frac{d v_{c}}{d t}
\end{gathered}
\]
The above equation must be differentiated before substituting for the currents and from the direction of our assumed currents, \(v_{L}=v_{i}-v_{o}\) and \(v_{C}=0-v_{o}\). The governing differential equation is then
\[
\boxed{\frac{d^{2} v_{o}}{d t^{2}}+\frac{v_{o}}{L C}=\frac{v_{i}}{L C}}
\]


\textbf{Problem:}
Subproblem 0: Write (but don't solve) the equation of motion for a pendulum consisting of a mass $m$ attached to a rigid massless rod, suspended from the ceiling and free to rotate in a single vertical plane.  Let the rod (of length $l$) make an angle of $\theta$ with the vertical.  Gravity ($mg$) acts directly downward, the system input is a horizontal external force $f(t)$, and the system output is the angle $\theta(t)$.  
Note: Do NOT make the small-angle approximation in your equation.


\textbf{Solution:}
From force balance, we can derive the equation of motion. Choosing the system variable system variable $\theta(t)$ with polar coordinates, we don't need to care about tension on the rod and centrifugal force.
We can use the relation between torque and angular momentum to immediately write down the equation for $\theta(t)$:
\[
m l^{2} \ddot{\theta}(t)-m g l \sin \theta(t)=f(t) l \cos \theta(t) .
\]
Dividing both sides by $l$ gives:
\[
\boxed{m l \ddot{\theta}(t)-m g \sin \theta(t)=f(t) \cos \theta(t)} .
\]
Note that inertia of the mass with respect to the rotation axis is $m l^{2}$. It is a non linear differential equation because it has $\sin \theta(t)$ term.


\textbf{Problem:}
Preamble: Here we consider a system described by the differential equation
\[
\ddot{y}+10 \dot{y}+10000 y=0 .
\]

Subproblem 0: What is the value of the natural frequency \(\omega_{n}\) in radians per second?


Solution: $\omega_{n}=\sqrt{\frac{k}{m}}$
So
$\omega_{n} =\boxed{100} \mathrm{rad} / \mathrm{s}$

Final answer: The final answer is 100. I hope it is correct.

Subproblem 1: What is the value of the damping ratio \(\zeta\)? 


Solution: $\zeta=\frac{b}{2 \sqrt{k m}}$
So
$\zeta =\boxed{0.05}$

Final answer: The final answer is 0.05. I hope it is correct.

Subproblem 2: What is the value of the damped natural frequency \(\omega_{d}\) in radians per second? Give your answer to three significant figures.


\textbf{Solution:}
$\omega_{d}=\omega_{n} \sqrt{1-\zeta^{2}}$
So
$\omega_{d}=\boxed{99.9} \mathrm{rad} / \mathrm{s}$


\textbf{Problem:}
Preamble: Here we consider a system described by the differential equation
\[
\ddot{y}+10 \dot{y}+10000 y=0 .
\]

Subproblem 0: What is the value of the natural frequency \(\omega_{n}\) in radians per second?


Solution: $\omega_{n}=\sqrt{\frac{k}{m}}$
So
$\omega_{n} =\boxed{100} \mathrm{rad} / \mathrm{s}$

Final answer: The final answer is 100. I hope it is correct.

Subproblem 1: What is the value of the damping ratio \(\zeta\)? 


\textbf{Solution:}
$\zeta=\frac{b}{2 \sqrt{k m}}$
So
$\zeta =\boxed{0.05}$


\section{Relativity (8.033 Fall 2006)}

\textbf{Problem:}
Subproblem 0: What is the speed of light in meters/second to 1 significant figure? Use the format $a \times 10^{b}$ where a and b are numbers. 


\textbf{Solution:}
$\boxed{3e8}$ m/s.


\textbf{Problem:}
Preamble: Give each of the following quantities to the nearest power of 10 and in the units requested. 

Subproblem 0: Age of our universe when most He nuclei were formed in minutes: 


Solution: \boxed{1} minute.

Final answer: The final answer is 1. I hope it is correct.

Subproblem 1: Age of our universe when hydrogen atoms formed in years:


Solution: \boxed{400000} years.

Final answer: The final answer is 400000. I hope it is correct.

Subproblem 2: Age of our universe today in Gyr:


Solution: \boxed{10} Gyr.

Final answer: The final answer is 10. I hope it is correct.

Subproblem 3: Number of stars in our Galaxy: (Please format your answer as 'xen' representing $x * 10^n$)


\textbf{Solution:}
\boxed{1e11}.


\textbf{Problem:}
Preamble: In a parallel universe, the Boston baseball team made the playoffs.

Subproblem 0: Manny Relativirez hits the ball and starts running towards first base at speed $\beta$. How fast is he running, given that he sees third base $45^{\circ}$ to his left (as opposed to straight to his left before he started running)? Assume that he is still very close to home plate. Give your answer in terms of the speed of light, $c$.


\textbf{Solution:}
Using the aberration formula with $\cos \theta^{\prime}=-1 / \sqrt{2}, \beta=1 / \sqrt{2}$, so $v=\boxed{\frac{1}{\sqrt{2}}c}$.


\textbf{Problem:}
Preamble: In the Sun, one of the processes in the He fusion chain is $p+p+e^{-} \rightarrow d+\nu$, where $d$ is a deuteron. Make the approximations that the deuteron rest mass is $2 m_{p}$, and that $m_{e} \approx 0$ and $m_{\nu} \approx 0$, since both the electron and the neutrino have negligible rest mass compared with the proton rest mass $m_{p}$.

Subproblem 0: In the lab frame, the two protons have the same energy $\gamma m_{p}$ and impact angle $\theta$, and the electron is at rest. Calculate the energy $E_{\nu}$ of the neutrino in the rest frame of the deuteron in terms of $\theta, m_{p}$ and $\gamma$.


\textbf{Solution:}
Use the fact that the quantity $E^{2}-p^{2} c^{2}$ is invariant. In the deutron's rest frame, after the collison:
\[
\begin{aligned}
E^{2}-p^{2} c^{2} &=\left(2 m_{p} c^{2}+E_{\nu}\right)^{2}-E_{\nu}^{2} \\
&=4 m_{p}^{2} c^{4}+4 m_{p} c^{2} E_{\nu}=4 m_{p} c^{2}\left(m_{p} c^{2}+E_{\nu}\right)
\end{aligned}
\]
In the lab frame, before collison:
\[
\begin{aligned}
E^{2}-p^{2} c^{2} &=\left(2 E_{p}\right)^{2}-\left(2 p_{p} \cos \theta c\right)^{2} \\
&=\left(2 \gamma m_{p} c^{2}\right)^{2}-\left(2 \gamma \beta m_{p} \cos \theta c^{2}\right)^{2}
\end{aligned}
\]
Use $\gamma^{2} \beta^{2}=\left(\gamma^{2}-1\right)$ in the second term and simplify the algebra to find
\[
E^{2}-p^{2} c^{2}=4 m_{p}^{2} c^{4}\left(\gamma^{2}-\left(\gamma^{2}-1\right) \cos ^{2} \theta\right)
\]
Equating the invariants in the two frames, we have
\[
\begin{aligned}
4 m_{p} c^{2}\left(m_{p} c^{2}+E_{\nu}\right) &=4 m_{p}^{2} c^{4}\left(\gamma^{2}-\left(\gamma^{2}-1\right) \cos ^{2} \theta\right) \\
\Rightarrow E_{\nu} &= \boxed{m_{p} c^{2}\left(\gamma^{2}-1\right) \sin ^{2} \theta}
\end{aligned}
\]


\textbf{Problem:}
Preamble: In a parallel universe, the Boston baseball team made the playoffs.

Subproblem 0: Manny Relativirez hits the ball and starts running towards first base at speed $\beta$. How fast is he running, given that he sees third base $45^{\circ}$ to his left (as opposed to straight to his left before he started running)? Assume that he is still very close to home plate. Give your answer in terms of the speed of light, $c$.


Solution: Using the aberration formula with $\cos \theta^{\prime}=-1 / \sqrt{2}, \beta=1 / \sqrt{2}$, so $v=\boxed{\frac{1}{\sqrt{2}}c}$.

Final answer: The final answer is \frac{1}{\sqrt{2}}c. I hope it is correct.

Subproblem 1: A player standing on third base is wearing red socks emitting light of wavelength $\lambda_{\text {red}}$. What wavelength does Manny see in terms of $\lambda_{\text {red}}$?


\textbf{Solution:}
Using the doppler shift formula, $\lambda^{\prime}= \boxed{\lambda_{\text {red}} / \sqrt{2}}$.


\textbf{Problem:}
Preamble: Give each of the following quantities to the nearest power of 10 and in the units requested. 

Subproblem 0: Age of our universe when most He nuclei were formed in minutes: 


Solution: \boxed{1} minute.

Final answer: The final answer is 1. I hope it is correct.

Subproblem 1: Age of our universe when hydrogen atoms formed in years:


Solution: \boxed{400000} years.

Final answer: The final answer is 400000. I hope it is correct.

Subproblem 2: Age of our universe today in Gyr:


\textbf{Solution:}
\boxed{10} Gyr.


\textbf{Problem:}
Subproblem 0: How many down quarks does a tritium ($H^3$) nucleus contain?


\textbf{Solution:}
\boxed{5}.


\textbf{Problem:}
Subproblem 0: How many up quarks does a tritium ($H^3$) nucleus contain?


\textbf{Solution:}
\boxed{4}.


\textbf{Problem:}
Preamble: Give each of the following quantities to the nearest power of 10 and in the units requested. 

Subproblem 0: Age of our universe when most He nuclei were formed in minutes: 


\textbf{Solution:}
\boxed{1} minute.


\textbf{Problem:}
Preamble: Give each of the following quantities to the nearest power of 10 and in the units requested. 

Subproblem 0: Age of our universe when most He nuclei were formed in minutes: 


Solution: \boxed{1} minute.

Final answer: The final answer is 1. I hope it is correct.

Subproblem 1: Age of our universe when hydrogen atoms formed in years:


Solution: \boxed{400000} years.

Final answer: The final answer is 400000. I hope it is correct.

Subproblem 2: Age of our universe today in Gyr:


Solution: \boxed{10} Gyr.

Final answer: The final answer is 10. I hope it is correct.

Subproblem 3: Number of stars in our Galaxy: (Please format your answer as 'xen' representing $x * 10^n$)


Solution: \boxed{1e11}.

Final answer: The final answer is 1e11. I hope it is correct.

Subproblem 4: Light travel time to closest star (Sun!:) in minutes. (Please format your answer as an integer.)


\textbf{Solution:}
\boxed{8} minutes.


\textbf{Problem:}
Preamble: Give each of the following quantities to the nearest power of 10 and in the units requested. 

Subproblem 0: Age of our universe when most He nuclei were formed in minutes: 


Solution: \boxed{1} minute.

Final answer: The final answer is 1. I hope it is correct.

Subproblem 1: Age of our universe when hydrogen atoms formed in years:


\textbf{Solution:}
\boxed{400000} years.


\section{Introduction to Solid State Chemistry (3.091 Fall 2010)}

\textbf{Problem:}
Subproblem 0: Potassium metal can be used as the active surface in a photodiode because electrons are relatively easily removed from a potassium surface. The energy needed is $2.15 \times 10^{5} J$ per mole of electrons removed ( 1 mole $=6.02 \times 10^{23}$ electrons). What is the longest wavelength light (in nm) with quanta of sufficient energy to eject electrons from a potassium photodiode surface?


\textbf{Solution:}
\includegraphics[scale=0.5]{set_02_img_00.jpg}
\nonessentialimage
$I_{p}$, the photocurrent, is proportional to the intensity of incident radiation, i.e. the number of incident photons capable of generating a photoelectron.
This device should be called a phototube rather than a photodiode - a solar cell is a photodiode. 
Required: $1 eV=1.6 \times 10^{-19} J$
\[
E_{\text {rad }}=h v=(hc) / \lambda
\]
The question is: below what threshold energy (hv) will a photon no longer be able to generate a photoelectron?\\
$2.15 x 10^{5}$ J/mole photoelectrons $\times \frac{1 \text{mole}}{6.02 \times 10^{23} \text{photoelectrons}} = 3.57 \times 10^{-19}$ J/photoelectron\\
$\lambda_{\text {threshold }}=\frac{hc}{3.57 \times 10^{-19}}=\frac{6.62 \times 10^{-34} \times 3 \times 10^{8}}{3.57 \times 10^{-19}}=5.6 \times 10^{-7} m= \boxed{560} nm$


\textbf{Problem:}
Preamble: For red light of wavelength $(\lambda) 6.7102 \times 10^{-5} cm$, emitted by excited lithium atoms, calculate:

Subproblem 0: the frequency $(v)$ in Hz, to 4 decimal places. 


Solution: $c=\lambda v$ and $v=c / \lambda$ where $v$ is the frequency of radiation (number of waves/s).
For: $\quad \lambda=6.7102 \times 10^{-5} cm=6.7102 \times 10^{-7} m$
\[
v=\frac{2.9979 \times 10^{8} {ms}^{-1}}{6.7102 \times 10^{-7} m}=4.4677 \times 10^{14} {s}^{-1}= \boxed{4.4677} Hz
\]

Final answer: The final answer is 4.4677. I hope it is correct.

Subproblem 1: the wave number $(\bar{v})$ in ${cm}^{-1}$. Please format your answer as $n \times 10^x$, where $n$ is to 4 decimal places. 


Solution: $\bar{v}=\frac{1}{\lambda}=\frac{1}{6.7102 \times 10^{-7} m}=1.4903 \times 10^{6} m^{-1}= \boxed{1.4903e4} {cm}^{-1}$

Final answer: The final answer is 1.4903e4. I hope it is correct.

Subproblem 2: the wavelength $(\lambda)$ in nm, to 2 decimal places. 


\textbf{Solution:}
$\lambda=6.7102 \times 10^{-5} cm \times \frac{1 nm}{10^{-7} cm}= \boxed{671.02} cm$


\textbf{Problem:}
Subproblem 0: What is the net charge of arginine in a solution of $\mathrm{pH} \mathrm{} 1.0$ ? Please format your answer as +n or -n. 


\textbf{Solution:}
\boxed{+2}.


\textbf{Problem:}
Preamble: For red light of wavelength $(\lambda) 6.7102 \times 10^{-5} cm$, emitted by excited lithium atoms, calculate:

Subproblem 0: the frequency $(v)$ in Hz, to 4 decimal places. 


Solution: $c=\lambda v$ and $v=c / \lambda$ where $v$ is the frequency of radiation (number of waves/s).
For: $\quad \lambda=6.7102 \times 10^{-5} cm=6.7102 \times 10^{-7} m$
\[
v=\frac{2.9979 \times 10^{8} {ms}^{-1}}{6.7102 \times 10^{-7} m}=4.4677 \times 10^{14} {s}^{-1}= \boxed{4.4677} Hz
\]

Final answer: The final answer is 4.4677. I hope it is correct.

Subproblem 1: the wave number $(\bar{v})$ in ${cm}^{-1}$. Please format your answer as $n \times 10^x$, where $n$ is to 4 decimal places. 


\textbf{Solution:}
$\bar{v}=\frac{1}{\lambda}=\frac{1}{6.7102 \times 10^{-7} m}=1.4903 \times 10^{6} m^{-1}= \boxed{1.4903e4} {cm}^{-1}$


\textbf{Problem:}
Subproblem 0: Determine the atomic weight of ${He}^{++}$ in amu to 5 decimal places from the values of its constituents.


\textbf{Solution:}
The mass of the constituents $(2 p+2 n)$ is given as:
\[
\begin{array}{ll}
2 p= & 2 \times 1.6726485 \times 10^{-24} g \\
2 n= & 2 \times 16749543 \times 10^{-24} g
\end{array}
\]
The atomic weight (calculated) in amu is given as:
\[
\begin{aligned}
&\frac{6.6952056 \times 10^{-24} g}{1.660565 \times 10^{-24} g} / amu \\
&{He}=\boxed{4.03188} amu
\end{aligned}
\]


\textbf{Problem:}
Preamble: Determine the following values from a standard radio dial. 

Subproblem 0: What is the minimum wavelength in m for broadcasts on the AM band? Format your answer as an integer. 


Solution: \[
\mathrm{c}=v \lambda, \therefore \lambda_{\min }=\frac{\mathrm{c}}{v_{\max }} ; \lambda_{\max }=\frac{\mathrm{c}}{v_{\min }}
\]
$\lambda_{\min }=\frac{3 \times 10^{8} m / s}{1600 \times 10^{3} Hz}=\boxed{188} m$

Final answer: The final answer is 188. I hope it is correct.

Subproblem 1: What is the maximum wavelength in m for broadcasts on the AM band? Format your answer as an integer. 


\textbf{Solution:}
\[
\mathrm{c}=v \lambda, \therefore \lambda_{\min }=\frac{\mathrm{c}}{v_{\max }} ; \lambda_{\max }=\frac{\mathrm{c}}{v_{\min }}
\]
\[
\lambda_{\max }=\frac{3 \times 10^{8}}{530 \times 10^{3}}=\boxed{566} m
\]


\textbf{Problem:}
Subproblem 0: Determine the wavelength of radiation emitted by hydrogen atoms in angstroms upon electron transitions from $n=6$ to $n=2$.


\textbf{Solution:}
From the Rydberg relationship we obtain:
\[
\begin{aligned}
&\frac{1}{\lambda}=\bar{v}=R\left(\frac{1}{n_{i}^{2}}-\frac{1}{n_{f}^{2}}\right)=1.097 \times 10^{7}\left(\frac{1}{36}-\frac{1}{4}\right)=(-) 2.44 \times 10^{6} \\
&\lambda=\frac{1}{v}=\frac{1}{2.44 \times 10^{6}}=4.1 \times 10^{-7} {~m}=0.41 \mu {m}=\boxed{4100} \text{angstroms}
\end{aligned}
\]


\textbf{Problem:}
Preamble: Determine the following values from a standard radio dial. 

Subproblem 0: What is the minimum wavelength in m for broadcasts on the AM band? Format your answer as an integer. 


Solution: \[
\mathrm{c}=v \lambda, \therefore \lambda_{\min }=\frac{\mathrm{c}}{v_{\max }} ; \lambda_{\max }=\frac{\mathrm{c}}{v_{\min }}
\]
$\lambda_{\min }=\frac{3 \times 10^{8} m / s}{1600 \times 10^{3} Hz}=\boxed{188} m$

Final answer: The final answer is 188. I hope it is correct.

Subproblem 1: What is the maximum wavelength in m for broadcasts on the AM band? Format your answer as an integer. 


Solution: \[
\mathrm{c}=v \lambda, \therefore \lambda_{\min }=\frac{\mathrm{c}}{v_{\max }} ; \lambda_{\max }=\frac{\mathrm{c}}{v_{\min }}
\]
\[
\lambda_{\max }=\frac{3 \times 10^{8}}{530 \times 10^{3}}=\boxed{566} m
\]

Final answer: The final answer is 566. I hope it is correct.

Subproblem 2: What is the minimum wavelength in m (to 2 decimal places) for broadcasts on the FM band? 


\textbf{Solution:}
\[
\mathrm{c}=v \lambda, \therefore \lambda_{\min }=\frac{\mathrm{c}}{v_{\max }} ; \lambda_{\max }=\frac{\mathrm{c}}{v_{\min }}
\]
$\lambda_{\min }=\frac{3 \times 10^{8}}{108 \times 10^{6}}=\boxed{2.78} m$


\textbf{Problem:}
Subproblem 0: Calculate the "Bohr radius" in angstroms to 3 decimal places for ${He}^{+}$.


\textbf{Solution:}
In its most general form, the Bohr theory considers the attractive force (Coulombic) between the nucleus and an electron being given by:
\[
F_{c}=\frac{Z e^{2}}{4 \pi \varepsilon_{0} r^{2}}
\]
where Z is the charge of the nucleus ( 1 for H, 2 for He, etc.). Correspondingly, the electron energy $\left(E_{e l}\right)$ is given as:
\[
E_{e l}=-\frac{z^{2}}{n^{2}} \frac{m e^{4}}{8 h^{2} \varepsilon_{0}^{2}}
\]
and the electronic orbit $\left(r_{n}\right)$ :
\[
\begin{aligned}
&r_{n}=\frac{n^{2}}{Z} \frac{n^{2} \varepsilon_{0}}{\pi m e^{2}} \\
&r_{n}=\frac{n^{2}}{Z} a_{0}
\end{aligned}
\]
For ${He}^{+}(Z=2), {r}_{1}=\frac{1}{2} {a}_{0}=\frac{0.529}{2} \times 10^{-10} m=\boxed{0.264}$ angstroms


\textbf{Problem:}
Preamble: For red light of wavelength $(\lambda) 6.7102 \times 10^{-5} cm$, emitted by excited lithium atoms, calculate:

Subproblem 0: the frequency $(v)$ in Hz, to 4 decimal places. 


\textbf{Solution:}
$c=\lambda v$ and $v=c / \lambda$ where $v$ is the frequency of radiation (number of waves/s).
For: $\quad \lambda=6.7102 \times 10^{-5} cm=6.7102 \times 10^{-7} m$
\[
v=\frac{2.9979 \times 10^{8} {ms}^{-1}}{6.7102 \times 10^{-7} m}=4.4677 \times 10^{14} {s}^{-1}= \boxed{4.4677} Hz
\]


\textbf{Problem:}
Subproblem 0: Electromagnetic radiation of frequency $3.091 \times 10^{14} \mathrm{~Hz}$ illuminates a crystal of germanium (Ge). Calculate the wavelength of photoemission in meters generated by this interaction. Germanium is an elemental semiconductor with a band gap, $E_{g}$, of $0.7 \mathrm{eV}$. Please format your answer as $n \times 10^x$ where $n$ is to 2 decimal places.


\textbf{Solution:}
First compare $E$ of the incident photon with $E_{g}$ :
\[
\begin{aligned}
&\mathrm{E}_{\text {incident }}=\mathrm{hv}=6.6 \times 10^{-34} \times 3.091 \times 10^{14}=2.04 \times 10^{-19} \mathrm{~J} \\
&\mathrm{E}_{\mathrm{g}}=0.7 \mathrm{eV}=1.12 \times 10^{-19} \mathrm{~J}<\mathrm{E}_{\text {incident }}
\end{aligned}
\]
$\therefore$ electron promotion is followed by emission of a new photon of energy equal to $E_{g}$, and energy in excess of $E_{g}$ is dissipated as heat in the crystal
\includegraphics[scale=0.5]{set_17_img_00.jpg}
\nonessentialimage
$$
\lambda_{\text {emitted }}=\frac{\mathrm{hc}}{\mathrm{E}_{\mathrm{g}}}=\frac{6.6 \times 10^{-34} \times 3 \times 10^{8}}{0.7 \times 1.6 \times 10^{-19}}= \boxed{1.77e-6} \mathrm{~m}
$$


\textbf{Problem:}
Subproblem 0: What is the energy gap (in eV, to 1 decimal place) between the electronic states $n=3$ and $n=8$ in a hydrogen atom?


\textbf{Solution:}
\[
\begin{array}{rlr}
\text { Required: } & \Delta {E}_{{el}}=\left(\frac{1}{{n}_{{i}}^{2}}-\frac{1}{{n}_{{f}}^{2}}\right) {K} ; & {K}=2.18 \times 10^{-18} \\
& \text { Or } \bar{v}=\left(\frac{1}{{n}_{{i}}^{2}}-\frac{1}{{n}_{{f}}^{2}}\right) {R} ; & {R}=1.097 \times 10^{7} {~m}^{-1}
\end{array}
\]
(Since only the energy gap is asked, we are not concerned about the sign.)
\[
\begin{aligned}
&\Delta {E}=(1 / 9-1 / 65) {K}=0.0955 \times 2.18 \times 10^{-18} {~J} \\
&\Delta {E}=2.08 \times 10^{-19} {~J}=\boxed{1.3} {eV}
\end{aligned}
\]


\textbf{Problem:}
Subproblem 0: Determine for hydrogen the velocity in m/s of an electron in an ${n}=4$ state. Please format your answer as $n \times 10^x$ where $n$ is to 2 decimal places. 


\textbf{Solution:}
This problem may be solved in a variety of ways, the simplest of which makes use of the Bohr quantization of the angular momentum:
\[
\begin{aligned}
&m v r=n \times \frac{h}{2 \pi} \quad\left(r=r_{0} n^{2}\right) \\
&m v r_{0} n^{2}=n \times \frac{h}{2 \pi} \\
&v=\frac{h}{2 \pi m r_{0} n}= \boxed{5.47e5} m/s
\end{aligned}
\]
(A numerically correct result is obtained by taking:
\[
E_{e l}=-\frac{1}{n^{2}} K=\frac{m v^{2}}{2}
\]
The negative sign reflects the $E_{\text {pot }}$ term, which happens to be $-2 E_{K i n}$.)


\textbf{Problem:}
Preamble: A pure crystalline material (no impurities or dopants are present) appears red in transmitted light.

Subproblem 0: Is this material a conductor, semiconductor or insulator? Give the reasons for your answer.


Solution: If the material is pure (no impurity states present), then it must be classified as a \boxed{semiconductor} since it exhibits a finite "band gap" - i.e. to activate charge carriers, photons with energies in excess of "red" radiation are required.

Final answer: The final answer is semiconductor. I hope it is correct.

Subproblem 1: What is the approximate band gap $\left(\mathrm{E}_{g}\right)$ for this material in eV? Please round your answer to 1 decimal place.


\textbf{Solution:}
"White light" contains radiation in wavelength ranging from about $4000 \AA$ (violet) to $7000 \AA$ (deep red). A material appearing red in transmission has the following absorption characteristics:
\includegraphics[scale=0.5]{set_17_img_06.jpg}
\nonessentialimage
Taking $\lambda=6500 \AA$ as the optical absorption edge for this material, we have:
\[
E=\frac{\mathrm{hc}}{\lambda}=3.05 \times 10^{-29} \mathrm{~J} \times \frac{1 \mathrm{eV}}{1.6 \times 10^{-19} \mathrm{~J}}=1.9 \mathrm{eV}
\]
Accordingly, the band gap for the material is $E_{g}= \boxed{1.9} \mathrm{eV}$.


\textbf{Problem:}
Subproblem 0: Calculate the minimum potential $(V)$ in volts (to 1 decimal place) which must be applied to a free electron so that it has enough energy to excite, upon impact, the electron in a hydrogen atom from its ground state to a state of $n=5$.


\textbf{Solution:}
We can picture this problem more clearly: an electron is accelerated by a potential, $V x$, and thus acquires the kinetic energy e $x V_{x}\left[=\left(m v^{2}\right) / 2\right.$ which is to be exactly the energy required to excite an electron in hydrogen from $n=1$ to $n=5$.\\
${e} \cdot {V}_{{x}} =-{K}\left(\frac{1}{25}-\frac{1}{1}\right) $\\
${V}_{{x}} =\frac{{K}}{{e}} \times \frac{24}{25}=\frac{2.18 \times 10^{-18}}{1.6 \times 10^{-19}} \times \frac{24}{25}= \boxed{13.1} {Volt}$ \\
${\left[13.1 {eV}=13.1 {eV} \times \frac{1.6 \times 10^{-19} {~J}}{{eV}}=2.08 \times 10^{-18} {~J}=-{K}\left(\frac{1}{{n}_{{f}}^{2}}-\frac{1}{{n}_{{i}}^{2}}\right)\right]}$ 


\textbf{Problem:}
Preamble: For light with a wavelength $(\lambda)$ of $408 \mathrm{~nm}$ determine:

Subproblem 0: the frequency in $s^{-1}$. Please format your answer as $n \times 10^x$, where $n$ is to 3 decimal places. 


Solution: To solve this problem we must know the following relationships:
\[
\begin{aligned}
v \lambda &=c
\end{aligned}
\]
$v$ (frequency) $=\frac{c}{\lambda}=\frac{3 \times 10^{8} m / s}{408 \times 10^{-9} m}= \boxed{7.353e14} s^{-1}$

Final answer: The final answer is 7.353e14. I hope it is correct.

Subproblem 1: the wave number in $m^{-1}$. Please format your answer as $n \times 10^x$, where $n$ is to 2 decimal places.


Solution: To solve this problem we must know the following relationships:
\[
\begin{aligned}
1 / \lambda=\bar{v} 
\end{aligned}
\]
$\bar{v}$ (wavenumber) $=\frac{1}{\lambda}=\frac{1}{408 \times 10^{-9} m}=\boxed{2.45e6} m^{-1}$

Final answer: The final answer is 2.45e6. I hope it is correct.

Subproblem 2: the wavelength in angstroms. 


\textbf{Solution:}
To solve this problem we must know the following relationships:
\[
\begin{aligned}
m =10^{10} angstrom
\end{aligned}
\]
$\lambda=408 \times 10^{-9} m \times \frac{10^{10} angstrom}{\mathrm{m}}=\boxed{4080} angstrom$


\textbf{Problem:}
Preamble: Reference the information below to solve the following problems. 
$\begin{array}{llll}\text { Element } & \text { Ionization Potential }  & \text { Element } & \text { Ionization Potential } \\ {Na} & 5.14 & {Ca} & 6.11 \\ {Mg} & 7.64 & {Sc} & 6.54 \\ {Al} & 5.98 & {Ti} & 6.82 \\ {Si} & 8.15 & {~V} & 6.74 \\ {P} & 10.48 & {Cr} & 6.76 \\ {~S} & 10.36 & {Mn} & 7.43 \\ {Cl} & 13.01 & {Fe} & 7.9 \\ {Ar} & 15.75 & {Co} & 7.86 \\ & & {Ni} & 7.63 \\ & & {Cu} & 7.72\end{array}$

Subproblem 0: What is the first ionization energy (in J, to 3 decimal places) for Na?


Solution: The required data can be obtained by multiplying the ionization potentials (listed in the Periodic Table) with the electronic charge ( ${e}^{-}=1.6 \times 10^{-19}$ C).
\boxed{0.822} J.

Final answer: The final answer is 0.822. I hope it is correct.

Subproblem 1: What is the first ionization energy (in J, to 2 decimal places) for Mg?


\textbf{Solution:}
The required data can be obtained by multiplying the ionization potentials (listed in the Periodic Table) with the electronic charge ( ${e}^{-}=1.6 \times 10^{-19}$ C).
\boxed{1.22} J.


\textbf{Problem:}
Subproblem 0: Light of wavelength $\lambda=4.28 \times 10^{-7} {~m}$ interacts with a "motionless" hydrogen atom. During this interaction it transfers all its energy to the orbiting electron of the hydrogen. What is the velocity in m/s of this electron after interaction? Please format your answer as $n \times 10^x$ where $n$ is to 2 decimal places.


\textbf{Solution:}
First of all, a sketch:
\includegraphics[scale=0.5]{set_03_img_00.jpg}
\nonessentialimage
\[
\begin{aligned}
&\text { possibly to } {n}=\infty \text { (ionization), } \\
&\text { depending on the magnitude of } E(h v)
\end{aligned}
\]
let us see: $E(h v)=(h c) / \lambda=4.6 \times 10^{-19} {~J}$
To move the electron from $n=1$ to $n=2$ (minimum energy required for absorption of the photon), we have:
\[
\begin{aligned}
\Delta {E}=\left(\frac{1}{{n}_{{i}}^{2}}-\frac{1}{{n}_{{f}}^{2}}\right) {K} &=\frac{3}{4} {~K} \\
&=\frac{3}{4} \times 2.18 \times 10^{-18} {~J}=1.6 \times 10^{-18} {~J}
\end{aligned}
\]
We recognize that the photon energy is less than the $\Delta E_{\min }$ (for $n=1 \rightarrow n=2$ ).
This means that no interaction can take place - the photon will "pass by" and the electron will continue to orbit in its $1 s$ state! Its orbiting velocity can be obtained from:
\[
\begin{aligned}
&m v r=n\left(\frac{h}{2 \pi}\right) \\
&v=n\left(\frac{h}{2 \pi m r}\right)= \boxed{2.19e6} {~m} / {s}
\end{aligned}
\]


\textbf{Problem:}
Subproblem 0: Determine the minimum potential in V (to 2 decimal places) that must be applied to an $\alpha$-particle so that on interaction with a hydrogen atom, a ground state electron will be excited to $n$ $=6$.


\textbf{Solution:}
\[
\Delta {E}_{1 \rightarrow 6}={qV} \quad \therefore {V}=\frac{\Delta {E}_{1 \rightarrow 6}}{{q}}
\]
\[
\begin{aligned}
& \Delta {E}_{1 \rightarrow 6}=-{K}\left(\frac{1}{1^{2}}-\frac{1}{6^{2}}\right)=\frac{35}{36} {K} \\
& {q}=+2 {e} \\
& \therefore \quad V=\frac{35}{36} \times \frac{2.18 \times 10^{18}}{2 \times 1.6 \times 10^{-19}}=\boxed{6.62} V 
\end{aligned}
\]


\textbf{Problem:}
Preamble: Reference the information below to solve the following problems. 
$\begin{array}{llll}\text { Element } & \text { Ionization Potential }  & \text { Element } & \text { Ionization Potential } \\ {Na} & 5.14 & {Ca} & 6.11 \\ {Mg} & 7.64 & {Sc} & 6.54 \\ {Al} & 5.98 & {Ti} & 6.82 \\ {Si} & 8.15 & {~V} & 6.74 \\ {P} & 10.48 & {Cr} & 6.76 \\ {~S} & 10.36 & {Mn} & 7.43 \\ {Cl} & 13.01 & {Fe} & 7.9 \\ {Ar} & 15.75 & {Co} & 7.86 \\ & & {Ni} & 7.63 \\ & & {Cu} & 7.72\end{array}$

Subproblem 0: What is the first ionization energy (in J, to 3 decimal places) for Na?


\textbf{Solution:}
The required data can be obtained by multiplying the ionization potentials (listed in the Periodic Table) with the electronic charge ( ${e}^{-}=1.6 \times 10^{-19}$ C).
\boxed{0.822} J.


\textbf{Problem:}
Preamble: For "yellow radiation" (frequency, $v,=5.09 \times 10^{14} s^{-1}$ ) emitted by activated sodium, determine:

Subproblem 0: the wavelength $(\lambda)$ in m. Please format your answer as $n \times 10^x$, where n is to 2 decimal places.


Solution: The equation relating $v$ and $\lambda$ is $c=v \lambda$ where $c$ is the speed of light $=3.00 \times 10^{8} \mathrm{~m}$.
\[
\lambda=\frac{c}{v}=\frac{3.00 \times 10^{8} m / s}{5.09 \times 10^{14} s^{-1}}=\boxed{5.89e-7} m
\]

Final answer: The final answer is 5.89e-7. I hope it is correct.

Subproblem 1: the wave number $(\bar{v})$ in ${cm}^{-1}$. Please format your answer as $n \times 10^x$, where n is to 2 decimal places.


\textbf{Solution:}
The wave number is $1 /$ wavelength, but since the wavelength is in m, and the wave number should be in ${cm}^{-1}$, we first change the wavelength into cm :
\[
\lambda=5.89 \times 10^{-7} m \times 100 cm / m=5.89 \times 10^{-5} cm
\]
Now we take the reciprocal of the wavelength to obtain the wave number:
\[
\bar{v}=\frac{1}{\lambda}=\frac{1}{5.89 \times 10^{-5} cm}= \boxed{1.70e4} {cm}^{-1}
\]


\textbf{Problem:}
Subproblem 0: In the balanced equation for the reaction between $\mathrm{CO}$ and $\mathrm{O}_{2}$ to form $\mathrm{CO}_{2}$, what is the coefficient of $\mathrm{CO}$?


Solution: \boxed{1}.

Final answer: The final answer is 1. I hope it is correct.

Subproblem 1: In the balanced equation for the reaction between $\mathrm{CO}$ and $\mathrm{O}_{2}$ to form $\mathrm{CO}_{2}$, what is the coefficient of $\mathrm{O}_{2}$ (in decimal form)?


\textbf{Solution:}
\boxed{0.5}. 


\textbf{Problem:}
Preamble: Calculate the molecular weight in g/mole (to 2 decimal places) of each of the substances listed below.

Subproblem 0: $\mathrm{NH}_{4} \mathrm{OH}$


\textbf{Solution:}
$\mathrm{NH}_{4} \mathrm{OH}$ :
$5 \times 1.01=5.05(\mathrm{H})$
$1 \times 14.01=14.01(\mathrm{~N})$
$1 \times 16.00=16.00(\mathrm{O})$
$\mathrm{NH}_{4} \mathrm{OH}= \boxed{35.06}$ g/mole


\textbf{Problem:}
Subproblem 0: In the balanced equation for the reaction between $\mathrm{CO}$ and $\mathrm{O}_{2}$ to form $\mathrm{CO}_{2}$, what is the coefficient of $\mathrm{CO}$?


Solution: \boxed{1}.

Final answer: The final answer is 1. I hope it is correct.

Subproblem 1: In the balanced equation for the reaction between $\mathrm{CO}$ and $\mathrm{O}_{2}$ to form $\mathrm{CO}_{2}$, what is the coefficient of $\mathrm{O}_{2}$ (in decimal form)?


Solution: \boxed{0.5}. 

Final answer: The final answer is 0.5. I hope it is correct.

Subproblem 2: In the balanced equation for the reaction between $\mathrm{CO}$ and $\mathrm{O}_{2}$ to form $\mathrm{CO}_{2}$, what is the coefficient of $\mathrm{CO}_{2}$ (in decimal form)?


\textbf{Solution:}
\boxed{1}.


\textbf{Problem:}
Subproblem 0: Magnesium (Mg) has the following isotopic distribution:
\[
\begin{array}{ll}
24_{\mathrm{Mg}} & 23.985 \mathrm{amu} \text { at } 0.7870 \text { fractional abundance } \\
25_{\mathrm{Mg}} & 24.986 \mathrm{amu} \text { at } 0.1013 \text { fractional abundance } \\
26_{\mathrm{Mg}} & 25.983 \mathrm{amu} \text { at } 0.1117 \text { fractional abundance }
\end{array}
\]
What is the atomic weight of magnesium (Mg) (to 3 decimal places) according to these data?


\textbf{Solution:}
The atomic weight is the arithmetic average of the atomic weights of the isotopes, taking into account the fractional abundance of each isotope.
\[
\text { At.Wt. }=\frac{23.985 \times 0.7870+24.986 \times 0.1013+25.983 \times 0.1117}{0.7870+0.1013+0.1117}=\boxed{24.310}
\]


\textbf{Problem:}
Preamble: Electrons are accelerated by a potential of 10 Volts.

Subproblem 0: Determine their velocity in m/s. Please format your answer as $n \times 10^x$, where $n$ is to 2 decimal places. 


\textbf{Solution:}
The definition of an ${eV}$ is the energy gained by an electron when it is accelerated through a potential of $1 {~V}$, so an electron accelerated by a potential of $10 {~V}$ would have an energy of $10 {eV}$.\\
${E}=\frac{1}{2} m {v}^{2} \rightarrow {v}=\sqrt{2 {E} / {m}}$
\[
E=10 {eV}=1.60 \times 10^{-18} {~J}
\]
\[
\begin{aligned}
& {m}=\text { mass of electron }=9.11 \times 10^{-31} {~kg} \\
& v=\sqrt{\frac{2 \times 1.6 \times 10^{-18} {~J}}{9.11 \times 10^{-31} {~kg}}}= \boxed{1.87e6} {~m} / {s} 
\end{aligned}
\]


\textbf{Problem:}
Subproblem 0: Determine the frequency (in $s^{-1}$ of radiation capable of generating, in atomic hydrogen, free electrons which have a velocity of $1.3 \times 10^{6} {~ms}^{-1}$. Please format your answer as $n \times 10^x$ where $n$ is to 2 decimal places. 


\textbf{Solution:}
Remember the ground state electron energy in hydrogen $\left({K}=-2.18 \times 10^{-18} {~J}\right)$. The radiation in question will impart to the removed electron a velocity of $1.3 {x}$ $10^{6} {~ms}^{-1}$, which corresponds to:
\[
\begin{aligned}
&E_{\text {Kin }}=\frac{m v^{2}}{2}=\frac{9.1 \times 10^{-31} \times\left(1.3 \times 10^{6}\right)^{2}}{2} \text { Joules }=7.69 \times 10^{-19} {~J} \\
&E_{\text {rad }}=E_{\text {Kin }}+E_{\text {ioniz }}=7.69 \times 10^{-19}+2.18 \times 10^{-18}=2.95 \times 10^{-18} {~J} \\
&E_{\text {rad }}=h_{v} ; \quad v=\frac{E}{h}=\frac{2.95 \times 10^{-18}}{6.63 \times 10^{-34}}= \boxed{4.45e15} {~s}^{-1}
\end{aligned}
\]


\textbf{Problem:}
Subproblem 0: In the balanced equation for the reaction between $\mathrm{CO}$ and $\mathrm{O}_{2}$ to form $\mathrm{CO}_{2}$, what is the coefficient of $\mathrm{CO}$?


\textbf{Solution:}
\boxed{1}.


\textbf{Problem:}
Preamble: Electrons are accelerated by a potential of 10 Volts.

Subproblem 0: Determine their velocity in m/s. Please format your answer as $n \times 10^x$, where $n$ is to 2 decimal places. 


Solution: The definition of an ${eV}$ is the energy gained by an electron when it is accelerated through a potential of $1 {~V}$, so an electron accelerated by a potential of $10 {~V}$ would have an energy of $10 {eV}$.\\
${E}=\frac{1}{2} m {v}^{2} \rightarrow {v}=\sqrt{2 {E} / {m}}$
\[
E=10 {eV}=1.60 \times 10^{-18} {~J}
\]
\[
\begin{aligned}
& {m}=\text { mass of electron }=9.11 \times 10^{-31} {~kg} \\
& v=\sqrt{\frac{2 \times 1.6 \times 10^{-18} {~J}}{9.11 \times 10^{-31} {~kg}}}= \boxed{1.87e6} {~m} / {s} 
\end{aligned}
\]

Final answer: The final answer is 1.87e6. I hope it is correct.

Subproblem 1: Determine their deBroglie wavelength $\left(\lambda_{p}\right)$ in m. Please format your answer as $n \times 10^x$, where $n$ is to 2 decimal places. 


\textbf{Solution:}
$\lambda_{p}=h / m v$
\[
\lambda_{p}=\frac{6.63 \times 10^{-34}}{9.11 \times 10^{-34} {~kg} \times 1.87 \times 10^{6} {~m} / {s}}= \boxed{3.89e-10} {~m}
\]


\textbf{Problem:}
Preamble: In all likelihood, the Soviet Union and the United States together in the past exploded about ten hydrogen devices underground per year.

Subproblem 0: If each explosion converted about $10 \mathrm{~g}$ of matter into an equivalent amount of energy (a conservative estimate), how many $k J$ of energy were released per device? Please format your answer as $n \times 10^{x}$. 


\textbf{Solution:}
$\Delta \mathrm{E}=\Delta \mathrm{mc}^{2}=10 \mathrm{~g} \times \frac{1 \mathrm{~kg}}{1000 \mathrm{~g}} \times\left(3 \times 10^{8} \mathrm{~ms}^{-1}\right)^{2}$ $=9 \times 10^{14} \mathrm{~kg} \mathrm{~m}^{2} \mathrm{~s}^{-2}=9 \times 10^{14} \mathrm{~J}= \boxed{9e11} \mathrm{~kJ} /$ bomb.


\textbf{Problem:}
Preamble: Calculate the molecular weight in g/mole (to 2 decimal places) of each of the substances listed below.

Subproblem 0: $\mathrm{NH}_{4} \mathrm{OH}$


Solution: $\mathrm{NH}_{4} \mathrm{OH}$ :
$5 \times 1.01=5.05(\mathrm{H})$
$1 \times 14.01=14.01(\mathrm{~N})$
$1 \times 16.00=16.00(\mathrm{O})$
$\mathrm{NH}_{4} \mathrm{OH}= \boxed{35.06}$ g/mole

Final answer: The final answer is 35.06. I hope it is correct.

Subproblem 1: $\mathrm{NaHCO}_{3}$


Solution: $\mathrm{NaHCO}_{3}: 3 \times 16.00=48.00(\mathrm{O})$
$1 \times 22.99=22.99(\mathrm{Na})$
$1 \times 1.01=1.01$ (H)
$1 \times 12.01=12.01$ (C)
$\mathrm{NaHCO}_{3}= \boxed{84.01}$ g/mole

Final answer: The final answer is 84.01. I hope it is correct.

Subproblem 2: $\mathrm{CH}_{3} \mathrm{CH}_{2} \mathrm{OH}$


\textbf{Solution:}
$\mathrm{CH}_{3} \mathrm{CH}_{2} \mathrm{OH}: \quad 2 \times 12.01=24.02$ (C)
$6 \times 1.01=6.06(\mathrm{H})$
$1 \times 16.00=16.00(\mathrm{O})$
$\mathrm{CH}_{3} \mathrm{CH}_{2} \mathrm{OH}: \boxed{46.08}$ g/mole


\textbf{Problem:}
Subproblem 0: In the balanced equation for the reaction between $\mathrm{CO}$ and $\mathrm{O}_{2}$ to form $\mathrm{CO}_{2}$, what is the coefficient of $\mathrm{CO}$?


Solution: \boxed{1}.

Final answer: The final answer is 1. I hope it is correct.

Subproblem 1: In the balanced equation for the reaction between $\mathrm{CO}$ and $\mathrm{O}_{2}$ to form $\mathrm{CO}_{2}$, what is the coefficient of $\mathrm{O}_{2}$ (in decimal form)?


Solution: \boxed{0.5}. 

Final answer: The final answer is 0.5. I hope it is correct.

Subproblem 2: In the balanced equation for the reaction between $\mathrm{CO}$ and $\mathrm{O}_{2}$ to form $\mathrm{CO}_{2}$, what is the coefficient of $\mathrm{CO}_{2}$ (in decimal form)?


Solution: \boxed{1}.

Final answer: The final answer is 1. I hope it is correct.

Subproblem 3: If $32.0 \mathrm{~g}$ of oxygen react with $\mathrm{CO}$ to form carbon dioxide $\left(\mathrm{CO}_{2}\right)$, how much CO was consumed in this reaction (to 1 decimal place)?


\textbf{Solution:}
Molecular Weight (M.W.) of (M.W.) of $\mathrm{O}_{2}: 32.0$
(M.W.) of CO: $28.0$
available oxygen: $32.0 \mathrm{~g}=1$ mole, correspondingly the reaction involves 2 moles of CO [see (a)]:
\[
\mathrm{O}_{2}+2 \mathrm{CO} \rightarrow 2 \mathrm{CO}_{2}
\]
mass of CO reacted $=2$ moles $\times 28 \mathrm{~g} /$ mole $=\boxed{56.0} g$


\textbf{Problem:}
Preamble: For "yellow radiation" (frequency, $v,=5.09 \times 10^{14} s^{-1}$ ) emitted by activated sodium, determine:

Subproblem 0: the wavelength $(\lambda)$ in m. Please format your answer as $n \times 10^x$, where n is to 2 decimal places.


\textbf{Solution:}
The equation relating $v$ and $\lambda$ is $c=v \lambda$ where $c$ is the speed of light $=3.00 \times 10^{8} \mathrm{~m}$.
\[
\lambda=\frac{c}{v}=\frac{3.00 \times 10^{8} m / s}{5.09 \times 10^{14} s^{-1}}=\boxed{5.89e-7} m
\]


\textbf{Problem:}
Subproblem 0: For a proton which has been subjected to an accelerating potential (V) of 15 Volts, determine its deBroglie wavelength in m. Please format your answer as $n \times 10^x$, where $n$ is to 1 decimal place. 


\textbf{Solution:}
\[
\begin{gathered}
E_{{K}}={eV}=\frac{{m}_{{p}} {v}^{2}}{2} ; \quad {v}_{{p}}=\sqrt{\frac{2 {eV}}{{m}_{{p}}}} \\
\lambda_{{p}}=\frac{{h}}{{m}_{{p}} {v}}=\frac{{h}}{{m}_{{p}} \sqrt{\frac{2 {eV}}{{m}_{{p}}}}}=\frac{{h}}{\sqrt{2 {eVm_{p }}}}=\frac{6.63 \times 10^{-34}}{\left(2 \times 1.6 \times 10^{-19} \times 15 \times 1.67 \times 10^{-27}\right)^{\frac{1}{2}}}
\\
= \boxed{7.4e-12} {~m}
\end{gathered}
\]


\textbf{Problem:}
Preamble: For light with a wavelength $(\lambda)$ of $408 \mathrm{~nm}$ determine:

Subproblem 0: the frequency in $s^{-1}$. Please format your answer as $n \times 10^x$, where $n$ is to 3 decimal places. 


\textbf{Solution:}
To solve this problem we must know the following relationships:
\[
\begin{aligned}
v \lambda &=c
\end{aligned}
\]
$v$ (frequency) $=\frac{c}{\lambda}=\frac{3 \times 10^{8} m / s}{408 \times 10^{-9} m}= \boxed{7.353e14} s^{-1}$


\textbf{Problem:}
Subproblem 0: Determine in units of eV (to 2 decimal places) the energy of a photon ( $h v)$ with the wavelength of $800$ nm.


\textbf{Solution:}
\[
\begin{aligned}
E_{(\mathrm{eV})}=\frac{\mathrm{hc}}{\lambda} \times \frac{\mathrm{leV}}{1.6 \times 10^{-19} \mathrm{~J}} &=\frac{6.63 \times 10^{-34}[\mathrm{~s}] \times 3 \times 10^{8}\left[\frac{\mathrm{m}}{\mathrm{s}}\right]}{8.00 \times 10^{-7} \mathrm{~m}} \times \frac{\mathrm{leV}}{1.6 \times 10^{-19} \mathrm{~J}} \\
=\boxed{1.55} eV
\end{aligned}
\]


\textbf{Problem:}
Subproblem 0: Determine for barium (Ba) the linear density of atoms along the $<110>$ directions, in atoms/m.


\textbf{Solution:}
Determine the lattice parameter and look at the unit cell occupation.
\includegraphics[scale=0.5]{set_23_img_02.jpg}
\nonessentialimage
Ba: $\quad$ BCC; atomic volume $=39.24 \mathrm{~cm}^{3} / \mathrm{mole} ; \mathrm{n}=2 \mathrm{atoms} /$ unit cell\\
$$
3.924 \times 10^{-5}\left(\mathrm{~m}^{3} / \text { mole }\right)=\frac{\mathrm{N}_{\mathrm{A}}}{2} \mathrm{a}^{3}
$$
$$
a=\sqrt[3]{\frac{2 \times 3.924 \times 10^{-5}}{6.02 \times 10^{23}}}=5.08 \times 10^{-10} \mathrm{~m}
$$
$$
\text { linear density }=\frac{1 \text { atom }}{a \sqrt{2}}=\frac{1}{5.08 \times 10^{-10} \times \sqrt{2}} = \boxed{1.39e9}
$$ atoms/m


\textbf{Problem:}
Subproblem 0: A photon with a wavelength $(\lambda)$ of $3.091 \times 10^{-7} {~m}$ strikes an atom of hydrogen. Determine the velocity in m/s of an electron ejected from the excited state, $n=3$. Please format your answer as $n \times 10^x$ where $n$ is to 2 decimal places. 


\textbf{Solution:}
\[
\begin{aligned}
&E_{\text {incident photon }}=E_{\text {binding }}+E_{\text {scattered } e^{-}} \\
&E_{\text {binding }}=-K\left(\frac{1}{3^{2}}\right) \quad \therefore \frac{hc}{\lambda}=\frac{K}{9}+\frac{1}{2} {mv^{2 }} \quad \therefore\left[\left(\frac{{hc}}{\lambda}-\frac{{K}}{9}\right) \frac{2}{{m}}\right]^{\frac{1}{2}}={v} \\
&{E}_{\text {incident photon }}=\frac{{hc}}{\lambda}=\frac{1}{2} {mv}^{2} \\
&{\left[\left(\frac{6.6 \times 10^{-34} \times 3 \times 10^{8}}{3.091 \times 10^{-7}}-\frac{2.18 \times 10^{-18}}{9}\right) \frac{2}{9.11 \times 10^{-31}}\right]^{\frac{1}{2}}={v}} \\
&\therefore {v}= \boxed{9.35e5} {m} / {s}
\end{aligned}
\]


\textbf{Problem:}
Preamble: For the element copper (Cu) determine:

Subproblem 0: the distance of second nearest neighbors (in meters). Please format your answer as $n \times 10^x$ where $n$ is to 2 decimal places.


\textbf{Solution:}
The answer can be found by looking at a unit cell of $\mathrm{Cu}$ (FCC).
\includegraphics[scale=0.5]{set_23_img_00.jpg}
\nonessentialimage
Nearest neighbor distance is observed along $<110>$; second-nearest along $<100>$. The second-nearest neighbor distance is found to be "a".
Cu: atomic volume $=7.1 \times 10^{-6} \mathrm{~m}^{3} /$ mole $=\frac{\mathrm{N}_{\mathrm{A}}}{4} \mathrm{a}^{3}$ ( $\mathrm{Cu}: \mathrm{FCC} ; 4$ atoms/unit cell) $a=\sqrt[3]{\frac{7.1 \times 10^{-6} \times 4}{6.02 \times 10^{23}}}= \boxed{3.61e-10} \mathrm{~m}$


\textbf{Problem:}
Subproblem 0: A line of the Lyman series of the spectrum of hydrogen has a wavelength of $9.50 \times 10^{-8} {~m}$. What was the "upper" quantum state $\left({n}_{{i}}\right)$ involved in the associated electron transition?


\textbf{Solution:}
The Lyman series in hydrogen spectra comprises all electron transitions terminating in the ground state $({n}=1)$. In the present problem it is convenient to convert $\lambda$ into $\bar{v}$ and to use the Rydberg equation. Since we have an "emission spectrum", the sign will be negative in the conventional approach. We can avoid the sign problem, however:
\[
\begin{aligned}
& \bar{v}=R\left(\frac{1}{n_{f}^{2}}-\frac{1}{n_{i}^{2}}\right)=R\left(1-\frac{1}{n_{i}^{2}}\right) \\
& \overline{\frac{v}{R}}=\left(1-\frac{1}{n_{i}^{2}}\right) \\
& \frac{1}{n_{i}^{2}}=1-\frac{\bar{v}}{R}=\frac{R-\bar{v}}{R} \\
& n_{i}^{2}=\frac{R}{R-\bar{v}} \\
& {n}_{{i}}^{2}=\sqrt{\frac{{R}}{{R}-\bar{v}}} \quad \bar{v}=\frac{1}{9.5 \times 10^{-8} {~m}}=1.053 \times 10^{7} {~m}^{-1} \\
& n_{i}=\sqrt{\frac{1.097 \times 10^{7}}{1.097 \times 10^{7}-1.053 \times 10^{7}}}= \boxed{5}
\end{aligned}
\]


\textbf{Problem:}
Subproblem 0: Determine the diffusivity $\mathrm{D}$ of lithium ( $\mathrm{Li}$ ) in silicon (Si) at $1200^{\circ} \mathrm{C}$, knowing that $D_{1100^{\circ} \mathrm{C}}=10^{-5} \mathrm{~cm}^{2} / \mathrm{s}$ and $\mathrm{D}_{695^{\circ} \mathrm{C}}=10^{-6} \mathrm{~cm}^{2} / \mathrm{s}$. Please format your answer as $n \times 10^x$ where $n$ is to 2 decimal places, in $\mathrm{~cm}^2/\mathrm{sec}$.


\textbf{Solution:}
\[
\begin{aligned}
&\frac{D_{1}}{D_{2}}=\frac{10^{-6}}{10^{-5}}=10^{-1}=e^{-\frac{E_{A}}{R}\left(\frac{1}{968}-\frac{1}{1373}\right)} \\
&E_{A}=\frac{R \ln 10}{\frac{1}{968}-\frac{1}{1373}}=62.8 \mathrm{~kJ} / \mathrm{mole} \\
&\frac{D_{1100}}{D_{1200}}=e^{-\frac{E_{A}}{R}\left(\frac{1}{1373}-\frac{1}{1473}\right)} \\
&D_{1200}=10^{-5} \times e^{\frac{E_{A}}{R}\left(\frac{1}{1373}-\frac{1}{1473}\right)}= \boxed{1.45e-5} \mathrm{~cm}^{2} / \mathrm{sec}
\end{aligned}
\]


\textbf{Problem:}
Subproblem 0: By planar diffusion of antimony (Sb) into p-type germanium (Ge), a p-n junction is obtained at a depth of $3 \times 10^{-3} \mathrm{~cm}$ below the surface. What is the donor concentration in the bulk germanium if diffusion is carried out for three hours at $790^{\circ} \mathrm{C}$? Please format your answer as $n \times 10^x$ where $n$ is to 2 decimal places, and express it in units of $1/\mathrm{cm}^3$. The surface concentration of antimony is held constant at a value of $8 \times 10^{18}$ $\mathrm{cm}^{-3} ; D_{790^{\circ} \mathrm{C}}=4.8 \times 10^{-11} \mathrm{~cm}^{2} / \mathrm{s}$.


\textbf{Solution:}
\includegraphics[scale=0.5]{set_37_img_00.jpg}
\nonessentialimage
\[
\begin{aligned}
&\frac{c}{c_{s}}=\operatorname{erfc} \frac{x}{2 \sqrt{D t}}=\operatorname{erfc} \frac{3 \times 10^{-3}}{2 \sqrt{D t}}=\operatorname{erfc}(2.083) \\
&\frac{c}{c_{s}}=1-\operatorname{erf}(2.083), \therefore 1-\frac{c}{c_{s}}=0.9964 \\
&\frac{c}{c_{s}}=3.6 \times 10^{-3}, \therefore c=2.88 \times 10^{16} \mathrm{~cm}^{-3}
\end{aligned}
\]
The donor concentration in germanium is $\boxed{2.88e16} / \mathrm{cm}^{3}$.


\textbf{Problem:}
Preamble: One mole of electromagnetic radiation (light, consisting of energy packages called photons) has an energy of $171 \mathrm{~kJ} /$ mole photons.

Subproblem 0: Determine the wavelength of this light in nm. 


\textbf{Solution:}
We know: $E_{\text {photon }}=h v=h c / \lambda$ to determine the wavelength associated with a photon we need to know its energy. $E=\frac{171 \mathrm{~kJ}}{\text { mole }}=\frac{1.71 \times 10^{5} \mathrm{~J}}{\text { mole }} \times \frac{1 \text { mole }}{6.02 \times 10^{23} \text { photons }}$
\[
=\frac{2.84 \times 10^{-19} \mathrm{~J}}{\text { photon }} ; \quad \mathrm{E}_{\text {photon }}=2.84 \times 10^{-19} \mathrm{~J}=\mathrm{h}_{v}=\frac{\mathrm{hc}}{\lambda}
\]
\[
\begin{aligned}
& \lambda=\frac{h c}{E_{\text {photon }}}=\frac{6.63 \times 10^{-34} \mathrm{Js} \times 3 \times 10^{8} \frac{\mathrm{m}}{\mathrm{s}}}{2.84 \times 10^{-19} \mathrm{~J}}=7.00 \times 10^{-7} \mathrm{~m} \\
& =\boxed{700} nm
\end{aligned}
\]


\textbf{Problem:}
Preamble: Two lasers generate radiation of (1) $9.5 \mu {m}$ and (2) $0.1 \mu {m}$ respectively.

Subproblem 0: Determine the photon energy (in eV, to two decimal places) of the laser generating radiation of $9.5 \mu {m}$.


\textbf{Solution:}
\[
\begin{aligned}
{E} &={h} v=\frac{{hc}}{\lambda} {J} \times \frac{1 {eV}}{1.6 \times 10^{-19} {~J}} \\
{E}_{1} &=\frac{{hc}}{9.5 \times 10^{-6}} \times \frac{1}{1.6 \times 10^{-19}} {eV}= \boxed{0.13} {eV}
\end{aligned}
\]


\textbf{Problem:}
Subproblem 0: At $100^{\circ} \mathrm{C}$ copper $(\mathrm{Cu})$ has a lattice constant of $3.655 \AA$. What is its density in $g/cm^3$ at this temperature? Please round your answer to 2 decimal places.


\textbf{Solution:}
$\mathrm{Cu}$ is FCC, so $\mathrm{n}=4$
\[
\begin{aligned}
&\mathrm{a}=3.655 \AA=3.655 \times 10^{-10} \mathrm{~m} \\
&\text { atomic weight }=63.55 \mathrm{~g} / \mathrm{mole} \\
&\frac{\text { atomic weight }}{\rho} \times 10^{-6}=\frac{N_{\mathrm{A}}}{\mathrm{n}} \times \mathrm{a}^{3} \\
&\rho=\frac{(63.55 \mathrm{~g} / \mathrm{mole})(4 \text { atoms } / \text { unit cell })}{\left(6.023 \times 10^{23} \text { atoms } / \mathrm{mole}\right)\left(3.655 \times 10^{-10} \mathrm{~m}^{3}\right)}= \boxed{8.64} \mathrm{~g} / \mathrm{cm}^{3}
\end{aligned}
\]


\textbf{Problem:}
Subproblem 0: Determine the atomic (metallic) radius of Mo in meters. Do not give the value listed in the periodic table; calculate it from the fact that Mo's atomic weight is $=95.94 \mathrm{~g} /$ mole and $\rho=10.2 \mathrm{~g} / \mathrm{cm}^{3}$. Please format your answer as $n \times 10^x$ where $n$ is to 2 decimal places.


\textbf{Solution:}
Mo: atomic weight $=95.94 \mathrm{~g} /$ mole
\[
\rho=10.2 \mathrm{~g} / \mathrm{cm}^{3}
\]
BCC, so $n=2$ atoms/unit cell
\[
\begin{aligned}
&\mathrm{a}^{3}=\frac{(95.94 \mathrm{~g} / \mathrm{mole})(2 \text { atoms/unit cell })}{\left(10.2 \mathrm{~g} / \mathrm{cm}^{3}\right)\left(6.023 \times 10^{23} \text { atoms } / \mathrm{mole}\right)} \times 10^{-6} \frac{\mathrm{m}^{3}}{\mathrm{~cm}^{3}} \\
&=3.12 \times 10^{-29} \mathrm{~m}^{3} \\
&a=3.22 \times 10^{-10} \mathrm{~m}
\end{aligned}
\]
For BCC, $a \sqrt{3}=4 r$, so $r= \boxed{1.39e-10} \mathrm{~m}$


\textbf{Problem:}
Preamble: Determine the following values from a standard radio dial. 

Subproblem 0: What is the minimum wavelength in m for broadcasts on the AM band? Format your answer as an integer. 


\textbf{Solution:}
\[
\mathrm{c}=v \lambda, \therefore \lambda_{\min }=\frac{\mathrm{c}}{v_{\max }} ; \lambda_{\max }=\frac{\mathrm{c}}{v_{\min }}
\]
$\lambda_{\min }=\frac{3 \times 10^{8} m / s}{1600 \times 10^{3} Hz}=\boxed{188} m$


\textbf{Problem:}
Subproblem 0: Consider a (111) plane in an FCC structure. How many different [110]-type directions lie in this (111) plane?


\textbf{Solution:}
Let's look at the unit cell.
\includegraphics[scale=0.5]{set_23_img_01.jpg}
\nonessentialimage
There are \boxed{6} [110]-type directions in the (111) plane. Their indices are:
\[
(10 \overline{1}),(\overline{1} 01),(\overline{1} 10),(\overline{1} 0),(0 \overline{1} 1),(01 \overline{1})
\]


\textbf{Problem:}
Subproblem 0: Determine the velocity of an electron (in $\mathrm{m} / \mathrm{s}$ ) that has been subjected to an accelerating potential $V$ of 150 Volt. Please format your answer as $n \times 10^x$, where $n$ is to 2 decimal places. 
(The energy imparted to an electron by an accelerating potential of one Volt is $1.6 \times 10^{-19}$ J oules; dimensional analysis shows that the dimensions of charge $x$ potential correspond to those of energy; thus: 1 electron Volt $(1 \mathrm{eV})=1.6 \times 10^{-19}$ Coulomb $\times 1$ Volt $=1.6 \times 10^{-19}$ Joules.)


\textbf{Solution:}
We know: $E_{\text {kin }}=m v^{2} / 2=e \times V$ (charge applied potential) $\mathrm{m}_{\mathrm{e}}=9.1 \times 10^{-31} \mathrm{~kg}$
\[
\begin{aligned}
&E_{\text {kin }}=e \times V=m v^{2} / 2 \\
&v=\sqrt{\frac{2 \mathrm{eV}}{\mathrm{m}}}=\sqrt{\frac{2 \times 1.6 \times 10^{-19} \times 150}{9.1 \times 10^{-31}}}=\boxed{7.26e6} \mathrm{~m} / \mathrm{s}
\end{aligned}
\]


\textbf{Problem:}
Subproblem 0: In a diffractometer experiment a specimen of thorium (Th) is irradiated with tungsten (W) $L_{\alpha}$ radiation. Calculate the angle, $\theta$, of the $4^{\text {th }}$ reflection. Round your answer (in degrees) to 2 decimal places.


\textbf{Solution:}
$\bar{v}=\frac{1}{\lambda}=\frac{5}{36}(74-7.4)^{2} \mathrm{R} \rightarrow \lambda=1.476 \times 10^{-10} \mathrm{~m}$
Th is FCC with a value of $\mathrm{V}_{\text {molar }}=19.9 \mathrm{~cm}^{3}$
$\therefore \frac{4}{\mathrm{a}^{3}}=\frac{\mathrm{N}_{\mathrm{A}}}{\mathrm{V}_{\text {molar }}} \rightarrow \mathrm{a}=\left(\frac{4 \times 19.9}{6.02 \times 10^{23}}\right)^{1 / 3}=5.095 \times 10^{-8} \mathrm{~cm}$
$\lambda=2 d \sin \theta ; d=\frac{a}{\sqrt{h^{2}+k^{2}+R^{2}}}$
4th reflection in FCC: $111 ; 200 ; 220 ; \mathbf{3 1 1} \rightarrow \mathrm{h}^{2}+\mathrm{k}^{2}+\mathrm{l}^{2}=11$
$\lambda_{\theta}=\frac{2 a \sin \theta}{\sqrt{h^{2}+k^{2}+L^{2}}} \rightarrow=\sin ^{-1}\left(\frac{\lambda \sqrt{h^{2}+k^{2}+l^{2}}}{2 a}\right)=\sin ^{-1}\left(\frac{1.476 \sqrt{11}}{2 \times 5.095}\right)=\boxed{28.71}^{\circ}$


\textbf{Problem:}
Subproblem 0: A metal is found to have BCC structure, a lattice constant of $3.31 \AA$, and a density of $16.6 \mathrm{~g} / \mathrm{cm}^{3}$. Determine the atomic weight of this element in g/mole, and round your answer to 1 decimal place.


\textbf{Solution:}
$B C C$ structure, so $\mathrm{n}=2$
\[
\begin{aligned}
&a=3.31 \AA=3.31 \times 10^{-10} \mathrm{~m} \\
&\rho=16.6 \mathrm{~g} / \mathrm{cm}^{3} \\
&\frac{\text { atomic weight }}{\rho} \times 10^{-6}=\frac{N_{A}}{n} \times a^{3}
\end{aligned}
\]
\[
\begin{aligned}
&\text { atomic weight }=\frac{\left(6.023 \times 10^{23} \text { atoms } / \text { mole }\right)\left(3.31 \times 10^{-10} \mathrm{~m}\right)^{3}}{(2 \text { atoms } / \text { unit cell })\left(10^{-6} \mathrm{~m}^{3} / \mathrm{cm}^{3}\right)} \times 16.6 \mathrm{~g} / \mathrm{cm}^{3} \\
&= \boxed{181.3} \mathrm{~g} / \text { mole }
\end{aligned}
\]


\textbf{Problem:}
Preamble: Iron $\left(\rho=7.86 \mathrm{~g} / \mathrm{cm}^{3}\right.$ ) crystallizes in a BCC unit cell at room temperature.

Subproblem 0: Calculate the radius in cm of an iron atom in this crystal. Please format your answer as $n \times 10^x$ where $n$ is to 2 decimal places.


\textbf{Solution:}
In $\mathrm{BCC}$ there are 2 atoms per unit cell, so $\frac{2}{\mathrm{a}^{3}}=\frac{\mathrm{N}_{\mathrm{A}}}{\mathrm{V}_{\text {molar }}}$, where $\mathrm{V}_{\text {molar }}=\mathrm{A} / \rho ; \mathrm{A}$ is the atomic mass of iron.
\[
\begin{aligned}
&\frac{2}{a^{3}}=\frac{N_{A} \times p}{A} \\
&\therefore a=\left(\frac{2 A}{N_{A} \times \rho}\right)^{\frac{1}{3}}=\frac{4}{\sqrt{3}} r \\
&\therefore r= \boxed{1.24e-8} \mathrm{~cm}
\end{aligned}
\]


\textbf{Problem:}
Preamble: For the element copper (Cu) determine:

Subproblem 0: the distance of second nearest neighbors (in meters). Please format your answer as $n \times 10^x$ where $n$ is to 2 decimal places.


Solution: The answer can be found by looking at a unit cell of $\mathrm{Cu}$ (FCC).
\includegraphics[scale=0.5]{set_23_img_00.jpg}
\nonessentialimage
Nearest neighbor distance is observed along $<110>$; second-nearest along $<100>$. The second-nearest neighbor distance is found to be "a".
Cu: atomic volume $=7.1 \times 10^{-6} \mathrm{~m}^{3} /$ mole $=\frac{\mathrm{N}_{\mathrm{A}}}{4} \mathrm{a}^{3}$ ( $\mathrm{Cu}: \mathrm{FCC} ; 4$ atoms/unit cell) $a=\sqrt[3]{\frac{7.1 \times 10^{-6} \times 4}{6.02 \times 10^{23}}}= \boxed{3.61e-10} \mathrm{~m}$

Final answer: The final answer is 3.61e-10. I hope it is correct.

Subproblem 1: the interplanar spacing of $\{110\}$ planes (in meters). Please format your answer as $n \times 10^x$ where $n$ is to 2 decimal places.


\textbf{Solution:}
$d_{h k l}=\frac{a}{\sqrt{h^{2}+k^{2}+1^{2}}}$
\[
d_{110}=\frac{3.61 \times 10^{-10}}{\sqrt{2}}= \boxed{2.55e-10} \mathrm{~m}
\]


\textbf{Problem:}
Subproblem 0: What is the working temperature for silica glass in Celsius?


Solution: \boxed{1950}.

Final answer: The final answer is 1950. I hope it is correct.

Subproblem 1: What is the softening temperature for silica glass in Celsius?


Solution: \boxed{1700}.

Final answer: The final answer is 1700. I hope it is correct.

Subproblem 2: What is the working temperature for Pyrex in Celsius?


Solution: \boxed{1200}.

Final answer: The final answer is 1200. I hope it is correct.

Subproblem 3: What is the softening temperature for Pyrex in Celsius?


Solution: \boxed{800}.

Final answer: The final answer is 800. I hope it is correct.

Subproblem 4: What is the working temperature for soda-lime glass in Celsius?


\textbf{Solution:}
\boxed{900}.


\textbf{Problem:}
Subproblem 0: Determine the wavelength of $\lambda_{K_{\alpha}}$ for molybdenum (Mo). Please format your answer as $n \times 10^x$ where $n$ is to 2 decimal places, in meters.


\textbf{Solution:}
$M o: Z=42 ; \mathrm{K}_{\alpha} \rightarrow \mathrm{n}_{\mathrm{i}}=2 ; \mathrm{n}_{\mathrm{f}}=1 ; \sigma=1$
\[
\begin{aligned}
&\bar{v}_{\mathrm{K}_{\alpha}}=R(Z-1)^{2}\left[\frac{1}{\mathrm{n}_{\mathrm{f}}^{2}}-\frac{1}{\mathrm{n}_{\mathrm{i}}^{2}}\right] \\
&\bar{v}_{\mathrm{K}_{\alpha}}=1.097 \times 10^{7}\left[\frac{1}{\mathrm{~m}}\right](42-1)^{2}\left[\frac{1}{1^{2}}-\frac{1}{2^{2}}\right] \\
&\bar{v}_{\mathrm{K}_{\alpha}}=1.38 \times 10^{10} \mathrm{~m}^{-1} \\
&\lambda_{\mathrm{K}_{\alpha}}=\frac{1}{\bar{v}_{\mathrm{K}_{\alpha}}}= \boxed{7.25e-11} \mathrm{~m}
\end{aligned}
\]


\textbf{Problem:}
Subproblem 0: Determine the second-nearest neighbor distance (in pm) for nickel (Ni) at $100^{\circ} \mathrm{C}$ if its density at that temperature is $8.83 \mathrm{~g} / \mathrm{cm}^{3}$. Please format your answer as $n \times 10^x$ where $n$ is to 2 decimal places.


\textbf{Solution:}
\[
\begin{array}{ll}
\mathrm{Ni}: \mathrm{n}=4 \\
\text { atomic weight }=58.70 \mathrm{~g} / \mathrm{mole} \\
\rho=8.83 \mathrm{~g} / \mathrm{cm}^{3}
\end{array}
\]
For a face-centered cubic structure, the second nearest neighbor distance equals "a".
\[
\begin{aligned}
& \frac{\text { atomic weight }}{\rho} \times 10^{-6}=\frac{N_{A}}{n} \times a^{3} \\
& a^{3}=\frac{(58.70 \mathrm{~g} / \mathrm{mole})\left(10^{-6} \mathrm{~m}^{3} / \mathrm{cm}^{3}\right)(4 \text { atoms } / \text { unit cell })}{\left(6.023 \times 10^{23} \text { atoms } / \mathrm{mole}\right)\left(8.83 \mathrm{~g} / \mathrm{cm}^{3}\right)} \\
& =4.41 \times 10^{-29} \mathrm{~m}^{3} \\
& \mathrm{a}=3.61 \times 10^{-10} \mathrm{~m} \times \frac{10^{12} \mathrm{pm}}{\mathrm{m}}= \boxed{3.61e2} \mathrm{pm} 
\end{aligned}
\]


\textbf{Problem:}
Subproblem 0: What is the working temperature for silica glass in Celsius?


\textbf{Solution:}
\boxed{1950}.


\textbf{Problem:}
Subproblem 0: What acceleration potential $V$ must be applied to electrons to cause electron diffraction on $\{220\}$ planes of gold $(\mathrm{Au})$ at $\theta=5^{\circ}$ ? Format your answer as an integer, in Volts.


\textbf{Solution:}
We first determine the wavelength of particle waves $\left(\lambda_{p}\right)$ required for diffraction and then the voltage to be applied to the electrons:
\[
\begin{aligned}
&\lambda=2 \mathrm{~d}_{\{220\}} \sin \theta=2 \frac{\mathrm{a}}{\sqrt{8}} \sin 5^{\circ} \\
&\mathrm{a}_{\mathrm{Au}}=\sqrt[3]{\frac{4 \times 10.2 \times 10^{-6}}{6.02 \times 10^{23}}}=4.08 \times 10^{-10} \mathrm{~m} \\
&\lambda=\frac{2 \times 4.08 \times 10^{-10}}{\sqrt{8}} \sin 5^{\circ}=\frac{4.08 \times 10^{-10}}{\sqrt{2}} \times 0.087=0.25 \times 10^{-10} \mathrm{~m}=\lambda_{\mathrm{p}} \\
&\mathrm{eV}=\frac{\mathrm{mv}}{2}, \therefore \mathrm{v}=\sqrt{\frac{2 \mathrm{eV}}{\mathrm{m}}} \\
&\lambda_{\mathrm{P}}=\frac{\mathrm{h}}{\mathrm{mv}}=\frac{\mathrm{h}}{\sqrt{2 \mathrm{meV}}}, \therefore V=\frac{\mathrm{h}^{2}}{2 \lambda^{2} \mathrm{me}}= \boxed{2415} \mathrm{~V}
\end{aligned}
\]


\textbf{Problem:}
Subproblem 0: To increase its corrosion resistance, chromium $(\mathrm{Cr})$ is diffused into steel at $980^{\circ} \mathrm{C}$. If during diffusion the surface concentration of chromium remains constant at $100 \%$, how long will it take (in days) to achieve a $\mathrm{Cr}$ concentration of $1.8 \%$ at a depth of $0.002 \mathrm{~cm}$ below the steel surface? Round your answer to 1 decimal place. $\left(D_{o}=0.54 \mathrm{~cm}^{2} / \mathrm{s} ; E_{A}=286 \mathrm{~kJ} / \mathrm{mol}\right.$ ) 


\textbf{Solution:}
A solution to Fick's second law for the given boundary conditions is:
$\frac{c}{c_{s}}=1-\operatorname{erf} \frac{x}{2 \sqrt{D t}}$, from which we get erf $\frac{x}{2 \sqrt{D t}}=1-0.018=0.982$
From the error function tables, $0.982$ is the erf of $1.67$. This means that
\[
\frac{0.002}{2 \sqrt{D t}}=\frac{0.001}{\sqrt{D t}}=1.67
\]
\[
\begin{aligned}
& \mathrm{D}=\mathrm{D}_{0} \mathrm{e}^{\left(\frac{-286 \times 10^{5}}{8.314 \times 1253}\right)}=6.45 \times 10^{-13} \mathrm{~cm}^{2} / \mathrm{s} \\
& \therefore \mathrm{t}=\frac{0.001^{2}}{1.67^{2} \times 6.45 \times 10^{-13}}=5.56 \times 10^{5} \mathrm{sec}=\boxed{6.4} \text { days }
\end{aligned}
\]


\textbf{Problem:}
Subproblem 0: What is the working temperature for silica glass in Celsius?


Solution: \boxed{1950}.

Final answer: The final answer is 1950. I hope it is correct.

Subproblem 1: What is the softening temperature for silica glass in Celsius?


Solution: \boxed{1700}.

Final answer: The final answer is 1700. I hope it is correct.

Subproblem 2: What is the working temperature for Pyrex in Celsius?


\textbf{Solution:}
\boxed{1200}.


\textbf{Problem:}
Preamble: Calculate the vacancy fraction in copper (Cu) in $\mathrm{~cm}^{-3}$ at the following temperatures. Measurements have determined the values of the enthalpy of vacancy formation, $\Delta \mathrm{H}_{\mathrm{V}}$, to be $1.03 \mathrm{eV}$ and the entropic prefactor, A, to be 1.1. Please format your answers as $n \times 10^x$ where $n$ is to 2 decimal places.

Subproblem 0: $20^{\circ} \mathrm{C}$. 


\textbf{Solution:}
number of sites / unit volume (also known as site density) is given by:
\[
\begin{aligned}
\frac{\mathrm{N}_{\mathrm{A}}}{\mathrm{V}_{\text {molar }}} & \therefore \text { site density }=6.02 \times 10^{23} / 7.11 \mathrm{~cm}^{3}=8.47 \times 10^{22} \\
& \rightarrow \text { vacancy density }=\mathrm{f}_{\mathrm{v}} \times \text { site density }
\end{aligned}
\]
$f_{V}=A e^{-\frac{\Delta H_{V}}{k_{B} T}}=1.1 \times e^{-\frac{1.03 \times 1.6 \times 10^{-19}}{1.38 \times 10^{-22} \times(20+273)}}=2.19 \times 10^{-18}$
vacancy density at $20^{\circ} \mathrm{C}= \boxed{1.85e5} \mathrm{~cm}^{-3}$


\textbf{Problem:}
Preamble: For aluminum at $300 \mathrm{~K}$, 

Subproblem 0: Calculate the planar packing fraction (fractional area occupied by atoms) of the ( 110 ) plane. Please round your answer to 3 decimal places. 


\textbf{Solution:}
Aluminum at $300 \mathrm{~K}$ has FCC structure:
\includegraphics[scale=0.5]{set_23_img_03.jpg}
\nonessentialimage
Volume unit of a cell:
\[
\begin{aligned}
&V=\frac{10 \mathrm{~cm}^{3}}{\text { mole }} \times \frac{1 \text { mole }}{6.02 \times 10^{23} \text { atoms }} \times \frac{4 \text { atoms }}{1 \text { unit cell }} \\
&=6.64 \times 10^{-23} \mathrm{~cm}^{3} / \text { unit cell }
\end{aligned}
\]
For FCC: $\sqrt{2} \mathrm{a}=4 \mathrm{r} \rightarrow$ atomic radius $\mathrm{r}=\frac{\sqrt{2}}{4} \mathrm{a}=\frac{\sqrt{2}}{4}\left(4.05 \times 10^{-8} \mathrm{~cm}\right)$
\[
=1.43 \times 10^{-8} \mathrm{~cm}
\]
Planar packing fraction of the $(110)$ plane:
area of shaded plane in above unit cell $=\sqrt{2} a^{2}$
number of lattice points in the shaded area $=2\left(\frac{1}{2}\right)+4\left(\frac{1}{4}\right)=2$
area occupied by 1 atom $=\pi r^{2}$
packing fraction $=\frac{\text { area occupied by atoms }}{\text { total area }}=\frac{2 \pi \mathrm{r}^{2}}{\sqrt{2} \mathrm{a}^{2}}$
\[
=\frac{2 \pi\left(1.43 \times 10^{-8} \mathrm{~cm}\right)^{2}}{\sqrt{2}\left(4.05 \times 10^{-8} \mathrm{~cm}\right)^{2}}= \boxed{0.554}
\]


\textbf{Problem:}
Subproblem 0: Determine the inter-ionic equilibrium distance in meters between the sodium and chlorine ions in a sodium chloride molecule knowing that the bond energy is $3.84 \mathrm{eV}$ and that the repulsive exponent is 8. Please format your answer as $n \times 10^x$ where $n$ is to 1 decimal place.


\textbf{Solution:}
$\mathrm{E}_{\mathrm{equ}}=-3.84 \mathrm{eV}=-3.84 \times 1.6 \times 10^{-19} \mathrm{~J}=-\frac{\mathrm{e}^{2}}{4 \pi \varepsilon_{0} r_{0}}\left(1-\frac{1}{\mathrm{n}}\right)$
\\
$r_{0}=\frac{\left(1.6 \times 10^{-19}\right)^{2}}{4 \pi 8.85 \times 10^{-12} \times 6.14 \times 10^{-19}}\left(1-\frac{1}{8}\right)= 
\boxed{3.3e-10} \mathrm{~m}$


\textbf{Problem:}
Preamble: A formation energy of $2.0 \mathrm{eV}$ is required to create a vacancy in a particular metal. At $800^{\circ} \mathrm{C}$ there is one vacancy for every 10,000 atoms.

Subproblem 0: At what temperature (in Celsius) will there be one vacancy for every 1,000 atoms? Format your answer as an integer. 


\textbf{Solution:}
We need to know the temperature dependence of the vacancy density:
\[
\frac{1}{10^{4}}=A e^{-\frac{\Delta H_{v}}{k T_{1}}} \quad \text { and } \frac{1}{10^{3}}=A e^{-\frac{\Delta H_{v}}{k T_{x}}}
\]
From the ratio: $\frac{\frac{1}{10^{4}}}{\frac{1}{10^{3}}}=\frac{10^{3}}{10^{4}}=\frac{\mathrm{Ae}^{-\Delta \mathrm{H}_{v} / \mathrm{k} T_{1}}}{\mathrm{Ae}^{-\Delta \mathrm{H}_{v} / \mathrm{kT} \mathrm{x}}}$ we get $-\ln 10=-\frac{\Delta \mathrm{H}_{\mathrm{v}}}{\mathrm{k}}\left(\frac{1}{\mathrm{~T}_{1}}-\frac{1}{\mathrm{~T}_{\mathrm{x}}}\right)$
\[
\begin{aligned}
&\therefore \quad\left(\frac{1}{T_{1}}-\frac{1}{T_{x}}\right)=\frac{k \ln 10}{\Delta H_{v}} \\
&\frac{1}{T_{x}}=\frac{1}{T_{1}}-\frac{k \ln 10}{\Delta H_{v}}=\frac{1}{1073}-\frac{1.38 \times 10^{-23} \times \ln 10}{2 \times 1.6 \times 10^{-19}}=8.33 \times 10^{-4} \\
&T_{x}=1200 \mathrm{~K}= \boxed{928}^{\circ} \mathrm{C}
\end{aligned}
\]


\textbf{Problem:}
Subproblem 0: For $\mathrm{NaF}$ the repulsive (Born) exponent, $\mathrm{n}$, is 8.7. Making use of data given in your Periodic Table, calculate the crystal energy ( $\left.\Delta \mathrm{E}_{\text {cryst }}\right)$ in kJ/mole, to 1 decimal place.


\textbf{Solution:}
\[
\Delta E=\frac{e^{2} N_{A} M}{4 \pi \varepsilon_{0} r_{0}}\left(1-\frac{1}{n}\right)
\]
The assumption must be made that the distance of separation of Na- $F$ is given by the sum of the ionic radii (that in a crystal they touch each other - a not unreasonable assumption). Thus, $r_{0}=0.95 \times 10^{-10}+1.36 \times 10^{-10} \mathrm{~m}=2.31 \AA$ and you must also assume $M$ is the same as for $\mathrm{NaCl}=1.747$ : $\mathrm{E}_{\text {cryst }}=-\frac{\left(1.6 \times 10^{-19}\right)^{2} 6.02 \times 10^{23} \times 1.747}{4 \pi 8.85 \times 10^{-12} \times 2.31 \times 10^{-10}}\left(1-\frac{1}{8.7}\right)$
\\
$\mathrm{E}_{\text {cryst }}=\boxed{927.5} /$ kJ/mole 


\textbf{Problem:}
Preamble: Calculate the molecular weight in g/mole (to 2 decimal places) of each of the substances listed below.

Subproblem 0: $\mathrm{NH}_{4} \mathrm{OH}$


Solution: $\mathrm{NH}_{4} \mathrm{OH}$ :
$5 \times 1.01=5.05(\mathrm{H})$
$1 \times 14.01=14.01(\mathrm{~N})$
$1 \times 16.00=16.00(\mathrm{O})$
$\mathrm{NH}_{4} \mathrm{OH}= \boxed{35.06}$ g/mole

Final answer: The final answer is 35.06. I hope it is correct.

Subproblem 1: $\mathrm{NaHCO}_{3}$


\textbf{Solution:}
$\mathrm{NaHCO}_{3}: 3 \times 16.00=48.00(\mathrm{O})$
$1 \times 22.99=22.99(\mathrm{Na})$
$1 \times 1.01=1.01$ (H)
$1 \times 12.01=12.01$ (C)
$\mathrm{NaHCO}_{3}= \boxed{84.01}$ g/mole


\textbf{Problem:}
Subproblem 0: In iridium (Ir), the vacancy fraction, $n_{v} / \mathrm{N}$, is $3.091 \times 10^{-5}$ at $12340^{\circ} \mathrm{C}$ and $5.26 \times 10^{-3}$ at the melting point. Calculate the enthalpy of vacancy formation, $\Delta \mathrm{H}_{\mathrm{v}}$. Round your answer to 1 decimal place. 


\textbf{Solution:}
All we need to know is the temperature dependence of the vacancy density:
$\frac{n_{v}}{N}=A e^{-\frac{\Delta H_{v}}{R T}}$, where $T$ is in Kelvins and the melting point of $I r$ is $2446^{\circ} \mathrm{C}$
$3.091 \times 10^{-5}=\mathrm{Ae}^{-\frac{\Delta \mathrm{H}_{\mathrm{V}}}{\mathrm{RT}_{1}}}$, where $\mathrm{T}_{1}=1234^{\circ} \mathrm{C}=1507 \mathrm{~K}$
$5.26 \times 10^{-3}=A e^{-\frac{\Delta H_{v}}{R T_{2}}}$, where $T_{2}=2446^{\circ} \mathrm{C}=2719 \mathrm{~K}$
Taking the ratio:
\[
\begin{aligned}
&\frac{5.26 \times 10^{-3}}{3.091 \times 10^{-5}}=\frac{A e^{-\frac{\Delta H_{v}}{R T_{1}}}}{A e^{-\frac{\Delta H_{v}}{R T_{2}}}}=e^{-\frac{\Delta H_{v}}{R}\left(\frac{1}{T_{1}}-\frac{1}{T_{2}}\right)} \\
&\therefore \ln 170.2=-\frac{\Delta H_{v}}{R}\left(\frac{1}{T_{1}}-\frac{1}{T_{2}}\right) \\
&\therefore \Delta H_{v}=-\frac{R \times \ln 170.2}{\frac{1}{1507}-\frac{1}{2719}}=-\frac{8.314 \times \ln 170.2}{\frac{1}{1507}-\frac{1}{2719}}=1.44 \times 10^{5} \mathrm{~J} / \mathrm{mole} \cdot \mathrm{vac} \\
&\therefore \Delta \mathrm{H}_{\mathrm{v}}=\frac{1.44 \times 10^{5}}{6.02 \times 10^{23}}=2.40 \times 10^{-19} \mathrm{~J} / \mathrm{vac}= \boxed{1.5} \mathrm{eV} / \mathrm{vac}
\end{aligned}
\]


\textbf{Problem:}
Subproblem 0: If no electron-hole pairs were produced in germanium (Ge) until the temperature reached the value corresponding to the energy gap, at what temperature (Celsius)  would Ge become conductive? Please format your answer as $n \times 10^x$ where n is to 1 decimal place. $\left(\mathrm{E}_{\mathrm{th}}=3 / 2 \mathrm{kT}\right)$


\textbf{Solution:}
\[
\begin{aligned}
&E_{t h}=\frac{3 K T}{2} ; E_{g}=0.72 \times 1.6 \times 10^{-19} \mathrm{~J} \\
&T=\frac{0.72 \times 1.6 \times 10^{-19} \times 2}{3 \times 1.38 \times 10^{-23}}=5565 \mathrm{~K}=5.3 \times 10^{3}{ }^{\circ} \mathrm{C}
\end{aligned}
\]
The temperature would have to be $\boxed{5.3e3}{ }^{\circ} \mathrm{C}$, about $4400^{\circ} \mathrm{C}$ above the melting point of Ge.


\textbf{Problem:}
Preamble: A first-order chemical reaction is found to have an activation energy $\left(E_{A}\right)$ of 250 $\mathrm{kJ} /$ mole and a pre-exponential (A) of $1.7 \times 10^{14} \mathrm{~s}^{-1}$.

Subproblem 0: Determine the rate constant at $\mathrm{T}=750^{\circ} \mathrm{C}$. Round your answer to 1 decimal place, in units of $\mathrm{s}^{-1}$.


\textbf{Solution:}
$\mathrm{k}=\mathrm{Ae} \mathrm{e}^{-\frac{\mathrm{E}_{\mathrm{A}}}{\mathrm{RT}}}=1.7 \times 10^{14} \times \mathrm{e}^{-\frac{2.5 \times 10^{5}}{8.31 \times 10^{23}}}= \boxed{28.8} \mathrm{~s}^{-1}$


\textbf{Problem:}
Subproblem 0: A cubic metal $(r=0.77 \AA$ ) exhibits plastic deformation by slip along $<111>$ directions. Determine its planar packing density (atoms $/ \mathrm{m}^{2}$) for its densest family of planes. Please format your answer as $n \times 10^x$ where $n$ is to 2 decimal places.


\textbf{Solution:}
Slip along $<111>$ directions suggests a BCC system, corresponding to $\{110\},<111>$ slip. Therefore:
\[
\begin{aligned}
&a \sqrt{3}=4 r \\
&a=\frac{4 r}{\sqrt{3}}=1.78 \times 10^{-10} \mathrm{~m}
\end{aligned}
\]
Densest planes are $\{110\}$, so we find:
\[
\frac{2 \text { atoms }}{a^{2} \sqrt{2}}=\boxed{4.46e19} \text { atoms } / \mathrm{m}^{2}
\]


\textbf{Problem:}
Subproblem 0: Determine the total void volume $(\mathrm{cm}^{3} / mole)$ for gold (Au) at $27^{\circ} \mathrm{C}$; make the hard-sphere approximation in your calculation. Note that the molar volume of gold (Au) is $10.3 \mathrm{~cm}^{3} / \mathrm{mole}$. Please round your answer to 2 decimal places.


\textbf{Solution:}
First determine the packing density for Au, which is $\mathrm{FC}$; then relate it to the molar volume given in the periodic table.
\[
\begin{aligned}
&\text { packing density }=\frac{\text { volume of atoms/unit cell }}{\text { volume of unit cell }}=\frac{\frac{16 \pi \mathrm{r}^{3}}{3}}{\mathrm{a}^{3}}=\frac{16 \pi \mathrm{r}^{3}}{3 \mathrm{a}^{3}} \\
&\text { packing density }=\frac{16 \pi \mathrm{r}^{3}}{3 \times 16 \sqrt{2} \mathrm{r}^{3}}=\frac{\pi}{3 \sqrt{2}}=0.74=74 \% \\
&\text { void volume }=1-\text { packing density }=26 \%
\end{aligned}
\]
From the packing density $(74 \%)$ we recognize the void volume to be $26 \%$. Given the molar volume as $10.3 \mathrm{~cm}^{3} / \mathrm{mole}$, the void volume is:
\[
0.26 \times 10.3 \mathrm{~cm}^{3} / \text { mole }= \boxed{2.68} \mathrm{~cm}^{3} / \text { mole }
\]


\textbf{Problem:}
Subproblem 0: What is the working temperature for silica glass in Celsius?


Solution: \boxed{1950}.

Final answer: The final answer is 1950. I hope it is correct.

Subproblem 1: What is the softening temperature for silica glass in Celsius?


Solution: \boxed{1700}.

Final answer: The final answer is 1700. I hope it is correct.

Subproblem 2: What is the working temperature for Pyrex in Celsius?


Solution: \boxed{1200}.

Final answer: The final answer is 1200. I hope it is correct.

Subproblem 3: What is the softening temperature for Pyrex in Celsius?


Solution: \boxed{800}.

Final answer: The final answer is 800. I hope it is correct.

Subproblem 4: What is the working temperature for soda-lime glass in Celsius?


Solution: \boxed{900}.

Final answer: The final answer is 900. I hope it is correct.

Subproblem 5: What is the softening temperature for soda-lime glass in Celsius?


\textbf{Solution:}
\boxed{700}.


\textbf{Problem:}
Subproblem 0: What is the maximum wavelength $(\lambda)$ (in meters) of radiation capable of second order diffraction in platinum (Pt)? Please format your answer as $n \times 10^x$ where $n$ is to 2 decimal places.


\textbf{Solution:}
The longest wavelength capable of $1^{\text {st }}$ order diffraction in Pt can be identified on the basis of the Bragg equation: $\lambda=2 \mathrm{~d} \sin \theta . \lambda_{\max }$ will diffract on planes with maximum interplanar spacing (in compliance with the selection rules): $\{111\}$ at the maximum value $\theta\left(90^{\circ}\right)$. We determine the lattice constant a for $\mathrm{Pt}$, and from it obtain $\mathrm{d}_{\{111\}}$. Pt is FCC with a value of atomic volume or $V_{\text {molar }}=9.1 \mathrm{~cm}^{3} /$ mole.
\[
\mathrm{V}_{\text {molar }}=\frac{N_{\mathrm{A}}}{4} \mathrm{a}^{3} ; \mathrm{a}=\sqrt[3]{\frac{9.1 \times 10^{-6} \times 4}{\mathrm{~N}_{\mathrm{A}}}}=3.92 \times 10^{-10} \mathrm{~m}
\]
If we now look at $2^{\text {nd }}$ order diffraction, we find $2 \lambda=2 \mathrm{~d}_{\{111\}} \sin 90^{\circ}$
\[
\therefore \lambda_{\max }=\mathrm{d}_{\{111\}}=\frac{\mathrm{a}}{\sqrt{3}}=\frac{3.92 \times 10^{-10}}{\sqrt{3}}= \boxed{2.26e-10} \mathrm{~m}
\]


\textbf{Problem:}
Subproblem 0: What is the activation energy of a process which is observed to increase by a factor of three when the temperature is increased from room temperature $\left(20^{\circ} \mathrm{C}\right)$ to $40^{\circ} \mathrm{C}$ ? Round your answer to 1 decimal place, and express it in $\mathrm{~kJ} / \mathrm{mole}$.


\textbf{Solution:}
\[
\mathrm{k}_{1}=A \mathrm{e}^{\frac{-E_{A}}{R T_{1}}} ; k_{2}=3 k_{1}=A e^{\frac{-E_{A}}{R T_{2}}} \rightarrow \frac{1}{3}=e^{-\frac{E_{A}}{R}\left(\frac{1}{T_{1}}-\frac{1}{T_{2}}\right)}
\]
\[
\begin{aligned}
&\ln 3=\frac{E_{A}}{R}\left(\frac{1}{T_{1}}-\frac{1}{T_{2}}\right) \\
&E_{A}=\frac{R \times \ln 3}{\frac{1}{293}-\frac{1}{313}}=4.19 \times 10^{4}= \boxed{41.9} \mathrm{~kJ} / \mathrm{mole}
\end{aligned}
\]


\textbf{Problem:}
Subproblem 0: How much oxygen (in kg, to 3 decimal places) is required to completely convert 1 mole of $\mathrm{C}_{2} \mathrm{H}_{6}$ into $\mathrm{CO}_{2}$ and $\mathrm{H}_{2} \mathrm{O}$ ?


\textbf{Solution:}
To get the requested answer, let us formulate a ``stoichiometric'' equation (molar quantities) for the reaction: $\mathrm{C}_{2} \mathrm{H}_{6}+70 \rightarrow 2 \mathrm{CO}_{2}+3 \mathrm{H}_{2} \mathrm{O}_{\text {. Each } \mathrm{C}_{2} \mathrm{H}_{6}}$ (ethane) molecule requires 7 oxygen atoms for complete combustion. In molar quantities: 1 mole of $\mathrm{C}_{2} \mathrm{H}_{6}=2 \times 12.01+6 \times 1.008=30.07 \mathrm{~g}$ requires
$7 \times 15.9984 \mathrm{~g}=1.12 \times 10^{2}$ oxygen $=\boxed{0.112} kg$ oxygen
We recognize the oxygen forms molecules, $\mathrm{O}_{2}$, and therefore a more appropriate formulation would be: $\mathrm{C}_{2} \mathrm{H}_{6}+7 / 2 \mathrm{O}_{2} \rightarrow 2 \mathrm{CO}_{2}+3 \mathrm{H}_{2} \mathrm{O}$. The result would be the same.


\textbf{Problem:}
Subproblem 0: Determine the differences in relative electronegativity $(\Delta x$ in $e V)$ for the systems ${H}-{F}$ and ${C}-{F}$ given the following data:
$\begin{array}{cl}\text { Bond Energy } & {kJ} / \text { mole } \\ {H}_{2} & 436 \\ {~F}_{2} & 172 \\ {C}-{C} & 335 \\ {H}-{F} & 565 \\ {C}-{H} & 410\end{array}$
\\
Please format your answer to 2 decimal places. 


\textbf{Solution:}
According to Pauling, the square of the difference in electro negativity for two elements $\left(X_{A}-X_{B}\right)^{2}$ is given by the following relationship: $\left(X_{A}-X_{B}\right)^{2}=[$ Bond Energy $(A-B)-\sqrt{\text { Bond Energy AA. Bond Energy } B B}] \times \frac{1}{96.3}$
If bond energies are given in ${kJ}$.\\
$\left(X_{H}-X_{F}\right)^{2}=[565-\sqrt{436 \times 172}] \frac{1}{96.3}=3.02$
\[
\begin{aligned}
& \left({X}_{{H}}-{X}_{{F}}\right)=\sqrt{3.02}=1.7 \\
& \left(X_{C}-X_{H}\right)^{2}=[410-\sqrt{335 \times 436}] \frac{1}{96.3}=0.29 \\
& \left(X_{C}-X_{H}\right)=\sqrt{0.29}= \boxed{0.54}
\end{aligned}
\]


\textbf{Problem:}
Preamble: The number of electron-hole pairs in intrinsic germanium (Ge) is given by:
\[
n_{i}=9.7 \times 10^{15} \mathrm{~T}^{3 / 2} \mathrm{e}^{-\mathrm{E}_{g} / 2 \mathrm{KT}}\left[\mathrm{cm}^{3}\right] \quad\left(\mathrm{E}_{\mathrm{g}}=0.72 \mathrm{eV}\right)
\]

Subproblem 0: What is the density of pairs at $\mathrm{T}=20^{\circ} \mathrm{C}$, in inverse $\mathrm{cm}^3$? Please format your answer as $n \times 10^x$ where n is to 2 decimal places.


\textbf{Solution:}
Recall: $\mathrm{T}$ in thermally activated processes is the absolute temperature: $\mathrm{T}^{\circ} \mathrm{K}=$ $\left(273.16+\mathrm{t}^{\circ} \mathrm{C}\right)$; Boltzmann's constant $=\mathrm{k}=1.38 \times 10^{-23} \mathrm{~J} /{ }^{\circ} \mathrm{K}$
$\mathrm{T}=293.16 \mathrm{~K}:$
\[
\begin{aligned}
&n_{i}=9.7 \times 10^{15} \times 293.16^{\frac{3}{2}} \times e^{-\frac{0.72 \times 16 \times 10^{-19}}{2 \times 1.38 \times 10^{-23} \times 293.16}} \\
&=9.7 \times 10^{15} \times 5019 \times 6.6 \times 10^{-7} \\
&n_{i}= \boxed{3.21e13} / \mathrm{cm}^{3}
\end{aligned}
\]


\textbf{Problem:}
Preamble: For light with a wavelength $(\lambda)$ of $408 \mathrm{~nm}$ determine:

Subproblem 0: the frequency in $s^{-1}$. Please format your answer as $n \times 10^x$, where $n$ is to 3 decimal places. 


Solution: To solve this problem we must know the following relationships:
\[
\begin{aligned}
v \lambda &=c
\end{aligned}
\]
$v$ (frequency) $=\frac{c}{\lambda}=\frac{3 \times 10^{8} m / s}{408 \times 10^{-9} m}= \boxed{7.353e14} s^{-1}$

Final answer: The final answer is 7.353e14. I hope it is correct.

Subproblem 1: the wave number in $m^{-1}$. Please format your answer as $n \times 10^x$, where $n$ is to 2 decimal places.


\textbf{Solution:}
To solve this problem we must know the following relationships:
\[
\begin{aligned}
1 / \lambda=\bar{v} 
\end{aligned}
\]
$\bar{v}$ (wavenumber) $=\frac{1}{\lambda}=\frac{1}{408 \times 10^{-9} m}=\boxed{2.45e6} m^{-1}$


\textbf{Problem:}
Subproblem 0: Calculate the volume in mL of $0.25 \mathrm{M} \mathrm{NaI}$ that would be needed to precipitate all the $\mathrm{g}^{2+}$ ion from $45 \mathrm{~mL}$ of a $0.10 \mathrm{M} \mathrm{Hg}\left(\mathrm{NO}_{3}\right)_{2}$ solution according to the following reaction:
\[
2 \mathrm{NaI}(\mathrm{aq})+\mathrm{Hg}\left(\mathrm{NO}_{3}\right)_{2}(\mathrm{aq}) \rightarrow \mathrm{HgI}_{2}(\mathrm{~s})+2 \mathrm{NaNO}_{3}(\mathrm{aq})
\]


\textbf{Solution:}
\[
\begin{aligned}
&2 \mathrm{NaI}(\mathrm{aq})+\mathrm{Hg}\left(\mathrm{NO}_{3}\right)_{2}(\mathrm{aq}) \rightarrow \mathrm{HgI}_{2}(\mathrm{~s})+\mathrm{NaNO}_{3}(\mathrm{aq}) \\
&\frac{0.10 \mathrm{~mol} \mathrm{Hg}\left(\mathrm{NO}_{3}\right)_{2}}{1 \mathrm{~L}} \times 0.045 \mathrm{~L}=4.5 \times 10^{-3} \mathrm{~mol} \mathrm{Hg}\left(\mathrm{NO}_{3}\right)_{2} \\
&4.5 \times 10^{-3} \mathrm{~mol} \mathrm{Hg}\left(\mathrm{NO}_{3}\right)_{2} \times \frac{2 \mathrm{~mol} \mathrm{NaI}}{1 \mathrm{~mol} \mathrm{Hg}\left(\mathrm{NO}_{3}\right)_{2}}=9.00 \times 10^{-3} \mathrm{~mol} \mathrm{NaI} \\
&\frac{9.00 \times 10^{-3} \mathrm{~mol} \mathrm{NaI}}{0.25 \frac{\mathrm{mol} \mathrm{NaI}}{\mathrm{L}}}=3.6 \times 10^{-2} \mathrm{~L} \times \frac{1000 \mathrm{ml}}{1 \mathrm{~L}}=\boxed{36} \mathrm{~mL} \mathrm{NaI}
\end{aligned}
\]


\textbf{Problem:}
Subproblem 0: A slab of plate glass containing dissolved helium (He) is placed in a vacuum furnace at a temperature of $400^{\circ} \mathrm{C}$ to remove the helium from the glass. Before vacuum treatment, the concentration of helium is constant throughout the glass. After 10 minutes in vacuum at $400^{\circ} \mathrm{C}$, at what depth (in $\mu \mathrm{m}$) from the surface of the glass has the concentration of helium decreased to $1 / 3$ of its initial value? The diffusion coefficient of helium in the plate glass at the processing temperature has a value of $3.091 \times 10^{-6} \mathrm{~cm}^{2} / \mathrm{s}$.


\textbf{Solution:}
\includegraphics[scale=0.5]{set_37_img_01.jpg}
\nonessentialimage
\[
\begin{aligned}
&c=A+B \text { erf } \frac{x}{2 \sqrt{D t}} ; c(0, t)=0=A ; c(\infty, t)=c_{0}=B \\
&\therefore c(x, t)=c_{0} \operatorname{erf} \frac{x}{2 \sqrt{D t}}
\end{aligned}
\]
What is $\mathrm{x}$ when $\mathrm{c}=\mathrm{c}_{0} / 3$ ?
\[
\begin{gathered}
\frac{c_{0}}{3}=c_{0} \operatorname{erf} \frac{x}{2 \sqrt{D t}} \rightarrow 0.33=\operatorname{erf} \frac{x}{2 \sqrt{D t}} ; \operatorname{erf}(0.30)=0.3286 \approx 0.33 \\
\therefore \frac{x}{2 \sqrt{D t}}=0.30 \rightarrow x=2 \times 0.30 \times \sqrt{3.091 \times 10^{-6} \times 10 \times 60}=2.58 \times 10^{-2} \mathrm{~cm}=\boxed{258} \mu \mathrm{m}
\end{gathered}
\]


\textbf{Problem:}
Subproblem 0: What is the working temperature for silica glass in Celsius?


Solution: \boxed{1950}.

Final answer: The final answer is 1950. I hope it is correct.

Subproblem 1: What is the softening temperature for silica glass in Celsius?


\textbf{Solution:}
\boxed{1700}.


\textbf{Problem:}
Preamble: Two lasers generate radiation of (1) $9.5 \mu {m}$ and (2) $0.1 \mu {m}$ respectively.

Subproblem 0: Determine the photon energy (in eV, to two decimal places) of the laser generating radiation of $9.5 \mu {m}$.


Solution: \[
\begin{aligned}
{E} &={h} v=\frac{{hc}}{\lambda} {J} \times \frac{1 {eV}}{1.6 \times 10^{-19} {~J}} \\
{E}_{1} &=\frac{{hc}}{9.5 \times 10^{-6}} \times \frac{1}{1.6 \times 10^{-19}} {eV}= \boxed{0.13} {eV}
\end{aligned}
\]

Final answer: The final answer is 0.13. I hope it is correct.

Subproblem 1: Determine the photon energy (in eV, to one decimal place) of the laser generating radiation of $0.1 \mu {m}$. 


\textbf{Solution:}
\[
\begin{aligned}
{E} &={h} v=\frac{{hc}}{\lambda} {J} \times \frac{1 {eV}}{1.6 \times 10^{-19} {~J}} \\
{E}_{2} &=\frac{{hc}}{0.1 \times 10^{-6}} \times \frac{1}{1.6 \times 10^{-19}} {eV}= \boxed{12.4} {eV}
\end{aligned}
\]


\textbf{Problem:}
Preamble: $\mathrm{Bi}_{2} \mathrm{~S}_{3}$ dissolves in water according to the following reaction:
\[
\mathrm{Bi}_{2} \mathrm{~S}_{3}(\mathrm{~s}) \Leftrightarrow 2 \mathrm{Bi}^{3+}(\mathrm{aq})+3 \mathrm{~s}^{2-}(\mathrm{aq})
\]
for which the solubility product, $\mathrm{K}_{\mathrm{sp}}$, has the value of $1.6 \times 10^{-72}$ at room temperature.

Subproblem 0: At room temperature how many moles of $\mathrm{Bi}_{2} \mathrm{~S}_{3}$ will dissolve in $3.091 \times 10^{6}$ liters of water? Please format your answer as $n \times 10^x$ where $n$ is to 1 decimal place.


\textbf{Solution:}
$\mathrm{Bi}_{2} \mathrm{~S}_{3}=2 \mathrm{Bi}^{3+}(\mathrm{aq})+3 \mathrm{~S}^{2-}(\mathrm{aq})$
\[
\therefore\left[\mathrm{Bi}^{3+}\right]=2 \mathrm{C}_{\mathrm{s}} \text { and }\left[\mathrm{s}^{2}\right]=3 \mathrm{C}_{\mathrm{s}}
\]
\[
\begin{aligned}
& \therefore \mathrm{K}_{\mathrm{sp}}=\left(2 \mathrm{C}_{\mathrm{s}}\right)^{2}\left(3 \mathrm{C}_{\mathrm{s}}\right)^{3}=4 \mathrm{C}_{\mathrm{s}}^{2} \cdot 27 \mathrm{C}_{\mathrm{s}}^{3}=108 \mathrm{C}_{\mathrm{s}}^{5} \\
& \therefore C_{\mathrm{s}}=\left(\frac{\mathrm{K}_{\mathrm{sp}}}{108}\right)^{1 / 5}=1.715 \times 10^{-15} \mathrm{~mol} / \mathrm{L} \\
& \therefore \text { in } 3.091 \times 10^{6} \mathrm{~L} \Rightarrow \boxed{5.3e-9} \mathrm{~mol} \mathrm{Bi}_{2} \mathrm{~S}_{3}
\end{aligned}
\]


\textbf{Problem:}
Subproblem 0: Whiskey, suspected to be of the "moonshine" variety, is analyzed for its age by determining its amount of naturally occurring tritium (T) which is a radioactive hydrogen isotope $\left({ }^{3} \mathrm{H}\right)$ with a half-life of $12.5$ years. In this "shine" the activity is found to be $6 \%$ of that encountered in fresh bourbon. What is the age (in years) of the whiskey in question?


\textbf{Solution:}
\[
\begin{aligned}
&\frac{c_{o}}{c}=e^{k t} ; c=0.06 c_{0} \\
&\ln \frac{c_{0}}{0.06 c_{0}}=k t_{x} \\
&\ln 0.06=-k_{x} \\
&t_{x}=-\frac{\ln 0.06}{\frac{\ln 2}{t_{1 / 2}}}=\frac{\ln 0.06}{\frac{0.693}{12.5}}= \boxed{50.7} \text { years }
\end{aligned}
\]


\textbf{Problem:}
Subproblem 0: What is the working temperature for silica glass in Celsius?


Solution: \boxed{1950}.

Final answer: The final answer is 1950. I hope it is correct.

Subproblem 1: What is the softening temperature for silica glass in Celsius?


Solution: \boxed{1700}.

Final answer: The final answer is 1700. I hope it is correct.

Subproblem 2: What is the working temperature for Pyrex in Celsius?


Solution: \boxed{1200}.

Final answer: The final answer is 1200. I hope it is correct.

Subproblem 3: What is the softening temperature for Pyrex in Celsius?


\textbf{Solution:}
\boxed{800}.


\textbf{Problem:}
Preamble: A first-order chemical reaction is found to have an activation energy $\left(E_{A}\right)$ of 250 $\mathrm{kJ} /$ mole and a pre-exponential (A) of $1.7 \times 10^{14} \mathrm{~s}^{-1}$.

Subproblem 0: Determine the rate constant at $\mathrm{T}=750^{\circ} \mathrm{C}$. Round your answer to 1 decimal place, in units of $\mathrm{s}^{-1}$.


Solution: $\mathrm{k}=\mathrm{Ae} \mathrm{e}^{-\frac{\mathrm{E}_{\mathrm{A}}}{\mathrm{RT}}}=1.7 \times 10^{14} \times \mathrm{e}^{-\frac{2.5 \times 10^{5}}{8.31 \times 10^{23}}}= \boxed{28.8} \mathrm{~s}^{-1}$

Final answer: The final answer is 28.8. I hope it is correct.

Subproblem 1: What percent of the reaction will be completed at $600^{\circ} \mathrm{C}$ in a period of 10 minutes?


\textbf{Solution:}
Requires knowledge of $k_{600}$ :
\[
\begin{aligned}
&\mathrm{k}_{600}=1.7 \times 10^{14} \times \mathrm{e}^{-\frac{2.5 \times 10^{5}}{8.31 \times 873}}=0.184 \\
&\frac{\mathrm{c}}{\mathrm{c}_{0}}=\mathrm{e}^{-\mathrm{kt}}=\mathrm{e}^{-0.184 \times 600}=1.3 \times 10^{-48} \approx 0
\end{aligned}
\]
$c=0$ means the reaction is essentially $ \boxed{100} \%$ complete.


\textbf{Problem:}
Subproblem 0: Determine the energy gap (in eV) between the electronic states $n=7$ and $n=8$ in hydrogen. Please format your answer as $n \times 10^x$ where $n$ is to 1 decimal place. 


\textbf{Solution:}
Here we need to know the "basis" of the Rydberg equation [ $E_{e l}=-\left(1 / n^{2}\right) K$ ] and $1 {eV}=1.6 \times 10^{-19} {~J}$ :
\[
\begin{aligned}
&\Delta {E}_{{el}}={K}\left(\frac{1}{{n}_{{i}}^{2}}-\frac{1}{{n}_{{f}}^{2}}\right)=2.18 \times 10^{-18}\left(\frac{1}{49}-\frac{1}{64}\right)=1.043 \times 10^{-20} {~J} \\
&\Delta {E}_{{el}}=1.043 \times 10^{-20} {~J} \times \frac{1 {eV}}{\left(1.6 \times 10^{-19} {~J}\right)}= \boxed{6.5e-2} {eV}
\end{aligned}
\]


\textbf{Problem:}
Preamble: The decay rate of ${ }^{14} \mathrm{C}$ in living tissue is $15.3$ disintegrations per minute per gram of carbon. Experimentally, the decay rate can be measured to $\pm 0.1$ disintegrations per minute per gram of carbon. The half-life of ${ }^{14} \mathrm{C}$ is 5730 years.

Subproblem 0: What is the maximum age of a sample that can be dated, in years?


\textbf{Solution:}
Radioactive decay is a $1^{\text {st }}$ order reaction which can be modeled as:
\[
-\frac{d c}{d t}=k c \text { or } c=c_{0} e^{-k t}
\]
With a little algebra we can get an expression for the relationship between time, $\mathrm{t}$, and the instant value of the decay rate.
At any time, t, we can write $\quad-\frac{\mathrm{dc}}{\mathrm{dt}}=\mathrm{kc}=\mathrm{kc}_{0} \mathrm{e}^{-\mathrm{kt}}$
and at time zero,
\[
-\frac{d c}{d t}=k c_{0}
\]
Divide eq. 1 by eq. 2 to get
where to reduce clutter let $r=\frac{d c}{d t}$
Take the logarithm of both sides of eq. 3 and substitute $k=\frac{\ln 2}{t_{1 / 2}}$.
With rearrangement, this gives $\quad t=-\frac{t_{1 / 2}}{\ln 2} \times \ln \frac{r_{t}}{r_{0}}$
So, for the oldest specimen we would measure the minimum instant decay rate of $0.1 \pm 0.1$ disintegrations per minute per gram. Set this equal to $r_{t}$ in eq. 4 and solve for $t$ to get $\boxed{41585} \pm 5730$ years.


\textbf{Problem:}
Subproblem 0: Estimate the ionic radius of ${Cs}^{+}$ in Angstroms to 2 decimal places. The lattice energy of $\mathrm{CsCl}$ is $633 \mathrm{~kJ} / \mathrm{mol}$. For $\mathrm{CsCl}$ the Madelung constant, $\mathrm{M}$, is $1.763$, and the Born exponent, $\mathrm{n}$, is 10.7. The ionic radius of $\mathrm{Cl}^{-}$is known to be $1.81 \AA$.


\textbf{Solution:}
\[
\mathrm{E}_{\text {lattice }}=\frac{\mathrm{Mq}_{1} \mathrm{q}_{2}}{4 \pi \varepsilon_{0} r_{\mathrm{o}}}\left(1-\frac{1}{\mathrm{n}}\right) \text { and } \mathrm{r}_{\mathrm{o}}=\mathrm{r}_{\mathrm{Cs}^{+}}+\mathrm{r}_{\mathrm{Cl}}
\]
Solve first for $r_{0}$
\[
\begin{aligned}
r_{0} &=\frac{M q_{1} q_{2} N_{A v}}{4 \pi \varepsilon_{0} E_{\text {lattice }}}\left(1-\frac{1}{n}\right)=\frac{1.763\left(1.6 \times 10^{-19}\right)^{2} 6.02 \times 10^{23}}{4 \pi 8.85 \times 10^{-12} 6.33 \times 10^{5}}\left(1-\frac{1}{10.7}\right) \\
&=3.50 \times 10^{-10} \mathrm{~m}=3.50 \AA=r_{\mathrm{Cs}^{+}}+r_{\mathrm{Cr}} \\
\therefore & r_{\mathrm{Cs}^{+}}=3.50-1.81=\boxed{1.69} \AA
\end{aligned}
\]


\textbf{Problem:}
Subproblem 0: Given the ionic radii, $\mathrm{Cs}^{+}=1.67 \AA, \mathrm{Cl}^{-}=1.81 \AA$, and the Madelung constant $\mathrm{M}(\mathrm{CsCl})=1.763$, determine to the best of your ability the molar Crystal energy ( $\Delta \mathrm{E}_{\text {cryst }}$ ) for $\mathrm{CsCl}$. Please format your answer as $n \times 10^x$ where n is to 2 decimal places; answer in $\mathrm{J} / \text{mole}$. 


\textbf{Solution:}
Given the radii $\mathrm{Cs}^{+}=1.67 \AA$ and $\mathrm{Cl}^{-}=1.81 \AA$, we can assume that $\mathrm{r}_{0}$ is the sum of the two. However, we need to know the exponential constant of the repulsive term which is not provided. Considering only the attractive force:
\[
\begin{array}{ll}
\Delta \mathrm{E}_{\text {cryst }}=\frac{-\mathrm{e}^{2} \mathrm{~N}_{\mathrm{A}} \mathrm{MQ}_{1} \mathrm{Q}_{2}}{4 \pi \varepsilon_{0} r_{0}} & \text { where: } \mathrm{Q}_{1}=\mathrm{Q}_{2}=1 \\
& \mathrm{M}=1.763 \\
& \mathrm{~N}_{\mathrm{A}}=6.02 \times 10^{23} \text { particle/mole }
\end{array}
\]
\[
\begin{aligned}
& \Delta \mathrm{E}_{\text {cryst }}=\frac{-\left(1.6 \times 10^{-19} \mathrm{coul}\right)^{2} \times 6.02 \times 10^{23} \times 1.763 \times 1 \times 1}{4 \pi 8.85 \times 10^{-12} \times(1.81+1.67) \times 10^{-10} \mathrm{~m}} \\
& = \boxed{7.02e5} \mathrm{~J} / \text { mole }
\end{aligned}
\]


\textbf{Problem:}
Subproblem 0: Determine the amount (in grams) of boron (B) that, substitutionally incorporated into $1 \mathrm{~kg}$ of germanium (Ge), will establish a charge carrier density of $3.091 \mathrm{x}$ $10^{17} / \mathrm{cm}^{3}$. Please format your answer as $n \times 10^x$ where $n$ is to 2 decimal places. 


\textbf{Solution:}
The periodic table gives the molar volume of Ge as $13.57 \mathrm{~cm}^{3}$ and 1 mole of Ge weighs $72.61 \mathrm{~g}$, so set up the ratio $\frac{72.61}{13.6}=\frac{1000 \mathrm{~g}}{\mathrm{x}}$ and solve for $\mathrm{x}$ to get $187.30$ $\mathrm{cm}^{3}$ for the total volume. The addition of boron gives 1 charge carrier/B atom.
$\rightarrow \mathrm{B}$ concentration in Si must be $3.091 \times 10^{17} \mathrm{~B} / \mathrm{cm}^{3}$
$\mathrm{N}_{\mathrm{A}}$ of $B$ atoms weighs $10.81 \mathrm{~g}$
$\therefore 3.091 \times 10^{17} \mathrm{~B}$ atoms weigh $\frac{3.091 \times 10^{17}}{6.02 \times 10^{23}} \times 10.81=5.55 \times 10^{-6} \mathrm{~g}$
$\therefore$ for every $1 \mathrm{~cm}^{3}$ of Ge, add $5.55 \times 10^{-6} \mathrm{~g} \mathrm{~B}$
$\rightarrow$ for $187.30 \mathrm{~cm}^{3}$ of Ge, add $187.30 \times 5.55 \times 10^{-6}= \boxed{1.04e-3} \mathrm{~g} \mathrm{~B}$


\textbf{Problem:}
Subproblem 0: Is an energy level of $-1.362 \times 10^{-19} {~J}$ an allowed electron energy state in atomic hydrogen?


Solution: $E_{e l} =-\frac{1}{n^{2}} {~K}$ \\
$-1.362 \times 10^{-19} {~J}=-\frac{1}{{n}^{2}} \times 2.18 \times 10^{-18} {~J}$\\
${n} &=\sqrt{\frac{2.18 \times 10^{-18}}{1.362 \times 10^{-19}}}=4.00$\\
The answer is \boxed{Yes}.

Final answer: The final answer is Yes. I hope it is correct.

Subproblem 1: If your answer is yes, determine its principal quantum number $(n)$. If your answer is no, determine ${n}$ for the "nearest allowed state".


\textbf{Solution:}
n = \boxed{4}. 


\textbf{Problem:}
Subproblem 0: Determine the highest linear density of atoms (atoms/m) encountered in vanadium (V). Please format your answer as $n \times 10^x$ where $n$ is to 2 decimal places.


\textbf{Solution:}
\[
\begin{aligned}
&\mathrm{V}: \quad \text { atomic weight }=50.94 \mathrm{~g} / \text { mole } \\
&\rho=5.8 \mathrm{~g} / \mathrm{cm}^{3}
\end{aligned}
\]
$B C C$, so $n=2$
The highest density would be found in the [111] direction. To find "a":
\[
\begin{aligned}
&\frac{\text { atomic weight }}{\rho}=a^{3} \frac{N_{A}}{n} \rightarrow a^{3}=\frac{50.94 \times 2}{5.8 \times 6.023 \times 10^{23}} \\
&a=3.08 \times 10^{-8} \mathrm{~cm}=3.08 \times 10^{-10} \mathrm{~m}
\end{aligned}
\]
The length in the [111] direction is $\mathrm{a} \sqrt{3}$, so there are:
\[
\begin{aligned}
&2 \text { atoms } / \mathrm{a} \sqrt{3}=2 \text { atoms/ }\left(3.08 \times 10^{-10} \mathrm{~m} \times \sqrt{3}\right) \\
&= \boxed{3.75e9} \text { atoms } / \mathrm{m}
\end{aligned}
\]


\textbf{Problem:}
Subproblem 0: Strontium fluoride, $\mathrm{SrF}_{2}$, has a $\mathrm{K}_{\mathrm{sp}}$ value in water of $2.45 \times 10^{-9}$ at room temperature.
Calculate the solubility of $\mathrm{SrF}_{2}$ in water. Express your answer in units of molarity. Please format your answer as $n \times 10^x$ where $n$ is to 2 decimal places.


\textbf{Solution:}
\[
\begin{aligned}
&\mathrm{SrF}_{2}=\mathrm{Sr}^{2+}+2 \mathrm{~F}^{-} \quad \mathrm{K}_{\mathrm{sp}}=\left[\mathrm{Sr}^{2+}\right]\left[\mathrm{F}^{-}\right]^{2}, \quad \text { but }[\mathrm{F}]=2\left[\mathrm{Sr}^{2+}\right]=2 \mathrm{c}_{\mathrm{s}} \\
&\therefore \mathrm{K}_{\mathrm{sp}}=\mathrm{c}_{\mathrm{s}}\left(2 \mathrm{c}_{\mathrm{s}}\right)^{2}=4 \mathrm{c}_{\mathrm{s}}^{3} \quad \therefore \quad \mathrm{c}_{\mathrm{s}}=\left(\frac{\mathrm{K}_{\mathrm{sp}}}{4}\right)^{1 / 3}= \boxed{8.49e-4} \mathrm{M}
\end{aligned}
\]


\textbf{Problem:}
Subproblem 0: You wish to dope a single crystal of silicon (Si) with boron (B). The specification reads $5 \times 10^{16}$ boron atoms/ $\mathrm{cm}^{3}$ at a depth of $25 \mu \mathrm{m}$ from the surface of the silicon. What must be the effective concentration of boron in units of atoms/ $\mathrm{cm}^{3}$ if you are to meet this specification within a time of 90 minutes? Round your answer to 4 decimal places. Assume that initially the concentration of boron in the silicon crystal is zero. The diffusion coefficient of boron in silicon has a value of $7.23 \times 10^{-9} \mathrm{~cm}^{2} / \mathrm{s}$ at the processing temperature. 


\textbf{Solution:}
\[
\begin{aligned}
&c(x, t)=A+B \text { erf } \frac{x}{2 \sqrt{D t}} ; c(0, t)=c_{s}=A ; c(x, 0)=c_{i}=0 \\
&c(\infty, t)=c_{i}=0=A+B \rightarrow A=-B \\
&\therefore c(x, t)=c_{s}-c_{s} \operatorname{erf} \frac{x}{2 \sqrt{D t}}=c_{s} \operatorname{erfc} \frac{x}{2 \sqrt{D t}} \rightarrow 5 \times 10^{16}=c_{s} \text { erfc } \frac{25 \times 10^{-4}}{2 \sqrt{7.23 \times 10^{-9} \times 90 \times 60}} \\
&\therefore c_{s}=\frac{5 \times 10^{16}}{\operatorname{erfc} \frac{25 \times 10^{-4}}{2 \sqrt{7.23 \times 10^{-9} \times 5400}}}=6.43 \times 10^{16} \mathrm{~cm}^{-3} \\
&\operatorname{erfc}(0.20)=1-\operatorname{erf}(0.20)=1-0.2227=\boxed{0.7773}
\end{aligned}
\]


\textbf{Problem:}
Subproblem 0: An electron beam strikes a crystal of cadmium sulfide (CdS). Electrons scattered by the crystal move at a velocity of $4.4 \times 10^{5} \mathrm{~m} / \mathrm{s}$. Calculate the energy of the incident beam. Express your result in eV, and as an integer. CdS is a semiconductor with a band gap, $E_{g}$, of $2.45$ eV.


\textbf{Solution:}
\includegraphics[scale=0.5]{set_18_img_01.jpg}
\nonessentialimage
\[
\begin{aligned}
&E_{\text {incident } e^{-}}=E_{\text {emitted } \mathrm{v}}+E_{\text {scattered } e^{-}}=E_{g}+\frac{\mathrm{mv}^{2}}{2} \\
&=2.45 \mathrm{eV}+\frac{1}{2} \times \frac{9.11 \times 10^{-31} \mathrm{~kg} \times\left(4.4 \times 10^{5} \mathrm{~m} / \mathrm{s}\right)^{2}}{1.6 \times 10^{-19} \mathrm{eV} / \mathrm{J}} \\
&=2.45 \mathrm{eV}+0.55 \mathrm{eV}=\boxed{3} \mathrm{eV}
\end{aligned}
\]


\textbf{Problem:}
Subproblem 0: Determine the inter-ionic equilibrium distance in meters between the sodium and chlorine ions in a sodium chloride molecule knowing that the bond energy is $3.84 \mathrm{eV}$ and that the repulsive exponent is 8. Please format your answer as $n \times 10^x$ where $n$ is to 1 decimal place.


Solution: $\mathrm{E}_{\mathrm{equ}}=-3.84 \mathrm{eV}=-3.84 \times 1.6 \times 10^{-19} \mathrm{~J}=-\frac{\mathrm{e}^{2}}{4 \pi \varepsilon_{0} r_{0}}\left(1-\frac{1}{\mathrm{n}}\right)$
\\
$r_{0}=\frac{\left(1.6 \times 10^{-19}\right)^{2}}{4 \pi 8.85 \times 10^{-12} \times 6.14 \times 10^{-19}}\left(1-\frac{1}{8}\right)= 
\boxed{3.3e-10} \mathrm{~m}$

Final answer: The final answer is 3.3e-10. I hope it is correct.

Subproblem 1: At the equilibrium distance, how much (in percent) is the contribution to the attractive bond energy by electron shell repulsion?


\textbf{Solution:}
Shell "repulsion" obviously constitutes a "negative" contribution to the bond energy. Looking at the energy equation we find:
\[
\begin{array}{ll}
\text { the attractive term as: } & -E \times(1)=-E \\
\text { the repulsion term as: } & -E \times(-1 / n)=E / n=E / 8
\end{array}
\]
The contribution to the bond energy by the repulsion term $=1 / 8 \times 100 = \boxed{12.5}\%$ Since the bond energy is negative, the $12.5 \%$ constitute a reduction in bond strength.


\section{Principles of Microeconomics (14.01 Fall 2011)}

\textbf{Problem:}
Preamble: A consumer's preferences are representable by the following utility function:
\[
  u(x, y)=x^{\frac{1}{2}}+y
\]

Subproblem 0: Obtain the marginal rate of substitution of the consumer at an arbitrary point $(X,Y)$, where $X>0$ and $Y>0$.


\textbf{Solution:}
\[ M R S=-\frac{\frac{1}{2} x^{-\frac{1}{2}}}{1}=\boxed{-\frac{1}{2} X^{-\frac{1}{2}}} \]


\textbf{Problem:}
Preamble: Xiaoyu spends all her income on statistical software $(S)$ and clothes (C). Her preferences can be represented by the utility function: $U(S, C)=4 \ln (S)+6 \ln (C)$.

Subproblem 0: Compute the marginal rate of substitution of software for clothes.


\textbf{Solution:}
We have that $M R S=\frac{\frac{4}{S}}{\frac{6}{C}}=\boxed{\frac{2}{3} \frac{C}{S}}$. 


\textbf{Problem:}
Subproblem 0: What algebraic condition describes a firm that is at an output level that maximizes its profits, given its capital in the short-term?  Use standard acronyms in your condition.


\textbf{Solution:}
The required condition is \boxed{MR=SRMC}, or marginal revenue is equal to short-run marginal cost.


\textbf{Problem:}
Preamble: Moldavia is a small country that currently trades freely in the world barley market. Demand and supply for barley in Moldavia is governed by the following schedules:
Demand: $Q^{D}=4-P$
Supply: $Q^{S}=P$
The world price of barley is $\$ 1 /$ bushel.

Subproblem 0: Calculate the free trade equilibrium price of barley in Moldavia, in dollars per bushel. 


Solution: In free trade, Moldavia will import barley because the world price of $\$ 1 /$ bushel is lower than the autarkic price of $\$ 2$ /bushel. Free trade equilibrium price will be \boxed{1} dollar per bushel.

Final answer: The final answer is 1. I hope it is correct.

Subproblem 1: Calculate the free trade equilibrium quantity of barley in Moldavia (in bushels). 


\textbf{Solution:}
In free trade, Moldavia will import barley because the world price of $\$ 1 /$ bushel is lower than the autarkic price of $\$ 2$ /bushel. Free trade equilibrium quantity will be \boxed{3} bushels, of which 1 is produced at home and 2 are imported. 


\textbf{Problem:}
Preamble: Consider the market for apple juice. In this market, the supply curve is given by $Q_{S}=$ $10 P_{J}-5 P_{A}$ and the demand curve is given by $Q_{D}=100-15 P_{J}+10 P_{T}$, where $J$ denotes apple juice, $A$ denotes apples, and $T$ denotes tea.

Subproblem 0: Assume that $P_{A}$ is fixed at $\$ 1$ and $P_{T}=5$. Calculate the equilibrium price in the apple juice market.


Solution: We have the system of equations $Q=10 P_{J}-5 \cdot 1$ and $Q=100-15 P_{J}+10 \cdot 5$. Solving for $P_{J}$ we get that $P_{J}=\boxed{6.2}$.

Final answer: The final answer is 6.2. I hope it is correct.

Subproblem 1: Assume that $P_{A}$ is fixed at $\$ 1$ and $P_{T}=5$. Calculate the equilibrium quantity in the apple juice market.


\textbf{Solution:}
We have the system of equations $Q=10 P_{J}-5 \cdot 1$ and $Q=100-15 P_{J}+10 \cdot 5$. Solving for $Q$ we get that $Q=\boxed{57}$.


\textbf{Problem:}
Preamble: Suppose, in the short run, the output of widgets is supplied by 100 identical competitive firms, each having a cost function:
\[
c_{s}(y)=\frac{1}{3} y^{3}+2
\]
The demand for widgets is given by:
\[
y^{d}(p)=6400 / p^{\frac{1}{2}}
\]

Subproblem 0: Obtain the short run industry supply function for widgets.


Solution: Since $P=M C=y^{2}$, the supply function of each firm is given by $y_{i}^{s}=p^{\frac{1}{2}}$. 
The industry supply function is $y^{s}(p)=100 y_{i}^{s}(p)=\boxed{100 p^{\frac{1}{2}}}$.

Final answer: The final answer is 100 p^{\frac{1}{2}}. I hope it is correct.

Subproblem 1: Obtain the short run equilibrium price of widgets.


Solution: $y^{s}=y^{d} \longrightarrow 100 p^{\frac{1}{2}}=\frac{6400}{p^{\frac{1}{2}}} \longrightarrow p=\boxed{64}$. 

Final answer: The final answer is 64. I hope it is correct.

Subproblem 2: Obtain the the output of widgets supplied by each firm.


\textbf{Solution:}
$y^{s}=y^{d} \longrightarrow 100 p^{\frac{1}{2}}=\frac{6400}{p^{\frac{1}{2}}} \longrightarrow p=64$. Hence $y^{*}=100 \cdot 8=800$ and $y_{i}=\boxed{8}.$


\textbf{Problem:}
Preamble: Sebastian owns a coffee factory in Argentina. His production function is:
\[
F(K, L)=(K-1)^{\frac{1}{4}} L^{\frac{1}{4}}
\]
Consider the cost of capital to be $r$ and the wage to be $w$. Both inputs are variable, and Sebastian faces no fixed costs.

Subproblem 0: What is the marginal rate of technical substitution of labor for capital?


\textbf{Solution:}
\[
M R T S=\frac{M P_{L}}{M P_{K}}=\boxed{\frac{K-1}{L}}
\]


\textbf{Problem:}
Preamble: There are two algebraic conditions describing a firm that is at a capital level that minimizes its costs in the long-term.

Subproblem 0: Write the condition which involves the SRAC, or short-run average cost?


\textbf{Solution:}
\boxed{SRAC=LRAC}, short-run average cost equals long-run average cost.


\textbf{Problem:}
Preamble: There are two algebraic conditions describing a firm that is at a capital level that minimizes its costs in the long-term.

Subproblem 0: Write the condition which involves the SRAC, or short-run average cost?


Solution: \boxed{SRAC=LRAC}, short-run average cost equals long-run average cost.

Final answer: The final answer is SRAC=LRAC. I hope it is correct.

Subproblem 1: Write the condition which involves SRMC, or short-run marginal cost?


\textbf{Solution:}
\boxed{SRMC=LRMC}, or short-run marginal cost equals long-run levels.


\textbf{Problem:}
Preamble: Suppose, in the short run, the output of widgets is supplied by 100 identical competitive firms, each having a cost function:
\[
c_{s}(y)=\frac{1}{3} y^{3}+2
\]
The demand for widgets is given by:
\[
y^{d}(p)=6400 / p^{\frac{1}{2}}
\]

Subproblem 0: Obtain the short run industry supply function for widgets.


\textbf{Solution:}
Since $P=M C=y^{2}$, the supply function of each firm is given by $y_{i}^{s}=p^{\frac{1}{2}}$. 
The industry supply function is $y^{s}(p)=100 y_{i}^{s}(p)=\boxed{100 p^{\frac{1}{2}}}$.


\textbf{Problem:}
Preamble: Moldavia is a small country that currently trades freely in the world barley market. Demand and supply for barley in Moldavia is governed by the following schedules:
Demand: $Q^{D}=4-P$
Supply: $Q^{S}=P$
The world price of barley is $\$ 1 /$ bushel.

Subproblem 0: Calculate the free trade equilibrium price of barley in Moldavia, in dollars per bushel. 


\textbf{Solution:}
In free trade, Moldavia will import barley because the world price of $\$ 1 /$ bushel is lower than the autarkic price of $\$ 2$ /bushel. Free trade equilibrium price will be \boxed{1} dollar per bushel.


\textbf{Problem:}
Preamble: Suppose, in the short run, the output of widgets is supplied by 100 identical competitive firms, each having a cost function:
\[
c_{s}(y)=\frac{1}{3} y^{3}+2
\]
The demand for widgets is given by:
\[
y^{d}(p)=6400 / p^{\frac{1}{2}}
\]

Subproblem 0: Obtain the short run industry supply function for widgets.


Solution: Since $P=M C=y^{2}$, the supply function of each firm is given by $y_{i}^{s}=p^{\frac{1}{2}}$. 
The industry supply function is $y^{s}(p)=100 y_{i}^{s}(p)=\boxed{100 p^{\frac{1}{2}}}$.

Final answer: The final answer is 100 p^{\frac{1}{2}}. I hope it is correct.

Subproblem 1: Obtain the short run equilibrium price of widgets.


\textbf{Solution:}
$y^{s}=y^{d} \longrightarrow 100 p^{\frac{1}{2}}=\frac{6400}{p^{\frac{1}{2}}} \longrightarrow p=\boxed{64}$. 


\textbf{Problem:}
Preamble: A consumer's preferences are representable by the following utility function:
\[
  u(x, y)=x^{\frac{1}{2}}+y
\]

Subproblem 0: Obtain the marginal rate of substitution of the consumer at an arbitrary point $(X,Y)$, where $X>0$ and $Y>0$.


Solution: \[ M R S=-\frac{\frac{1}{2} x^{-\frac{1}{2}}}{1}=\boxed{-\frac{1}{2} X^{-\frac{1}{2}}} \]

Final answer: The final answer is -\frac{1}{2} X^{-\frac{1}{2}}. I hope it is correct.

Subproblem 1: Suppose the price of the second good $(y)$ is 1 , and the price of the first good $(x)$ is denoted by $p>0$. If the consumer's income is $m>\frac{1}{4p}$, in the optimal consumption bundle of the consumer (in terms of $m$ and $p$ ), what is the quantity of the first good $(x)$?


\textbf{Solution:}
The consumer solves $\max x^{\frac{1}{2}}+y$ so that $p x+y=m$. We look for stationary values of the Lagrangian $L=x^{\frac{1}{2}}+y+\lambda(m-p x-y)$. The first-order conditions for stationarity are
\[
  \begin{aligned}
    &\frac{\partial L}{\partial x}=\frac{1}{2} x^{-\frac{1}{2}}-\lambda p=0 \\
    &\frac{\partial L}{\partial y}=1-\lambda=0 \\
    &\frac{\partial L}{\partial \lambda}=m-p x-y=0
  \end{aligned}
\]
Combining the first two equations above gives $\frac{1}{2 x^{\frac{1}{2}}}=p$, or $x^{*}=\frac{1}{4 p^{2}}$. Substituting $x^{*}$ into the budget constraint gives $y=m-p x^{*}=m-\frac{1}{4 p}$.
Case 1) $m \geq \frac{1}{4 p} \longrightarrow x^{*}=\frac{1}{4 p^{2}}$ and $y=m-\frac{1}{4 p} \geq 0$.
Case 2) $m \leq \frac{1}{4 p} \longrightarrow x^{*}=\frac{m}{p}$ and $y=0$.
Since we know $m>\frac{1}{4p}$, we use case 1, in which case our optimal consumption bundle $(x*,y*)$ is $(\frac{1}{4p^2},m-\frac{1}{4p})$.  So the answer is $\boxed{\frac{1}{4p^2}}$.


\textbf{Problem:}
Preamble: Consider the market for apple juice. In this market, the supply curve is given by $Q_{S}=$ $10 P_{J}-5 P_{A}$ and the demand curve is given by $Q_{D}=100-15 P_{J}+10 P_{T}$, where $J$ denotes apple juice, $A$ denotes apples, and $T$ denotes tea.

Subproblem 0: Assume that $P_{A}$ is fixed at $\$ 1$ and $P_{T}=5$. Calculate the equilibrium price in the apple juice market.


\textbf{Solution:}
We have the system of equations $Q=10 P_{J}-5 \cdot 1$ and $Q=100-15 P_{J}+10 \cdot 5$. Solving for $P_{J}$ we get that $P_{J}=\boxed{6.2}$.


\textbf{Problem:}
Preamble: In Cambridge, shoppers can buy apples from two sources: a local orchard, and a store that ships apples from out of state. The orchard can produce up to 50 apples per day at a constant marginal cost of 25 cents per apple. The store can supply any remaining apples demanded, at a constant marginal cost of 75 cents per unit. When apples cost 75 cents per apple, the residents of Cambridge buy 150 apples in a day.

Subproblem 0: Assume that the city of Cambridge sets the price of apples within its borders. What price should it set, in cents?


\textbf{Solution:}
The city should set the price of apples to be $\boxed{75}$ cents since that is the marginal cost when residents eat at least 50 apples a day, which they do when the price is 75 cents or less. 


\textbf{Problem:}
Preamble: You manage a factory that produces cans of peanut butter. The current market price is $\$ 10 /$ can, and you know the following about your costs (MC stands for marginal cost, and ATC stands for average total cost):
\[
\begin{array}{l}
MC(5)=10 \\
ATC(5)=6 \\
MC(4)=4 \\
ATC(4)=4
\end{array}
\]

Subproblem 0: A case of food poisoning breaks out due to your peanut butter, and you lose a lawsuit against your company. As punishment, Judge Judy decides to take away all of your profits, and considers the following two options to be equivalent:
i. Pay a lump sum in the amount of your profits.
ii. Impose a tax of $\$\left[P-A T C\left(q^{*}\right)\right]$ per can since that is your current profit per can, where $q^{*}$ is the profit maximizing output before the lawsuit.
How much is the tax, in dollars per can?


\textbf{Solution:}
You maximize profits where $P=M C$, and since $P=10=M C(5)$ you would set $q^{*}=5$.
\[
\pi / q=(P-A T C)=(10-6)=4
\]
The tax would be $\$ \boxed{4} /$ can.


\textbf{Problem:}
Preamble: Suppose there are exactly two consumers (Albie and Bubbie) who demand strawberries. Suppose that Albie's demand for strawberries is given by
\[
q_{a}(p)=p^{\alpha} f_{a}\left(I_{a}\right)
\]
and Bubbie's demand is given by
\[
q_{b}(p)=p^{\beta} f_{b}\left(I_{b}\right)
\]
where $I_{a}$ and $I_{b}$ are Albie and Bubbie's incomes, and $f_{a}(\cdot)$ and $f_{b}(\cdot)$ are two unknown functions.

Subproblem 0: Find Albie's (own-price) elasticity of demand, $\epsilon_{q_{a}, p}$. Use the sign convention that $\epsilon_{y, x}=\frac{\partial y}{\partial x} \frac{x}{y}$.


\textbf{Solution:}
\[
\epsilon_{q_{a}, p}=\frac{\partial q_{a}}{\partial p} \frac{p}{q_{a}(p)}=\left[\alpha p^{\alpha-1} f_{a}\left(I_{a} s\right)\right] \frac{p}{p^{\alpha} f_{a}\left(I_{a}\right)}=\boxed{\alpha}
\]


\textbf{Problem:}
Preamble: You have been asked to analyze the market for steel. From public sources, you are able to find that last year's price for steel was $\$ 20$ per ton. At this price, 100 million tons were sold on the world market. From trade association data you are able to obtain estimates for the own price elasticities of demand and supply on the world markets as $-0.25$ for demand and $0.5$ for supply. Assume that steel has linear demand and supply curves throughout, and that the market is competitive.

Subproblem 0: Solve for the equations of demand in this market.  Use $P$ to represent the price of steel in dollars per ton, and $X_{d}$ to represent the demand in units of millions of tons.


\textbf{Solution:}
Assume that this is a competitive market and assume that demand and supply are linear. Thus, $X_{d}=a-b P$ and $X_{s}=c+d P$. We know from the equation for own-price elasticity of demand that
\[
E_{Q_{X} P_{X}}=\frac{d X_{d}}{d P_{X}} \frac{P_{X}}{X_{d}}=-b \frac{P_{X}}{X_{d}}=-b \frac{20}{100}=-0.25
\]
Solving for $b$, then, we have $b=1.25$. Substituting back into the equation for demand, $X_{d}=$ $a-1.25 P$ or $100=a-1.25(20)$. Solving for $a$ we have $a=125$. Hence, the equation for last year's demand is $\boxed{X_{d}=125-1.25 P}$.


\section{Physical Chemistry (5.61 Fall 2017}

\textbf{Problem:}
Subproblem 0: Harmonic Oscillator Subjected to Perturbation by an Electric Field: An electron is connected by a harmonic spring to a fixed point at $x=0$. It is subject to a field-free potential energy
\[
V(x)=\frac{1}{2} k x^{2} .
\]
The energy levels and eigenstates are those of a harmonic oscillator where
\[
\begin{aligned}
\omega &=\left[k / m_{e}\right]^{1 / 2} \\
E_{v} &=\hbar \omega(v+1 / 2) \\
\psi_{v}(x) &=(v !)^{-1 / 2}\left(\hat{\boldsymbol{a}}^{\dagger}\right)^{v} \psi_{v=0}(x) .
\end{aligned}
\]
Now a constant electric field, $E_{0}$, is applied and $V(x)$ becomes
\[
V(x)=\frac{1}{2} k x^{2}+E_{0} e x \quad(e>0 \text { by definition }) .
\]
Write an expression for the energy levels $E_{v}$ as a function of the strength of the electric field.


\textbf{Solution:}
The total potential, including the interaction with the electric field is
\[
V(x)=\frac{m \omega^{2}}{2} x^{2}+E_{0} e x .
\]
We find its minimum to be
\[
\begin{aligned}
\frac{d V}{d x}=m \omega^{2} x &+E_{0} e=0 \\
\Rightarrow x_{\min } &=\frac{E_{0} e}{m \omega^{2}}, \\
V\left(x_{\min }\right) &=\frac{m \omega^{2}}{2} \frac{E_{0}^{2} e^{2}}{m^{2} \omega^{2}}-\frac{E_{0}^{2} e^{2}}{m \omega^{2}} \\
&=\frac{E_{0}^{2} e^{2}}{2 m \omega^{2}} .
\end{aligned}
\]
Defining the displacement from the minimum $x^{\prime}=x-x_{\min }$, we arrive at
\[
\begin{aligned}
V\left(x^{\prime}\right) &=\frac{m \omega^{2}}{2}\left(x^{\prime}-\frac{E_{0} e}{m \omega^{2}}\right)^{2}+E_{0} e\left(x^{\prime}-\frac{E_{0} e}{m \omega^{2}}\right) \\
&=\frac{m \omega^{2}}{2} x^{\prime 2}-\frac{E_{0}^{2} e^{2}}{2 m \omega^{2}}
\end{aligned}
\]
Thus, we see that the system is still harmonic! All we have done is to shift the minimum position and minimum energy, but the potential is still quadratic. The harmonic frequency $\omega$ remains unchanged.
Since the potential now is a harmonic oscillator with frequency $\omega$ and a constant offset, we can easily write down the energy levels:
\[
E_{v}=\boxed{\hbar \omega(v+1 / 2)-\frac{E_{0}^{2} e^{2}}{2 m \omega^{2}}}
\]


\textbf{Problem:}
Preamble: The following concern the independent particle model. You may find the following set of Coulomb and exchange integrals useful (energies in $\mathrm{eV}$):
$\mathrm{J}_{1 s 1 s}=17.0 Z$ 
$\mathrm{~J}_{1 s 2 s}=4.8 Z$ 
$\mathrm{~K}_{1 s 2 s}=0.9 Z$ 
$\mathrm{~J}_{2 s 2 s}=3.5 Z$ 
$\mathrm{J}_{1 s 2 p}=6.6 Z$ 
$\mathrm{~K}_{1 s 2 p}=0.5 Z$ 
$\mathrm{~J}_{2 s 2 p}=4.4 Z$ 
$\mathrm{~K}_{2 s 2 p}=0.8 Z$ 
$\mathrm{J}_{2 p_{i}, 2 p_{i}}=3.9 Z$
$\mathrm{~J}_{2 p_{i}, 2 p_{k}}=3.5 Z$
$\mathrm{~K}_{2 p_{i}, 2 p_{k}}=0.2 Z i \neq k$ 

Subproblem 0: Using the independent particle model, what is the energy difference between the $1 s^{2} 2 p_{x}^{2}$ configuration and the $1 s^{2} 2 s^{2}$ configuration? Give your answer in eV, in terms of $Z$, and round to a single decimal place.


\textbf{Solution:}
We are asked to calculate the energy difference between a $1 s^{2} 2 p_{x}^{2}$ and a $1 s^{2} 2 s^{2}$ configuration. Let's compute the energy for each using the independent particle model
\[
\begin{aligned}
E\left[1 s^{2} 2 p_{x}^{2}\right]=& \sum_{i} E_{i}+\sum_{i, j}^{i>j} \widetilde{J}_{i j}-\widetilde{K}_{i j} \\
=& 2 E_{1 s}+2 E_{2 p} \\
&+\widetilde{J}_{1 s \alpha, 1 s \beta}+\widetilde{J}_{1 s \alpha, 2 p_{x} \alpha}+\widetilde{J}_{1 s \alpha, 2 p_{x} \beta}+\widetilde{J}_{1 s \beta, 2 p_{x} \alpha}+\widetilde{J}_{1 s \beta, 2 p_{x} \beta}+\widetilde{J}_{2 p_{x} \alpha, 2 p_{x} \beta} \\
&-\widetilde{K}_{1 s \alpha, 1 s \beta}-\widetilde{K}_{1 s \alpha, 2 p_{x} \alpha}-\widetilde{K}_{1 s \alpha, 2 p_{x} \beta}-\widetilde{K}_{1 s \beta, 2 p_{x} \alpha}-\widetilde{K}_{1 s \beta, 2 p_{x} \beta}-\widetilde{K}_{2 p_{x} \alpha, 2 p_{x} \beta} \\
=& 2 E_{1 s}+2 E_{2 p}+J_{1 s, 1 s}+4 J_{1 s, 2 p}+J_{2 p_{i}, 2 p_{i}}-2 K_{1 s, 2 p} \\
E\left[1 s^{2} 2 s^{2}\right]=& 2 E_{1 s}+2 E_{2 s}+J_{1 s, 1 s}+4 J_{1 s, 2 s}+J_{2 s, 2 s}-2 K_{1 s, 2 s} \\
\Rightarrow \Delta E=& 4\left(J_{1 s, 2 p}-J_{1 s, 2 s}\right)+\left(J_{2 p_{i}, 2 p_{i}}-J_{2 s, 2 s}\right)-2\left(K_{1 s, 2 p}-K_{1 s, 2 s}\right) \\
=& Z[4(6.6-4.8)-(3.9-3.5)-2(0.5-0.9)] \\
=&+\boxed{7.6 Z} \mathrm{eV}
\end{aligned}
\]


\textbf{Problem:}
Preamble: A pulsed Nd:YAG laser is found in many physical chemistry laboratories.

Subproblem 0: For a $2.00 \mathrm{~mJ}$ pulse of laser light, how many photons are there at $1.06 \mu \mathrm{m}$ (the Nd:YAG fundamental) in the pulse?  PAnswer to three significant figures.


\textbf{Solution:}
For $1.06 \mu \mathrm{m}$ Light
Energy of one photon $=E_{p}=h \nu ; \nu=c / \lambda ; E_{p}=h c / \lambda$
\[
\begin{aligned}
\lambda &=1.06 \mu \mathrm{m}=1.06 \times 10^{-6} \mathrm{~m} \\
c &=3 \times 10^{8} \mathrm{~m} / \mathrm{s} \\
h &=\text { Planck's constant }=6.626 \times 10^{-34} \mathrm{~kg} \mathrm{} \mathrm{m}^{2} / \mathrm{s}
\end{aligned}
\]
$E_{p}=1.88 \times 10^{-19} \mathrm{~J}$
$1.88 \times 10^{-19} \mathrm{~J} /$ photon, we want photons/pulse.
\[
\frac{1}{1.88 \times 10^{19} \mathrm{~J} / \text { photon }} \times \frac{2.00 \times 10^{-3}}{\text { pulse }}=\boxed{1.07e16} \mathrm{photons} / \mathrm{pulse}
\]


\textbf{Problem:}
Subproblem 0: Given that the work function of chromium is $4.40 \mathrm{eV}$, calculate the kinetic energy of electrons in Joules emitted from a clean chromium surface that is irradiated with ultraviolet radiation of wavelength $200 \mathrm{~nm}$.


\textbf{Solution:}
The chromium surface is irradiated with $200 \mathrm{~nm}$ UV light. These photons have energy
\[
\begin{aligned}
E &=\frac{h c}{\lambda}=\frac{\left(6.626 \times 10^{34} \mathrm{~J} \cdot \mathrm{s}\right)\left(3 \times 10^{8} \mathrm{~m} \cdot \mathrm{s}^{-1}\right)}{200 \times 10^{-9} \mathrm{~m}} \\
&=9.94 \times 10^{-19} \mathrm{~J} \\
&=6.20 \mathrm{eV}
\end{aligned}
\]
The photo-ejected electron has kinetic energy
\[
K E=E_{\text {photon }}-\phi_{o}=6.20 \mathrm{eV}-4.40 \mathrm{eV}=1.80 \mathrm{eV}=\boxed{2.88e-19} \mathrm{~J}
\]


\textbf{Problem:}
Subproblem 0: Compute the momentum of one $500 \mathrm{~nm}$ photon using $p_{\text {photon }}=E_{\text {photon }} / c$ where $c$ is the speed of light, $c=3 \times 10^{8} \mathrm{~m} / \mathrm{s}$, and $\nu=c / \lambda$.  Express your answer in kilogram meters per second, rounding your answer to three decimal places.


\textbf{Solution:}
\[
\begin{aligned}
p_{\text {proton }} &=E_{\text {proton }} / c \\
p &=\text { Momentum } \\
E &=\text { Energy }=h \nu \\
c &=\text { Speed of light, } 3 \times 10^{8} \mathrm{~m} / \mathrm{s}
\end{aligned}
\]
\[
\begin{aligned}
& p_{\mathrm{PH}}=\frac{h \nu}{c} \quad \nu=c / \lambda \\
& p_{\mathrm{PH}}=h / \lambda(\lambda \text { in meters }), 500 \mathrm{~nm}=500 \times 10^{-9} \mathrm{~m} \\
& p_{\mathrm{PH}}=h / 500 \times 10^{-9}=6.626 \times 10^{-34} / 500 \times 10^{-9}=\boxed{1.325e-27} \mathrm{~kg} \mathrm{~m} / \mathrm{s}
\end{aligned}
\]


\textbf{Problem:}
Preamble: This problem deals with the H\"uckel MO theory of $\pi$-conjugated systems.
To answer each question, you will need to construct the Hückel MOs for each of the molecules pictured, divide them into sets of occupied and unoccupied orbitals, and determine the relevant properties, such as ground state energy, bond order, etc.
NOTE: For all parts we take $\alpha=\alpha_{\mathrm{C}}=-11.2 \mathrm{eV}$ and $\beta=\beta_{\mathrm{CC}}=-0.7 \mathrm{eV}$.

Subproblem 0: Determine the ionization potential of benzene (remember, ionization potential $\left[\mathrm{IP}=\mathrm{E}\left(\mathrm{B}^{+}\right)-\mathrm{E}(\mathrm{B})\right]$), in $\mathrm{eV}$, rounded to one decimal place.  The benzene molecule is shown below:
\chemfig{C*6((-H)-C(-H)=C(-H)-C(-H)=C(-H)-C(-H)=)} 


\textbf{Solution:}
Let's build the Hückel MO Hamiltonian from the 6 carbon atoms.  The differences between benzene and hexatriene are only connectivity:
\[
H_{\text {benzene }}=\left(\begin{array}{cccccc}
\alpha & \beta & 0 & 0 & 0 & \beta \\
\beta & \alpha & \beta & 0 & 0 & 0 \\
0 & \beta & \alpha & \beta & 0 & 0 \\
0 & 0 & \beta & \alpha & \beta & 0 \\
0 & 0 & 0 & \beta & \alpha & \beta \\
\beta & 0 & 0 & 0 & \beta & \alpha
\end{array}\right)
\]
We now substitute $\alpha$ and $\beta$ with the values above and find the eigenvalues of the Hamiltonian numerically. The eigenvalues of $\mathrm{H}_{\text {benzene }}$ (in $\mathrm{eV}$ ) are
\[
E^{\mu}=\{-12.6,-11.9,-11.9,-10.5,-10.5,-9.8\}
\].
The ionization potential in this model is simply the energy of the HOMO of the ground state of each molecule (this is the orbital from which the electron is ejected). Since there are $6 \pi$-electrons, we can fill the three lowest MOs and the HOMO will be the third lowest. Therefore, the IP of benzene is $\boxed{11.9} \mathrm{eV}$


\textbf{Problem:}
Subproblem 0: A baseball has diameter $=7.4 \mathrm{~cm}$. and a mass of $145 \mathrm{~g}$. Suppose the baseball is moving at $v=1 \mathrm{~nm} /$ second. What is its de Broglie wavelength
\[
\lambda=\frac{h}{p}=\frac{h}{m \nu}
\]
?  Give answer in meters.


\textbf{Solution:}


\[
\begin{aligned}
D_{\text {ball }} &=0.074 m \\
m_{\text {ball }} &=0.145 \mathrm{~kg} \\
v_{\text {ball }} &=1 \mathrm{~nm} / \mathrm{s}=1 \times 10^{-9} \mathrm{~m} / \mathrm{s}
\end{aligned}
\]
Using de Broglie:
\[
\lambda_{\text {ball }}=\frac{h}{p}=\frac{h}{m \nu}=\frac{6.626 \times 10^{-34} \mathrm{~m}^{2} \mathrm{~kg} / \mathrm{s}}{0.145 \mathrm{~kg} \cdot 1 \times 10^{-9} \mathrm{~m} / \mathrm{s}}=\boxed{4.6e-24} \mathrm{~m}=\lambda_{\text {ball }}
\]


\textbf{Problem:}
Preamble: Consider the Particle in an Infinite Box ``superposition state'' wavefunction,
\[
\psi_{1,2}=(1 / 3)^{1 / 2} \psi_{1}+(2 / 3)^{1 / 2} \psi_{2}
\]
where $E_{1}$ is the eigen-energy of $\psi_{1}$ and $E_{2}$ is the eigen-energy of $\psi_{2}$.

Subproblem 0: Suppose you do one experiment to measure the energy of $\psi_{1,2}$.  List the possible result(s) of your measurement.


Solution: Since the only eigenergies are $E_{1}$ and $E_{2}$, the possible outcomes of the measurement are $\boxed{E_{1},E_{2}}$.

Final answer: The final answer is E_{1},E_{2}. I hope it is correct.

Subproblem 1: Suppose you do many identical measurements to measure the energies of identical systems in state $\psi_{1,2}$. What average energy will you observe?


\textbf{Solution:}
\[
\langle E\rangle =\boxed{\frac{1}{3} E_{1}+\frac{2}{3} E_{2}}
\]
This value of $\langle E\rangle$ is between $E_{1}$ and $E_{2}$ and is the weighted average energy.


\textbf{Problem:}
Preamble: Consider the Particle in an Infinite Box ``superposition state'' wavefunction,
\[
\psi_{1,2}=(1 / 3)^{1 / 2} \psi_{1}+(2 / 3)^{1 / 2} \psi_{2}
\]
where $E_{1}$ is the eigen-energy of $\psi_{1}$ and $E_{2}$ is the eigen-energy of $\psi_{2}$.

Subproblem 0: Suppose you do one experiment to measure the energy of $\psi_{1,2}$.  List the possible result(s) of your measurement.


\textbf{Solution:}
Since the only eigenergies are $E_{1}$ and $E_{2}$, the possible outcomes of the measurement are $\boxed{E_{1},E_{2}}$.


\textbf{Problem:}
Preamble: Evaluate the following integrals for $\psi_{J M}$ eigenfunctions of $\mathbf{J}^{2}$ and $\mathbf{J}_{z}$. 

Subproblem 0: $\int \psi_{22}^{*}\left(\widehat{\mathbf{J}}^{+}\right)^{4} \psi_{2,-2} d \tau$


\textbf{Solution:}
\[
\begin{gathered}
\int \psi_{22}^{*}\left(\hat{J}_{+}\right)^{4} \psi_{2,-2} d \tau=\int \psi_{22}^{*} \sqrt{2(2+1)-(-2)(-2+1)}\left(\hat{J}_{+}\right)^{3} \psi_{2,-1} d \tau \\
=\int \psi_{22}^{*} \sqrt{2(2+1)-(-2)(-2+1)} \sqrt{2(2+1)-(-1)(-1+1)}\left(\hat{J}_{+}\right)^{2} \psi_{2,0} d \tau \\
=\int \psi_{22}^{*} \sqrt{2(2+1)-(-2)(-2+1)} \sqrt{2(2+1)-(-1)(-1+1)} \\
\times \sqrt{2(2+1)-(0)(0+1)}\left(\hat{J}_{+}\right) \psi_{2,1} d \tau \\
=\int \psi_{22}^{*} \sqrt{2(2+1)-(-2)(-2+1)} \sqrt{2(2+1)-(-1)(-1+1)} \\
\times \sqrt{2(2+1)-(0)(0+1)} \sqrt{2(2+1)-(1)(1+1)} \psi_{22} d \tau \\
=\sqrt{4} \times \sqrt{6} \times \sqrt{6} \times \sqrt{4} \int \psi_{22}^{*} \psi_{22} d \tau \\
=\boxed{24}
\end{gathered}
\]


\textbf{Problem:}
Preamble: Consider the 3-level $\mathbf{H}$ matrix
\[
\mathbf{H}=\hbar \omega\left(\begin{array}{ccc}
10 & 1 & 0 \\
1 & 0 & 2 \\
0 & 2 & -10
\end{array}\right)
\]
Label the eigen-energies and eigen-functions according to the dominant basis state character. The $\widetilde{10}$ state is the one dominated by the zero-order state with $E^{(0)}=10, \tilde{0}$ by $E^{(0)}=0$, and $-\widetilde{10}$ by $E^{(0)}=-10$ (we will work in units where $\hbar \omega = 1$, and can be safely ignored).

Subproblem 0: Use non-degenerate perturbation theory to derive the energy $E_{\widetilde{10}}$.  Carry out your calculations to second order in the perturbing Hamiltonian, and round to one decimal place.


\textbf{Solution:}
$E_{\widetilde{10}} = 10 + \frac{1^2}{10 - 0} = \boxed{10.1}.$


\end{document}